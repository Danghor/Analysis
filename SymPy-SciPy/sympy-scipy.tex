\documentclass{report}
\usepackage[latin1]{inputenc}
\usepackage{a4wide}
\usepackage{epsfig}
\usepackage{amssymb}
\usepackage{amsmath}
\usepackage{fancyvrb}
\usepackage{alltt}
\usepackage{fleqn}
\usepackage{epic}
\usepackage{color} 
\usepackage{theorem}
\usepackage{hyperref}
\usepackage[all]{hypcap}
\hypersetup{
        colorlinks = true, % comment this to make xdvi work
        linkcolor  = blue,
        citecolor  = red,
        filecolor  = [rgb]{0.1, 0.1, 1.0},
        urlcolor   = [rgb]{0.1, 0.1, 1.0},
        pdfborder  = {0 0 0} 
}

\usepackage{fancyhdr}
\usepackage{lastpage} 

\pagestyle{fancy}
\renewcommand*{\familydefault}{\sfdefault}

\fancyfoot[C]{--- \thepage/\pageref{LastPage}\ ---}

\fancypagestyle{plain}{%
\fancyhf{}
\fancyfoot[C]{--- \thepage/\pageref{LastPage}\ ---}
\renewcommand{\headrulewidth}{0pt}
}

\renewcommand{\chaptermark}[1]{\markboth{\chaptername \ \thechapter.\ #1}{}}
\renewcommand{\sectionmark}[1]{\markright{\thesection. \ #1}{}}
\fancyhead[R]{\leftmark}
\fancyhead[L]{\rightmark}

\definecolor{amethyst}{rgb}{0.2, 0.4, 0.6}
\definecolor{orange}{rgb}{1, 0.9, 0.0}

{\theorembodyfont{\sf}
\newtheorem{Definition}{Definition}
\newtheorem{Axiom}[Definition]{Axiom}
\newtheorem{Notation}[Definition]{Notation}
\newtheorem{Lemma}[Definition]{Lemma}
\newtheorem{Theorem}[Definition]{Theorem}
}

\newcommand{\proof}{\vspace*{0.2cm}

\noindent
\textbf{Proof}: }
 
\newcommand{\qed}{\hspace*{\fill} $\Box$
\vspace*{0.2cm}

}

\newcommand{\eod}{\hspace*{\fill} $\diamond$
\vspace*{0.2cm}

}

\newcommand{\eox}{\hspace*{\fill} $\diamond$
\vspace*{0.2cm}

}

\newcommand{\solution}{\vspace*{0.2cm}

\noindent
\textbf{Solution}: }

\newcounter{exercise}
\newcommand{\exercise}{\vspace*{0.2cm}
\stepcounter{exercise}

\noindent
\textbf{Exercise \arabic{exercise}}: }

\newcommand{\example}{\vspace*{0.2cm}

\noindent
\textbf{Example}: \ }

\newcommand{\examples}{\vspace*{0.2cm}

\noindent
\textbf{Examples}: \ }
 
\newcommand{\remark}{\vspace*{0.2cm}
\noindent
\textbf{Remark}: }

\newcommand{\lb}{\hspace*{\fill} \linebreak}

\newcommand{\bruch}[2]{\displaystyle\frac{\;\displaystyle#1\;}{\;\displaystyle#2\;}}
\newcommand{\bruchs}[2]{\textstyle\frac{\;\textstyle#1\;}{\;\textstyle#2\;}}
\newcommand{\bint}{\displaystyle\int}
\newcommand{\dint}[2]{\displaystyle\int_{#1}^{#2}\hspace{-0.2cm}}
\newcommand{\Oh}{\mathcal{O}}
\newcommand{\df}{\displaystyle\frac{d\;}{dx}}
\newcommand{\ds}{\displaystyle}
\newcommand{\norm}[1]{\big\|#1\bigr\|_{\infty}}

\def\pair(#1,#2){\langle #1, #2 \rangle}

\newlength{\mylength}
\setlength{\mathindent}{1.3cm}

\title{SymPy \& SciPy \\[0.3cm]
      --- An Appetizer --- } 
\author{Karl Stroetmann} 
\date{\today}
 
\begin{document}
\maketitle
\tableofcontents

\chapter{Preliminary Remarks}
This tutorial demonstrates some of the most important features of \href{http://sympy.org/en/index.html}{\textsl{SymPy}}
and \href{http://www.scipy.org}{\textsl{SciPy}}.  \textsl{SymPy} and \textsl{SciPy} are modules that extend the
programming language \href{http://www.python.org}{\textsl{Python}}.  
\begin{enumerate}
\item \textsl{SymPy} supports symbolic mathematics, i.e.~it can be used to solve equations symbolically, or for symbolic
      integration. 
\item \textsl{SciPy} implements a large number of numerical routines.
\end{enumerate}
Both of the packages can be used without prior knowledge of the programming language \textsl{Python}.
In this short paper we do not have the time to describe either \textsl{SymPy} or \textsl{SciPy} exhaustively.  Rather,
the idea is to arise the curiosity of the reader by describing some of the most interesting features.

\section{Installation}
This tutorial assumes that both \textsl{Python} and
\textsl{SymPy} have been installed.  \textsl{Python} can be installed  by following the instructions given on  
\\[0.2cm]
\hspace*{1.3cm}
\href{https://www.python.org/download/}{\texttt{https://www.python.org/download/}}.
\\[0.2cm]
The module \textsl{SymPy} can be found at
\\[0.2cm]
\hspace*{1.3cm}
\href{http://docs.sympy.org/latest/install.html}{\texttt{http://docs.sympy.org/latest/install.html}}
\\[0.2cm]
and the module \textsl{SciPy} is available at
\\[0.2cm]
\hspace*{1.3cm}
\href{http://www.scipy.org/install.html}{\texttt{http://www.scipy.org/install.html}}
\\[0.2cm]
Alternatively, \textsl{Python}, \textsl{SciPy}, and \textsl{SymPy} can be installed via 
\href{https://store.continuum.io/cshop/anaconda/}{\emph{Anaconda}}.  This is a lot simpler than installing
\textsl{Python}, \textsl{SciPy}, and \textsl{SymPy} separately.  The reason is that many systems (notably \textsl{Linux}
and \textsl{Mac OS X})  already have \textsl{Python} preinstalled, since a lot of other packages depend
on it.  Fiddling around with the preinstalled version of \textsl{Python} can compromise the
stability of the system.

\chapter{Introduction to \textsl{SymPy}}
\textsl{SymPy} is the module that supports symbolic computation.  In particular, \textsl{SymPy} is able to
\begin{enumerate}
\item perform symbolic differentiation and integration,
\item compute limits,
\item compute the Taylor series of a given function,
\item compute closed expressions for both finite sums and infinite  series,
\item solve ordinary equations, recurrence equations, and differential equations.
\end{enumerate}
We will demonstrate all these features in the following subsections.

To start \textsl{SymPy}, start a python interpreter by typing either the command \texttt{python} or \texttt{ipython} 
on the command line and issue the following command:
\\[0.2cm]
\hspace*{1.3cm}
\texttt{import sympy as sym}
\\[0.2cm]
Now all functions $f()$ defined in the package \texttt{sympy} can be accessed as
\\[0.2cm]
\hspace*{1.3cm}
\texttt{sym.$f$()}.
\\[0.2cm]
Alternatively, we can also import everything of \texttt{sympy} into the global namespace via the
following command :
\\[0.2cm]
\hspace*{1.3cm}
\texttt{from sympy import *}
\\[0.2cm]
In this short tutorial, we will use the second method for the sake of brevity.  This is a valid
approach as long as we only intend to use \textsl{SymPy} as a symbolic calculator.  If our intention
had been to develop complex software that is built on top of \textsl{SymPy}, it would be far better
to avoid the pollution of the namespace that is the consequence of importing everythong from
\texttt{sympy}.  In that case we really should use the \texttt{import} declaration presented first.

To get started, we need to define some symbolic variables.  The commands
\\[0.2cm]
\hspace*{1.3cm}
\texttt{x = symbols("x")} \\
\hspace*{1.3cm}
\texttt{y = symbols("y")}
\\[0.2cm]
define two new symbolic variables that have the names ``\texttt{x}'' and ``\texttt{y}''.  These
symbolic variables are assigned to the \textsl{Python} variables \texttt{x} and \texttt{y},
respectively.  There is a shorthand available to declare several symbolic variables simultaneously:
In order to declare both ``\texttt{x}'' and ``\texttt{y}'' as symbolic variables, we could have used
the following command:
\\[0.2cm]
\hspace*{1.3cm}
\texttt{x, y = symbols("x y")}
\\[0.2cm]
Let us verify the first Binomial formula using the variable \texttt{x} and \texttt{y}.  To do this,
we issue the command
\\[0.2cm]
\hspace*{1.3cm}
\texttt{expand((x + y) * (x + y))}
\\[0.2cm]
As the function \texttt{expand} tries to expand its argument, \textsl{SymPy} will respond with the expression
\\[0.2cm]
\hspace*{1.3cm}
\texttt{x**2 + 2*x*y + y**2}.
\\[0.2cm]
This shows that the equation
\\[0.2cm]
\hspace*{1.3cm}
$(x + y) \cdot (x + y) = x^2 + 2 \cdot x \cdot y + y^2$
\\[0.2cm]
is valid.  For a more impressive result, we issue the command
\\[0.2cm]
\hspace*{1.3cm}
\texttt{expand((x + y) ** 5}
\\[0.2cm]
The result is
\\[0.2cm]
\hspace*{1.3cm}
\texttt{x**5 + 5*x**4*y + 10*x**3*y**2 + 10*x**2*y**3 + 5*x*y**4 + y**5}.
\\[0.2cm]
Since \textsl{Python} has a \texttt{for}-loop, we can also compute Pascal's triangle in one fell
swoop by writing
\\[0.2cm]
\hspace*{1.3cm}
\texttt{for n in range(1, 7): expand((x+y)**n)}
\\[0.2cm]
Note that we have to press \texttt{return} two times when entering this command.  This is necessary
in order to notify \textsl{Python} that the \texttt{for} loop is finished.  The result is:
\begin{verbatim}
x + y
x**2 + 2*x*y + y**2
x**3 + 3*x**2*y + 3*x*y**2 + y**3
x**4 + 4*x**3*y + 6*x**2*y**2 + 4*x*y**3 + y**4
x**5 + 5*x**4*y + 10*x**3*y**2 + 10*x**2*y**3 + 5*x*y**4 + y**5
x**6 + 6*x**5*y + 15*x**4*y**2 + 20*x**3*y**3 + 15*x**2*y**4 + 6*x*y**5 + y**6
\end{verbatim}
The opposite of the command \texttt{expand} is the command \texttt{factor}.  Suppose we want to know
the solution of the equation
\\[0.2cm]
\hspace*{1.3cm}
$x^2 - 8 \cdot x + 15 = 0$
\\[0.2cm]
and we suspect that the solutions are, in fact, integers.  The command
\\[0.2cm]
\hspace*{1.3cm}
\texttt{factor(x ** 2 - 8 * x + 15)}
\\[0.2cm]
yields the result
\\[0.2cm]
\hspace*{1.3cm}
\texttt{(x - 5)*(x - 3)}
\\[0.2cm]
and thereby shows that
\\[0.2cm]
\hspace*{1.3cm}
$x^2 - 8 \cdot x + 15 = (x - 5) \cdot (x - 3)$.
\\[0.2cm]
We can also compute the \emph{partial fraction decomposition} of an expression.  For example, the
command  
\\[0.2cm]
\hspace*{1.3cm}
\texttt{apart(x/(x**2 -8*x + 15))}
\\[0.2cm]
yields the result
\\[0.2cm]
\hspace*{1.3cm}
\texttt{-3/(2*(x - 3)) + 5/(2*(x - 5))}
\\[0.2cm]
thereby showing that
\\[0.2cm]
\hspace*{1.3cm}
$\ds \frac{x}{x^2 - 8 \cdot x + 15} = \frac{5}{2} \cdot \frac{1}{x-5} - \frac{3}{2} \cdot \frac{1}{x-3}$.


\section{Differentiation and Integration}
In order to compute the derivative of
\\[0.2cm]
\hspace*{1.3cm}
$x \cdot \sin(x)$
\\[0.2cm]
with respect to $x$ we can write
\\[0.2cm]
\hspace*{1.3cm}
\texttt{diff(x * sin(x), x)}
\\[0.2cm]
\textsl{SymPy} will compute the result
\\[0.2cm]
\hspace*{1.3cm}
\texttt{x*cos(x) + sin(x)}.
\\[0.2cm]
Symbolic integration is possible, too:  To compute the indefinite integral
\\[0.2cm]
\hspace*{1.3cm}
$\displaystyle\int x \cdot \sin(x)\; dx$
\\[0.2cm]
we issue the command
\\[0.2cm]
\hspace*{1.3cm}
\texttt{integrate(x * sin(x), x)}
\\[0.2cm]
The result is
\\[0.2cm]
\hspace*{1.3cm}
\texttt{-x*cos(x) + sin(x)}.
\\[0.2cm]
\textsl{Sympy} can compute definite integrals.  To compute the integral
\\[0.2cm]
\hspace*{1.3cm}
$\displaystyle\int_{-\infty}^{+\infty} e^{-x^2} \; dx$
\\[0.2cm]
we issue the command
\\[0.2cm]
\hspace*{1.3cm}
\texttt{integrate(exp(-x ** 2), (x, -oo, oo))}
\\[0.2cm]
The result is
\\[0.2cm]
\hspace*{1.3cm}
\texttt{sqrt(pi)}.
\\[0.2cm]
This show that
\\[0.2cm]
\hspace*{1.3cm}
$\displaystyle\int_{-\infty}^{+\infty} e^{-x^2} \; dx = \sqrt{\pi}$.
\\[0.2cm]
Note that $\infty$ is represented as the variable \texttt{oo} in the module \texttt{sympy}.  

\section{Computation of Limits}
\textsl{SymPy} supports the computation of limits.  In order to verify that 
\\[0.2cm]
\hspace*{1.3cm}
$\lim\limits_{n \rightarrow \infty} \sqrt{n+1} - \sqrt{n} = 0$
\\[0.2cm]
we can issue the command
\\[0.2cm]
\hspace*{1.3cm}
\texttt{limit(sqrt(x+1) - sqrt(x), x, oo)}
\\[0.2cm]
\textsl{SmyPy} indeed returns \texttt{0} as the result.  As another example, the command
\\[0.2cm]
\hspace*{1.3cm}
\texttt{limit(sin(x)/x, x, 0)}
\\[0.2cm]
shows that
\\[0.2cm]
\hspace*{1.3cm}
$\lim\limits_{x \rightarrow 0} \bruch{\sin(x)}{x} = 0$.

\section{Taylor Series}
Many interesting functions can be approximated via a Taylor series.  If $f(x)$ can be written
as
\\[0.2cm]
\hspace*{1.3cm}
$\ds f(x) = \sum\limits_{n=0}^\infty a_n \cdot x^n$,
\\[0.2cm]
then the coefficients $a_n$ are given as
\\[0.2cm]
\hspace*{1.3cm}
$\ds a_n = \frac{f^{(n)}(x)}{n!}$
\\[0.2cm]
where $f^{(n)}(x)$ denotes the $n$-th derivative of $f$.  For example, the exponential function can
be written as
\\[0.2cm]
\hspace*{1.3cm}
$\ds\exp(x) = \sum\limits_{n=0}^\infty \frac{1}{n!} \cdot x^n$
\\[0.2cm]
\textsl{SymPy} can compute the Taylor series of a given function up to a given value of $n$.  For
example, to compute the 
Taylor series of the exponential function we can declare $x$ to be a variable and then use the command
\\[0.2cm]
\hspace*{1.3cm}
\texttt{series(exp(x), x)}.
\\[0.2cm]
In this case, \textsl{SymPy} returns
\\[0.2cm]
\hspace*{1.3cm}
\texttt{1 + x + x**2/2 + x**3/6 + x**4/24 + x**5/120 + O(x**6)},
\\[0.2cm]
showing the first 6 terms of the Taylor series.  In order to compute the terms up to and including
the term containing $x^{10}$ we have to use the command
\\[0.2cm]
\hspace*{1.3cm}
\texttt{series(exp(x), x, n = 11)}
\\[0.2cm]
In this case, \textsl{SymPy} returns the result
\\[0.2cm]
\hspace*{1.3cm}
\texttt{1 + x + x**2/2 + x**3/6 + x**4/24 + x**5/120 + x**6/720 + x**7/5040 +} \\
\hspace*{1.3cm}
\texttt{x**8/40320 + x**9/362880 + x**10/3628800 + O(x**11)}.


\section{Finite Sums and Infinite Series}
In order to compute an analytical expression for the sum
\\[0.2cm]
\hspace*{1.3cm}
$\ds\sum\limits_{i=1}^n i^3$
\\[0.2cm]
we can issue the following commands:
\\[0.2cm]
\hspace*{1.3cm}
\texttt{i = symbols("i")} \\  
\hspace*{1.3cm}
\texttt{n = symbols("n")} \\  
\hspace*{1.3cm}
\texttt{summation(i**3, (i, 1, n))}
\\[0.2cm]
The result is
\\[0.2cm]
\hspace*{1.3cm}
\texttt{n**4/4 + n**3/2 + n**2/4}.
\\[0.2cm]
Therefore, we have shown that
\\[0.2cm]
\hspace*{1.3cm}
$\ds\sum\limits_{i=1}^n i^3 = \frac{n^{4}}{4} + \frac{n^{3}}{2} + \frac{n^{2}}{4}$
\\[0.2cm]
holds.  
The command
\\[0.2cm]
\hspace*{1.3cm}
\texttt{summation(q**i, (i, 0, n))}
\\[0.2cm]
yields the result
\\[0.2cm]
\hspace*{1.3cm}
\texttt{Piecewise((n + 1, q == 1), ((-q**(n + 1) + 1)/(-q + 1), True))}
\\[0.2cm]
which shows that
\\[0.2cm]
\hspace*{1.3cm}
$\ds\sum\limits_{i=0}^n q^i =
\begin{cases} 
  n + 1                    & \text{for}\: q = 1\text{;} \\[0.2cm]
  \frac{\;\ds 1 - q^{n+1}\;}{\ds 1 - q} & \text{otherwise.} 
\end{cases}
$
\\[0.2cm]
Note that \textsl{SymPy} has discovered that the case $q = 1$ needs to be treated differently from
the general case.

\textsl{SymPy} can evaluate infinite series.  For example, the expression
\\[0.2cm]
\hspace*{1.3cm}
\texttt{summation(1/i ** 2, (i, 1, oo))}
\\[0.2cm]
yields the result
\\[0.2cm]
\hspace*{1.3cm}
\texttt{pi**2/6}
\\[0.2cm]
showing that
\\[0.2cm]
\hspace*{1.3cm}
$\ds\sum\limits_{i=1}^\infty \frac{1}{i^2} = \frac{\pi}{6}$.


\section{Solving Equations}
In order to solve the quadratic equation
\\[0.2cm]
\hspace*{1.3cm}
$x^2 - x - 1 = 0$
\\[0.2cm]
we use the command
\\[0.2cm]
\hspace*{1.3cm}
\texttt{solve(x**2 - x - 1, x)}
\\[0.2cm]
Since there are two solutions, the result is the list
\\[0.2cm]
\hspace*{1.3cm}
\texttt{[1/2 + sqrt(5)/2, -sqrt(5)/2 + 1/2]}.
\\[0.2cm]
To solve the system of linear equations
\\[0.2cm]
\hspace*{1.3cm}
$
\begin{array}[t]{lcr}
  x + y & = &  1 \\
  x - y & = & -1
\end{array}
$
\\[0.2cm]
we can use the command
\\[0.2cm]
\hspace*{1.3cm}
\texttt{solve(\{x + y - 1, x - y + 1\}, \{x,y\})}
\\[0.2cm]
Note that we have to specify the equations as terms that are equal to $0$.  Therefore, the equation
$x + y = 1$ has to be converted into $x + y - 1 = 0$.  Also note that both the equations to solve
and the variables to solve for have to be 
specified as sets.

\section{Solving Recurrence Equations}
Suppose we want to solve the recurrence equation
\\[0.2cm]
\hspace*{1.3cm}
$a(n+2) = a(n+1) + a(n)$  \quad with the initial values $a(0) = 0$ and $a(1) = 1$.
\\[0.2cm]
To solve this recurrence equation, the following commands can be used:
\\[0.2cm]
\hspace*{1.3cm}
\texttt{a = Function('a')}\\
\hspace*{1.3cm}
\texttt{n = symbols('n')} \\
\hspace*{1.3cm}
\texttt{rsolve(a(n+2) - a(n+1) - a(n), a(n), \{ a(0):0, a(1):1 \})}
\\[0.2cm]
This function call solves the equation
\\[0.2cm]
\hspace*{1.3cm}
\texttt{a(n+2) - a(n+1) - a(n) = 0} 
\\[0.2cm]
for the function \texttt{a(n)} where the initial values are given as
\\[0.2cm]
\hspace*{1.3cm}
 \texttt{a(0) = 0} \quad and \quad \texttt{a(1) = 1}. 
\\[0.2cm]
The solution that is computed is
\\[0.2cm]
\hspace*{1.3cm}
\texttt{sqrt(5)*(1/2 + sqrt(5)/2)**n/5 - sqrt(5)*(-sqrt(5)/2 + 1/2)**n/5}.
\\[0.2cm]
When converted into \LaTeX, this prints as
\\[0.2cm]
\hspace*{1.3cm}
$\displaystyle a(n) = \frac{\sqrt{5}}{5} \left(\frac{1}{2} + \frac{\sqrt{5}}{2}\right)^{n} - \frac{\sqrt{5}}{5} \left(- \frac{\sqrt{5}}{2} + \frac{1}{2}\right)^{n}$.
\\[0.2cm]
Incidentally, $a(n)$ is the $n$-th 
\href{http://en.wikipedia.org/wiki/Fibonacci_number}{\emph{Fibonacci number}}.

\section{Solving Differential Equations}
In order to solve the differential equation
\\[0.2cm]
\hspace*{1.3cm}
$\displaystyle x \cdot \frac{d\hspace{0.1pt}f}{dx} - f(x) = x^2$
\\[0.2cm]
we first declare $f$ to be a function using the command 
\\[0.2cm]
\hspace*{1.3cm}
\texttt{f = Function("f")}
\\[0.2cm]
Then, issuing the command
\\[0.2cm]
\hspace*{1.3cm}
\texttt{dsolve(Eq(x * f(x).diff(x) - f(x), x), f(x))}
\\[0.2cm]
yields the solution
\\[0.2cm]
\hspace*{1.3cm}
\texttt{x*(C1 + x)}
\\[0.2cm]
which shows that for any $c_1 \in \mathbb{R}$ the function
\\[0.2cm]
\hspace*{1.3cm}
$f(x) = x \cdot (c_{1} + x)$
\\[0.2cm]
is a solution for the given differential equation.

\section{Substitution and Simplification}
In order to substitute a value for a variable, we can use the function \texttt{subs}.  For
example, after declaring \texttt{n} as a symbolic variable via
\\[0.2cm]
\hspace*{1.3cm}
\texttt{n = symbols("n")}
\\[0.2cm]
we can define
\\[0.2cm]
\hspace*{1.3cm}
\texttt{expr = n * (n+1) / 2}
\\[0.2cm]
and then use the function \texttt{subs} to substitute \texttt{n+1} for \texttt{n} as follows:
\\[0.2cm]
\hspace*{1.3cm}
\texttt{expr.subs(n, n+1)}.
\\[0.2cm]
This yields the result
\\[0.2cm]
\hspace*{1.3cm}
\texttt{(n + 1)*(n + 2)/2}.
\\[0.2cm]
We can simplify an expression using the function \texttt{simplify}.
Figure \ref{fig:induction.py} shows a short program that can be used to verify a formula like
\\[0.2cm]
\hspace*{1.3cm}
$\displaystyle  \sum\limits_{i=1}^n \frac{1}{i \cdot (i+1)} = \frac{n}{n+1}$ \hspace*{\fill} (1)
\\[0.2cm]
by induction.  We discuss this function line by line.

\begin{figure}[!ht]
\centering 
\begin{Verbatim}[ frame         = lines, 
                  framesep      = 0.3cm, 
                  firstnumber   = 1,
                  labelposition = bottomline,
                  numbers       = left,
                  numbersep     = -0.2cm,
                  xleftmargin   = -0.3cm,
                  xrightmargin  = -0.3cm,
                ]
    from sympy import * 
    
    n = symbols("n") 
    i = symbols("i") 
    
    def verifySum(s, e, i, n):
        """
        check by induction whether the folowing equation holds:
               sum(e(i), i=1..n) == s 
        """
        lhs = e.subs(i, 1) 
        rhs = s.subs(n, 1) 
        base_case = simplify(lhs - rhs) 
        lhs = s + e.subs(i, n+1) 
        rhs = s.subs(n, n + 1) 
        induction_step = simplify(lhs - rhs) 
        return base_case == 0 and induction_step == 0 
    
    def test(s, e, i, n):
        if verifySum(s, e, i, n):
            print "sum(" + str(e) + ", " + str(i) + "= 1.." + str(n) + ") = " + str(s) 
        else:
            print "unable to prove:"
            print "sum(" + str(e) + ", " + str(i) + "= 1.." + str(n) + ") == " + str(s) 
            
    s = n / (n + 1) 
    test(s, 1/(i*(i+1)), i, n) 
\end{Verbatim}
\vspace*{-0.3cm}
\caption{A function to prove a summation formula by induction.}
\label{fig:induction.py}
\end{figure}

\begin{enumerate}
\item The purpose of the function call
      \\[0.2cm]
      \hspace*{1.3cm}
      $\texttt{verifySum}(s, e, i, n)$
      \\[0.2cm]
      is to prove the formula
      \\[0.2cm]
      \hspace*{1.3cm}
      $\sum\limits_{i=1}^n e = s$  \hspace*{\fill} (2)
      \\[0.2cm]
      by mathematical induction.
      Here, $e$ denotes an expression containing the variable $i$, while $s$ is an expression containing
      the variable $n$.  For example, 
      \\[0.2cm]
      \hspace*{1.3cm}
      \texttt{verifySum(n / (n + 1), 1/(i*(i+1)), i, n)}
      \\[0.2cm]
      would try to verify the formula (1) by mathematical induction.
\item In order to verify the base case, we have to substitute the number 1 for the variable
      \texttt{i} in the expression \texttt{e} and we have to substitute the number 1 for the
      variable \texttt{n} in the expression \texttt{s}.  The resulting expressions are called
      \texttt{lhs} and \texttt{rhs} in line 11 and line 12 of Figure \ref{fig:induction.py}, respectively.
      In order to check that these expressin have the same value, we simplify their difference
      using the function call
      \\[0.2cm]
      \hspace*{1.3cm}
      \texttt{simplify(lhs - rhs)}.
      \\[0.2cm]
      If this function call returns 0, we can be sure that \texttt{lhs} and \texttt{rhs} have the
      same value.
\item For the induction step, we have to substitute \texttt{n+1} for the variable \texttt{i} in the
      expression \texttt{e} and add the resulting expression to the expression \texttt{s}.  If the
      resulting expression is the same as the expression we get when substituting \texttt{n+1} for
      \texttt{n} in \texttt{s}, then the formula (2) has been proven by mathematical induction.
\end{enumerate}

\section{Miscellaneous}
In the following subsections we discuss some minor issues and helper functions.

\subsection{Dealing with Rational Numbers}
In \textsl{Python}, an expression of the form 
\\[0.2cm]
\hspace*{1.3cm}
\texttt{1/3}
\\[0.2cm]
is not interpreted as the rational number $\frac{1}{3}$.  Instead, in \textsl{Python} 2, this
is truncated to $0$, while \textsl{Python} 3 converts this into the floating point number
$0.3333333333333333$.  Neither of these behaviors is acceptable when doing symbolic computations.
In order to avoid this pitfall, we should use the expression 
\\[0.2cm]
\hspace*{1.3cm}
\texttt{Rational(1,3)}
\\[0.2cm]
This creates the rational number $\ds\frac{1}{3}$ which is printed as \texttt{1/3}.

\subsection{Converting output to \LaTeX}
Sometimes it is useful to convert the output produced by a \textsl{SymPy} function into \LaTeX.
For example, the command
\\[0.2cm]
\hspace*{1.3cm}
\texttt{solve(x**3 + 2 * x + 1, x)}
\\[0.2cm]
produces the following output:
\begin{verbatim}
[2/(3*(-1/2 - sqrt(3)*I/2)*(1/2 + sqrt(177)/18)**(1/3)) - 
 (-1/2 - sqrt(3)*I/2)*(1/2 + sqrt(177)/18)**(1/3), 
 -(-1/2 + sqrt(3)*I/2)*(1/2 + sqrt(177)/18)**(1/3) + 
 2/(3*(-1/2 + sqrt(3)*I/2)*(1/2 + sqrt(177)/18)**(1/3)), 
 -(1/2 + sqrt(177)/18)**(1/3) + 
 2/(3*(1/2 + sqrt(177)/18)**(1/3))
]
\end{verbatim}
Since this is hard to read, we might want to convert this into \LaTeX\ using the command
\\[0.2cm]
\hspace*{1.3cm}
\texttt{latex(solve(x**3 + 2 * x + 1, x))}
\\[0.2cm]
This will produce \LaTeX\ output that can be typeset as
\\[0.2cm]
\hspace*{1.3cm}
$\begin{array}[t]{lcl}
 z_1 & = & \ds \frac{2}{3 \left(- \frac{1}{2} - \frac{\sqrt{3} i}{2}\right)
  \sqrt[3]{\frac{1}{2} + \frac{\sqrt{177}}{18}}} - \left(- \frac{1}{2} - \frac{\sqrt{3}i}{2}\right) \sqrt[3]{\frac{1}{2} + \frac{\sqrt{177}}{18}}, \\[0.8cm]
 z_2 & = & \ds - \left(- \frac{1}{2} +  \frac{\sqrt{3} i}{2}\right) \sqrt[3]{\frac{1}{2} + \frac{\sqrt{177}}{18}} + \frac{2}{3
  \left(- \frac{1}{2} + \frac{\sqrt{3} i}{2}\right) \sqrt[3]{\frac{1}{2} +
    \frac{\sqrt{177}}{18}}}, \\[0.8cm]
 z_3 & = & \ds - \sqrt[3]{\frac{1}{2} + \frac{\sqrt{177}}{18}} + \frac{2}{3
  \sqrt[3]{\frac{1}{2} + \frac{\sqrt{177}}{18}}},
\end{array}
$
\\[0.2cm]
where $z_1$, $z_2$, and $z_3$ are the three solutions to the third order equation
\\[0.2cm]
\hspace*{1.3cm}
$z^3 + 2 \cdot z + 1 = 0$.
\\[0.2cm]
In this case, I had to massage the output produced by the function \texttt{latex} manually, but I
only had to do minor edits to achieve the result displayed above.

\chapter{Introduction to \textsl{SciPy}}
In this short section we will show how \textsl{SciPy} can be used for
\begin{enumerate}
\item solving equations numerically, 
\item numerical integration, and
\item linear algebra.
\end{enumerate}

\section{Solving Equations}
Suppose we want to solve the equation
\\[0.2cm]
\hspace*{1.3cm}
$\cos(x) - x = 0$
\\[0.2cm]
numerically.  Since the function $f(x) := \cos(x) - x$ has a sign change in the intervall $[0,\pi/2]$ the 
\href{http://en.wikipedia.org/wiki/Intermediate_value_theorem}{\emph{intermediate value theorem}}
tells us that a solution to this equation exists.  In order to solve this equation, we can use the
following commands:
\\[0.2cm]
\hspace*{1.3cm}
\texttt{from scipy import *} \\
\hspace*{1.3cm}
\texttt{from scipy.optimize import root} \\
\hspace*{1.3cm}
\texttt{root(lambda x: x - cos(x), 0)}
\\[0.2cm]
This will produce the following output:
\begin{verbatim}
      status: 1
     success: True
         qtf: array([  2.66786593e-13])
        nfev: 9
           r: array([-1.67361202])
         fun: array([ 0.])
           x: array([ 0.73908513])
     message: 'The solution converged.'
        fjac: array([[-1.]])
\end{verbatim}
This shows that the equation has the solution $\mathtt{x} = 0.73908513$.

\section{Numerical Integration}
Suppose we need to evaluate the integral
\\[0.2cm]
\hspace*{1.3cm}
$\displaystyle \int\limits_{0}^{1} \mathtt{e}^{\sin(x)} dx$
\\[0.2cm]
Unfortunately, this integral can not be evaluated symbolically.  If we would try \textsl{SymPy}'s
\texttt{integrate} command, we would be left with the result
\\[0.2cm]
\hspace*{1.3cm}
\texttt{Integral(exp(sin(x)), (x, 0, 1))}.
\\[0.2cm]
This result indicates that \textsl{SymPy} is not able to evaluate this integral in closed terms.
In order to evaluate this integral at least numerically, we first import the modules needed via the
commmands
\\[0.2cm]
\hspace*{1.3cm}
\texttt{from scipy import *} \\
\hspace*{1.3cm}
\texttt{from scipy.integrate import quad}
\\[0.2cm]
Then, we can use the command
\\[0.2cm]
\hspace*{1.3cm}
\texttt{quad(lambda x: exp(sin(x)), 0, 1)}
\\[0.2cm]
The result of this command is the pair
\\[0.2cm]
\hspace*{1.3cm}
\texttt{(1.6318696084180513, 1.8117392124517587e-14)}.
\\[0.2cm]
The first component of this pair is the value of the integral, while the second value is the
precision.

\section{Linear Algebra}
Assume we have been given the matrix
\\[0.2cm]
\hspace*{1.3cm}
$A = \left(
\begin{array}[c]{lll}
  3 & 1 & 3 \\
  1 & 3 & 1 \\
  1 & 1 & 0
\end{array}
\right)
$
\\[0.2cm]
and we want to compute the inverse of $A$ as well as the eigenvalues and the corresponding
eigenvectors.  In order to do so, we first import the module \texttt{numpy} via the command
\\[0.2cm]
\hspace*{1.3cm}
\texttt{import numpy as np}
\\[0.2cm]
Then, we can define the matrix $A$ as follows:
\\[0.2cm]
\hspace*{1.3cm}
\texttt{A = np.mat("[3 1 3; 1 3 1; 3 1 0]")}
\\[0.2cm]
Next, we import the \texttt{linalg} module that contains the functions supporting linear algebra:
\\[0.2cm]
\hspace*{1.3cm}
\texttt{from scipy import linalg}
\\[0.2cm]
Then, the command
\\[0.2cm]
\hspace*{1.3cm}
\texttt{linalg.inv(A)}
\\[0.2cm]
yields the output
\begin{verbatim}
    array([[ 0.04166667, -0.125     ,  0.33333333],
           [-0.125     ,  0.375     ,  0.        ],
           [ 0.33333333,  0.        , -0.33333333]])
\end{verbatim}
This shows that
\\[0.2cm]
\hspace*{1.3cm}
$A^{-1} = \left(
\begin{array}[c]{rrr}
  \frac{1}{24} & -\frac{1}{8} & \frac{1}{3} \\[0.2cm]
 -\frac{1}{8}  &  \frac{3}{8} & 0           \\[0.2cm]
  \frac{1}{3}  & 0 & -\frac{1}{3}
\end{array}
\right).
$
\\[0.2cm]
Next let us compute the eigenvalues and eigenvectors of the matrix
\\[0.2cm]
\hspace*{1.3cm}
$A = \left(
\begin{array}[c]{lll}
  4 & 1 \\
  1 & 4
\end{array}
\right)
$.
\\[0.2cm]
We define $A$ via the following command:
\\[0.2cm]
\hspace*{1.3cm}
\texttt{A = np.mat("[4 1; 1 4]")}
\\[0.2cm]
To compute the eigenvalues and eigenvectors of $A$ we can use the command:
\\[0.2cm]
\hspace*{1.3cm}
\texttt{linalg.eig(A)}
\\[0.2cm]
This will produce the output
\begin{verbatim}
    (array([ 5.+0.j,  3.+0.j]),
     array([[ 0.70710678, -0.70710678],
           [ 0.70710678,  0.70710678]])).
\end{verbatim}
The first array contains the eigenvalues.  These are $5$ and $3$, respectively.  
Note that their imaginary component is zero.  In
\textsl{Python}, the complex number $z = a + b \cdot i$ is written as
\\[0.2cm]
\hspace*{1.3cm}
$a+b\mathtt{j}$
\\[0.2cm]
where $a$ is the real part while $b$ is the complex part.  
The second array contains the corresponding eigenvectors.  The length of the eigenvectors is $1$.
Effectively, the eigenvector corresponding to the eigenvalue 5 is
\\[0.2cm]
\hspace*{1.3cm}
$\displaystyle
  \frac{1}{\sqrt{2}} \cdot
  \left(
  \begin{array}[c]{l}
    1 \\ 1
  \end{array}
\right)
$,
\\[0.2cm]
while the eigenvector corresponding to the eigenvalue 3 is
\\[0.2cm]
\hspace*{1.3cm}

$\displaystyle
  \frac{1}{\sqrt{2}} \cdot
  \left(
  \begin{array}[c]{r}
    -1 \\ 1
  \end{array}
\right)
$.


\chapter{Plotting using \texttt{Matplotlib}}
\textsl{Python} has a module called \texttt{matplotlib} that is capable of producing high quality
plots of data.  In order to use it, we have to use the following import declarations:
\\[0.2cm]
\hspace*{1.3cm}
\texttt{import matplotlib.pyplot as plt} \\
\hspace*{1.3cm}
\texttt{import numpy as np}
\\[0.2cm]
The module \texttt{numpy} provides some basic numerical routines that come in handy when plotting
functions.  As an example, let us plot the sine function in the interval from $-2\cdot\pi$ to $2\cdot\pi$.  In
order to do this, we first create an array of $x$-values via the command
\\[0.2cm]
\hspace*{1.3cm}
\texttt{xs = np.linspace(-2*np.pi, 2*np.pi, 200)}
\\[0.2cm]
This command creates an array of a 200 values evenly spaced between $-2\cdot\pi$ and $2\cdot\pi$.  Next, we
need to compute the corresponding values of the sine function.  This can be done using the command
\\[0.2cm]
\hspace*{1.3cm}
\texttt{ys = [np.sin(x) for x in xs]}
\\[0.2cm]
Now we are ready to start some plotting.  Since we would like to have the plot appear immediately
after typing the \texttt{plot} command, we activate the \emph{interactive mode} via the command:
\\[0.2cm]
\hspace*{1.3cm}
\texttt{plt.ion()}
\\[0.2cm]
This enables us to build a plot incrementally. Then, we create a plot of the sine in the intervall
$[-2\cdot \pi, 2\cdot\pi]$ using the command:
\\[0.2cm]
\hspace*{1.3cm}
\texttt{plt.plot(xs, ys)}
\\[0.2cm]
This command causes a window to pop up that contains the plot shown in Figure
\ref{fig:naked-sine.eps}.  If we had not used the command \texttt{plt.ion()} to turn on interactive
mode we would have to issue the command
\\[0.2cm]
\hspace*{1.3cm}
\texttt{plt.show()}
\\[0.2cm]
to make the plot visible.

\begin{figure}[!ht]
  \centering
  \framebox{\epsfig{file=naked-sine.eps, scale=0.6}} 
  \caption{A plot of the naked sine function in the interval $[-2\cdot\pi, 2\cdot\pi]$.}
  \label{fig:naked-sine.eps}
\end{figure} 

We can add a plot of the cosine to the plot of the sine using the commands:
\\[0.2cm]
\hspace*{1.3cm}
\texttt{zs = np.cos(xs)}
\\[0.2cm]
\hspace*{1.3cm}
\texttt{plt.plot(xs, zs)}
\\[0.2cm]
The resulting plot is shown in Figure \ref{fig:naked-sine-and-cosine.eps}.

\begin{figure}[!ht]
  \centering
  \framebox{\epsfig{file=naked-sine-and-cosine.eps, scale=0.6}} 
  \caption{A plot of the sine and cosine functions in the interval $[-2\cdot\pi, 2\cdot\pi]$.}
  \label{fig:naked-sine-and-cosine.eps}
\end{figure}


We should add some decorations to the plot.  For example, the plot still needs a title and we
should put labels at the axis.  All this is done using the following commands:
\\[0.2cm]
\hspace*{1.3cm}
\texttt{plt.title(\symbol{34}The function sin(x)\symbol{34})} \\
\hspace*{1.3cm} 
\texttt{plt.xlabel(\symbol{34}x\symbol{34})} \\
\hspace*{1.3cm}
\texttt{plt.ylabel(\symbol{34}sin(x) vs. cos(x)\symbol{34})} 
\\[0.2cm]
Finally, the command
\\[0.2cm]
\hspace*{1.3cm}
\texttt{plt.savefig(\symbol{34}sine-and-cosine.eps\symbol{34})}
\\[0.2cm]
will save the plot as an encapsulated postscript file.  We can also store the plot in other formats
like, e.g.~\texttt{jpg} or \texttt{png}.   The resulting plot is shown in Figure \ref{fig:sine-and-cosine.eps}.  

\begin{figure}[!ht]
  \centering
  \framebox{\epsfig{file=sine-and-cosine.eps, scale=0.6}} 
  \caption{A plot of the sine function in the interval $[-\pi, \pi]$.}
  \label{fig:sine-and-cosine.eps}
\end{figure}


There are lots of options to tweak the figures produced with \texttt{matplotlib}.  Unfortunately, we
do not have the time to cover this topic in more depth.  Instead, I would encourage you to visit
\\[0.2cm]
\hspace*{1.3cm}
\href{http://matplotlib.org}{\texttt{http://matplotlib.org}}
\\[0.2cm]
for further information.

\chapter{Conclusion}
The combination of \textsl{Python}, \textsl{SymPy}, \textsl{Scipy}, and \textsl{Matplotlib} is a
very powerful and versatile toolbox.  Unfortunately, this short paper can only serve as an appetizer.
In order to get a deeper understanding of the above mentioned tools, the lectures of J.R.~Johansson, which are
available at
\\[0.2cm]
\hspace*{1.3cm}
\href{http://github.com/jrjohansson/scientific-python-lectures}{\texttt{http://github.com/jrjohansson/scientific-python-lectures}},
\\[0.2cm]
are a good starting point for further investigations.
\end{document}



