\chapter{Folgen und Reihen \label{chapter:folgen-und-reihen}}
Die Begriffe \emph{Folgen} und \emph{Reihen} sowie der Begriff des \emph{Grenzwerts} bilden die
Grundlage, auf der die Analysis aufgebaut ist.  Da Reihen nichts anderes sind als
spezielle Folgen, beginnen wir unsere Diskussion mit den Folgen.

\section{Folgen}
Anschaulich k\"onnen wir uns Folgen als unendlich lange Listen vorstellen.  Ein Beispiel f\"ur eine
solche Folge w\"are die Liste
\\[0.2cm]
\hspace*{1.3cm}
$\ds \Bigl[\frac{1}{1},\frac{1}{2},\frac{1}{3}, \cdots, \frac{1}{n}, \frac{1}{n+1}, \cdots\Bigr]$.
\\[0.2cm]
Diese Notation ist zwar suggestiv, aber f\"ur komplexere Beispiele nicht ad\"aquat.
Formal definieren wir Folgen daher als Funktionen, die auf den nat\"urlichen Zahlen definiert sind. 


\begin{Definition}[Folge]
  Eine Funktion $f\!:\! \mathbb{N} \rightarrow \mathbb{R}$ bezeichnen wir als eine  \emph{reellwertige Folge}. 
  Eine Funktion $f\!:\! \mathbb{N} \rightarrow \mathbb{C}$ bezeichnen wir als eine  \emph{komplexwertige Folge}. \eod
\end{Definition}

\noindent
Ist die Funktion $f$ ein Folge, so schreiben wir dies k\"urzer als 
$\bigl(f(n)\bigr)_{n\in\mathbb{N}}$ oder $\bigl(f_n\bigr)_{n\in\mathbb{N}}$ oder noch k\"urzer als $\bigl(f_n\bigr)_n$.


\examples
\begin{enumerate}
\item Die Funktion $a:\mathbb{N} \rightarrow \mathbb{R}$, die durch  
      $a(n) = \frac{1}{n}$ definiert ist, schreiben wir als die Folge
      $\Bigl(\frac{1}{n}\Bigr)_{n\in\mathbb{N}}$.
\item Die Funktion $a:\mathbb{N} \rightarrow \mathbb{R}$, die durch  
      $a(n) = (-1)^n$ definiert ist, schreiben wir als die Folge
      $\bigl((-1)^n\bigr)_{n\in\mathbb{N}}$.
\item Die Funktion $a:\mathbb{N} \rightarrow \mathbb{R}$, die durch  
      $a(n) = n$ definiert ist, schreiben wir als die Folge
      $\bigl(n\bigr)_{n\in\mathbb{N}}$. 
\end{enumerate}
Folgen k\"onnen auch induktiv definiert werden.  Um die Gleichung $x = \cos(x)$ zu l\"osen,
k\"onnen wir eine Folge $(x_n)_{n\in\mathbb{N}}$ induktiv wie folgt definieren:
\begin{enumerate}
\item Induktions-Anfang: $n = 1$.  Wir setzen
      \\[0.2cm]
      \hspace*{1.3cm}
      $x_1 := 1$.
\item Induktions-Schritt: $n \mapsto n+1$.  Nach Induktions-Voraussetzung ist $x_n$ bereits
      definiert.  Wir definieren $x_{n+1}$ als
      \\[0.2cm]
      \hspace*{1.3cm}
      $x_{n+1} := \cos(x_n)$. \eod
\end{enumerate}
Wir k\"onnen die ersten $39$ Glieder dieser Folge mit dem in Abbildung
\ref{fig:solve.stlx} gezeigten \href{http://randoom.org/Software/SetlX}{\textsc{SetlX}}-Programm
berechnen. Wir erhalten dann  
die in der  Tabelle \ref{tab:x-cos-x} auf Seite \pageref{tab:x-cos-x} 
gezeigten Ergebnisse.  Bei n\"aherer Betrachtung der Ergebnisse stellen wir fest,
dass die Folge $\folge{x_n}$ in einem gewissen Sinne gegen einen festen
\href{http://de.wikipedia.org/wiki/Grenzwert_(Folge)}{\emph{Grenzwert}} strebt.   
Diese Beobachtung wollen wir in der folgenden Definition
pr\"azisieren.  Vorab 
bezeichnen wir  die Menge der positiven reellen Zahlen mit $\mathbb{R}_+$, es gilt also
\[ \mathbb{R}_+ = \bigl\{ x \in \mathbb{R} \mid x > 0 \bigl\}. \]


\begin{figure}[!ht]
  \centering
\begin{Verbatim}[ frame         = lines, 
                  framesep      = 0.3cm, 
                  labelposition = bottomline,
                  numbers       = left,
                  numbersep     = -0.2cm,
                  xleftmargin   = 1.3cm,
                  xrightmargin  = 1.3cm,
                ]
    solve := procedure(k) {
        x    := []; 
        x[1] := 1.0;
        for (n in [1 .. k]) {
            x[n+1] := cos(x[n]);
            print("x_{$n$} = $x[n]$");
        }
    };
\end{Verbatim}
\vspace*{-0.3cm}
  \caption{Berechnung der durch  $x_1 = 1$ und $x_{n+1} = \cos(x_n)$ definierten Folge.}
  \label{fig:solve.stlx}
\end{figure} %\$

\begin{table}[!h]
  \centering
\framebox{
  \begin{tabular}{|l|c|l|c|l|c|l|c|}
\hline
   $n$ & $x_n$ & $n$ & $x_n$ & $n$ & $x_n$ & $n$ & $x_n$ \\
\hline
\hline
  &          & 10 & 0.731404 & 20 & 0.738938 & 30 & 0.739082 \\
\hline
1 & 1.000000 & 11 & 0.744237 & 21 & 0.739184 & 31 & 0.739087 \\
\hline
2 & 0.540302 & 12 & 0.735605 & 22 & 0.739018 & 32 & 0.739084 \\
\hline
3 & 0.857553 & 13 & 0.741425 & 23 & 0.739130 & 33 & 0.739086 \\
\hline
4 & 0.654290 & 14 & 0.737507 & 24 & 0.739055 & 34 & 0.739085 \\
\hline
5 & 0.793480 & 15 & 0.740147 & 25 & 0.739106 & 35 & 0.739086 \\
\hline
6 & 0.701369 & 16 & 0.738369 & 26 & 0.739071 & 36 & 0.739085 \\
\hline
7 & 0.763960 & 17 & 0.739567 & 27 & 0.739094 & 37 & 0.739085 \\
\hline
8 & 0.722102 & 18 & 0.738760 & 28 & 0.739079 & 38 & 0.739085 \\
\hline
9 & 0.750418 & 19 & 0.739304 & 29 & 0.739089 & 39 & 0.739085 \\
\hline
  \end{tabular}}
  \caption{Die ersten 39 Glieder der durch $x_1 = 1$ und $x_{n+1} = \cos(x_n)$ definierten Folge.}
  \label{tab:x-cos-x}
\end{table} 

\begin{Definition}[Grenzwert, \href{http://en.wikipedia.org/wiki/Bernard_Bolzano}{Bernard Bolzano} (1781--1848)]
\hspace*{\fill} \\
Eine Folge $\folge{a_n}$ \emph{konvergiert} gegen den \emph{Grenzwert} $g$, falls gilt:
\\[0.2cm]
\hspace*{1.3cm}
$\forall \varepsilon \in\mathbb{R}_+: \exists K \in \mathbb{R}: \forall n \in \mathbb{N}:\bigl( n \geq K \rightarrow |a_n - g| < \varepsilon\bigr)$. 
\\[0.2cm]
In diesem Fall schreiben wir
\\[0.2cm]
\hspace*{1.3cm}
$\lim\limits_{n\rightarrow\infty} a_n = g$.  \eod
\end{Definition}
Anschaulich besagt diese Definition, dass fast alle Glieder $a_n$ der Folge $\folge{a_n}$
einen beliebig kleinen Abstand zu dem Grenzwert $g$ haben.
F\"ur die oben induktiv definierte Folge $x_n$ k\"onnen wir den Nachweis der Konvergenz erst
in einem sp\"ateren Kapitel antreten.  Wir betrachten statt dessen ein einfacheres Beispiel
und beweisen, dass 
\\[0.2cm]
\hspace*{1.3cm}
$\ds\lim\limits_{n\rightarrow\infty} \frac{1}{n} = 0$
\\[0.2cm]
gilt. 
\pagebreak

\proof
F\"ur jedes  $\varepsilon > 0$ m\"ussen wir eine Zahl $K$ angeben, so
dass f\"ur alle nat\"urlichen Zahlen $n$, die gr\"o\ss{}er-gleich $K$ sind, die Absch\"atzung
\\[0.2cm]
\hspace*{1.3cm}
$\ds \left| \frac{1}{n} - 0 \right| < \varepsilon $
\\[0.2cm]
gilt.  Wir definieren $K := \frac{1}{\varepsilon} + 1$.  Damit ist $K$ wohldefiniert,
denn da $\varepsilon$ positiv ist, gilt sicher auch $\varepsilon \not= 0$.  Nun benutzen
wir die Voraussetzung $n \geq K$ f\"ur $K = \frac{1}{\varepsilon} + 1$:
\\[0.2cm]
\hspace*{1.3cm}
$
\begin{array}{cll}
            & \ds n \geq \frac{1}{\varepsilon} + 1      \\[0.4cm] 
\Rightarrow & \ds n > \frac{1}{\varepsilon} & \mid \cdot \;\varepsilon \\[0.2cm]
\Rightarrow & n \cdot \varepsilon > 1        & \ds\mid \cdot \;\frac{1}{n} \\[0.2cm]
\Rightarrow & \ds \varepsilon > \frac{1}{n} & 
\end{array}
$
\\[0.2cm]
Da andererseits  $0 < \frac{1}{n}$ gilt, haben wir insgesamt f\"ur alle $n > K$
\\[0.2cm]
\hspace*{1.3cm}
$
\begin{array}{cl}
            & \ds 0 < \frac{1}{n} < \varepsilon              \\[0.3cm]
\Rightarrow & \ds \left|\frac{1}{n}\right| < \varepsilon     \\[0.4cm]
\Rightarrow & \ds \left|\frac{1}{n}-0\right| < \varepsilon
\end{array}
$
\\[0.2cm]
gezeigt und damit ist der Beweis abgeschlossen. \qed

\exercise
\renewcommand{\labelenumi}{(\alph{enumi})}
\begin{enumerate}
\item Beweisen Sie unter R\"uckgriff auf die Definition des Grenzwert-Begriffs, dass
\\[0.2cm]
\hspace*{1.3cm}
$\ds\lim\limits_{n\rightarrow\infty} \frac{1}{2^n} = 0$ 
\\[0.2cm] gilt.
\item Beweisen Sie unter R\"uckgriff auf die Definition des Grenzwert-Begriffs, dass
\\[0.2cm]
\hspace*{1.3cm}
$\ds\lim\limits_{n\rightarrow\infty} \frac{1}{\sqrt{n}} = 0$ 
\\[0.2cm] 
gilt.  \eox
\end{enumerate} 
\renewcommand{\labelenumi}{\arabic{enumi}.}


\noindent
Wir formulieren und beweisen einige unmittelbare Folgerungen aus der obigen Definition des Grenzwerts.
\begin{Satz}[Eindeutigkeit des Grenzwerts]
Konvergiert die Folge $\folge{a_n}$ sowohl gegen den Grenzwert $g_1$ als auch gegen den
Grenzwert $g_2$, so gilt $g_1 = g_2$.
\end{Satz}
\textbf{Beweis}:  Wir f\"uhren den Beweis indirekt und nehmen an, dass $g_1 \not= g_2$ ist.
Dann definieren wir
\\[0.2cm]
\hspace*{1.3cm}
 $\ds\varepsilon = \frac{1}{2}\cdot|g_2 - g_1|$ 
\\[0.2cm]
und aus der Annahme $g_1 \not= g_2$ folgt
$\varepsilon > 0$.  Aus der Voraussetzung, dass $\folge{a_n}$ gegen $g_1$ konvergiert,
folgt, dass es ein $K_1$ gibt, so dass 
\\[0.2cm]
\hspace*{1.3cm}
$\forall n \in \mathbb{N}:\bigl( n \geq K_1 \rightarrow | a_n - g_1 | < \varepsilon\bigr)$
\\[0.2cm]
gilt.  Analog folgt  aus der Voraussetzung, dass $\folge{a_n}$ gegen $g_2$ konvergiert,
dass es ein $K_2$ gibt, so dass 
\\[0.2cm]
\hspace*{1.3cm}
$\forall n \in \mathbb{N}:\bigl( n \geq K_2 \rightarrow | a_n - g_2 | < \varepsilon\bigr)$
\\[0.2cm]
gilt.  Wir setzen $K := \max(K_1, K_2)$.  Alle $n\in\mathbb{N}$, die gr\"o\ss{}er-gleich $K$ sind,
sind dann sowohl gr\"o\ss{}er-gleich $K_1$ als auch gr\"o\ss{}er-gleich $K_2$. Unter Benutzung der 
\emph{Dreiecksungleichung}\footnote{
Sind $a, b\in \mathbb{R}$, so gilt $|a+b| \leq |a| + |b|$.}
erhalten wir f\"ur alle $n \geq K$ die folgende Kette von Ungleichungen:
\\[0.2cm]
\hspace*{1.3cm}
$
\begin{array}{lcll}  
   2 \cdot \varepsilon & = & |g_2 - g_1| \\
                   & = & |(g_2 - a_n) + (a_n - g_1)| \\
                   & \leq & |g_2 - a_n| + |a_n - g_1| 
                          & \mbox{(Dreiecksungleichung)} \\
                   & < & \varepsilon + \varepsilon \\
                   & = & 2 \cdot \varepsilon \\
\end{array}
$
\\[0.2cm]
Aus dieser Ungleichungs-Kette w\"urde aber $2\cdot \varepsilon < 2\cdot \varepsilon$ folgen und das
ist ein Widerspruch.  Somit ist die Annahme $g_1 \not= g_2$ falsch und es muss $g_1 = g_2$
gelten.  
\qed

\remark
Die Schreibweise $\lim\limits_{n\rightarrow\infty} a_n = g$
wird durch den letzten Satz im Nachhinein gerechtfertigt.

\exercise
Zeigen Sie, dass die Folge $\folge{(-1)^n}$ nicht konvergent ist. \eox

\solution
Wir f\"uhren den Beweis indirekt und nehmen an, dass die Folge
$\folge{(-1)^n}$ konvergiert.  Bezeichnen wir diesen Grenzwert mit $s$, so gilt also
\\[0.2cm]
\hspace*{1.3cm}
$\forall \varepsilon \in \mathbb{R}_+: \exists K \in \mathbb{R}: \forall n \in \mathbb{N}: \bigl( n \geq K \rightarrow \bigl|(-1)^n - s\bigr| < \varepsilon\bigr)$
\\[0.2cm]
Daher gibt es f\"ur $\varepsilon = 1$ eine Zahl $K$, so dass 
\\[0.2cm]
\hspace*{1.3cm}
$ \forall n \in \mathbb{N} : n \geq K \rightarrow \bigl|(-1)^n - s\bigr| < 1 $
\\[0.2cm]
gilt.  Da aus $n \geq K$ sicher auch $2 \cdot n \geq K$ und $2\cdot n+1 \geq K$ folgt, h\"atten wir dann
f\"ur $n \geq K$ die beiden folgenden Ungleichungen:
\\[0.2cm]
\hspace*{1.3cm}
$
  \bigl| (-1)^{2\cdot n} - s\bigr| < 1 \quad \mbox{und} \quad
  \bigl| (-1)^{2\cdot n+1} - s\bigr| < 1 
$
\\[0.2cm]
Wegen $(-1)^{2\cdot n} = 1$ und $(-1)^{2\cdot n + 1} = -1$ haben wir also 
\\[0.2cm]
\hspace*{1.3cm}
$\bigl|  1 - s\bigr| < 1$ \quad und \quad $\bigl| -1 - s\bigr| < 1$.
\\[0.2cm]
Wegen $-1 - s = (-1) \cdot (1 + s)$ und $|a \cdot b| = |a|\cdot|b|$ k\"onnen wir die letzte Ungleichung
\\[0.2cm]
\hspace*{1.3cm}
$\bigl| 1 + s\bigr| < 1$
\\[0.2cm]
noch zu  vereinfachen.
Aus den beiden Ungleichungen $|1-s|<1$ und $|1+s|<1$  erhalten wir nun einen Widerspruch:
\\[0.2cm]
\hspace*{1.3cm}
$
\begin{array}{lcll}
  2 & = & \bigl| 1 + 1 \bigr| \\[0.2cm]
    & = & \bigl| (1 - s) + (s + 1) \bigr| \\[0.2cm]
    & \leq & \bigl|1 - s\bigr| + \bigl|1+s \bigr| & \quad \mbox{(Dreiecksungleichung)} \\[0.2cm]
    & <    & 1 + 1                         \\[0.2cm]
    & =    & 2 & 
\end{array}
$
\\[0.2cm]
Fassen wir diese Ungleichungs-Kette zusammen, so haben die (offensichtlich falsche)
Ungleichung $2<2$ abgeleitet.
Damit haben wir aus der Annahme, dass die Folge gegen den Grenzwert $s$ konvergiert, einen
Widerspruch hergeleitet. \qed
\pagebreak

\begin{Definition}[Beschr\"ankte Folge]
Eine Folge $\folge{a_n}$ ist \emph{beschr\"ankt}, falls es eine \emph{Schranke} $S$ gibt, so dass 
\\
\hspace*{1.3cm}
$ \forall n \in \mathbb{N}: \bigl|a_n\bigr| \leq S$
\\
gilt.  \eod
\end{Definition}

\examples
\begin{enumerate}
\item Die Folge $\folge{(-1)^n}$ ist durch die Schranke $S=1$ beschr\"ankt, denn offenbar gilt
      \\[0.2cm]
      \hspace*{1.3cm}
      $ \bigl|(-1)^n\bigr| = 1 \leq 1$.
\item  Die Folge $\folge{n}$ ist nicht beschr\"ankt, denn sonst g\"abe es eine Zahl $S$, so dass
       f\"ur alle nat\"urlichen Zahlen $n$ die Ungleichung $n \leq S$ gilt.  Da es beliebig gro\ss{}e
       nat\"urliche Zahlen gibt, kann dies nicht sein. \eox
\end{enumerate}

\begin{Satz}[Beschr\"anktheit konvergenter Folgen]
Jede konvergente Folge  ist beschr\"ankt.
\end{Satz}
\textbf{Beweis}: Es sei  $\folge{a_n}$ eine konvergente Folge und es gelte
\\[0.2cm]
\hspace*{1.3cm}
$ \lim\limits_{n\rightarrow\infty} a_n = g. $
\\[0.2cm]
Dann gibt es f\"ur beliebige $\varepsilon > 0$ ein $K$, so dass gilt
\\[0.2cm]
\hspace*{1.3cm}
$\forall n \in \mathbb{N}: \bigl(n \geq K \rightarrow \bigl| a_n - g \bigr| < \varepsilon\bigr)$.
\\[0.2cm]
Wir k\"onnen also f\"ur $\varepsilon = 1$ ein $K$ finden, so dass
\\[0.2cm]
\hspace*{1.3cm}
$\forall n \in \mathbb{N}:\bigl(n \geq K \rightarrow \bigl| a_n - g \bigr| < 1\bigr)$
\\[0.2cm]
gilt.  Wir k\"onnen ohne Einschr\"ankung der Allgemeinheit davon ausgehen, dass
$K$ eine nat\"urliche Zahl ist, denn wenn $K$ keine nat\"urliche Zahl ist, k\"onnen wir $K$
einfach durch die erste nat\"urliche Zahl ersetzen, die gr\"o\ss{}er als $K$ ist.
Dann definieren wir 
\\[0.2cm]
\hspace*{1.3cm}
$ S := \max\bigl\{ |a_1|, \cdots, |a_K|, 1 + |g| \bigr\}.  $
\\[0.2cm]
Wir behaupten, dass $S$ eine Schranke f\"ur die Folge $\folge{a_n}$ ist, wir zeigen also,
dass f\"ur alle $n \in \mathbb{N}$ 
\\[0.2cm]
\hspace*{1.3cm}
$ |a_n| \leq S $
\\[0.2cm] 
gilt.
Um diese Ungleichung nachzuweisen, f\"uhren wir eine Fall-Unterscheidung durch:
\begin{enumerate}
\item Fall: $n \leq K$.  Dann gilt offenbar 
      \\[0.2cm]
      \hspace*{1.3cm}      
      $|a_n| \in \bigl\{ |a_1|, \cdots, |a_K|, 1 + |g| \bigr\}$.
      \\[0.2cm]
      und daraus folgt sofort
      \\[0.2cm]
      \hspace*{1.3cm}      
      $|a_n| \leq \max\bigl\{ |a_1|, \cdots, |a_K|, 1 + |g| \bigr\} = S$.
\item Fall: $n > K$.  Dann haben wir
      \\[0.2cm]
      \hspace*{1.3cm}
$
      \begin{array}[b]{lcll}        
         |a_n| & =    & | a_n - g + g | \\[0.2cm]
               & \leq & | a_n - g | + |g| 
                      & \mbox{(Dreiecksungleichung)} \\[0.3cm]
               & <    & 1 + |g|           & \mbox{wegen $n>K$} \\[0.2cm]
               & \leq & S. \\[0.2cm]
      \end{array} \hspace*{\fill} \Box
      $
\end{enumerate}

\example
Aus dem letzten Satz folgt sofort, dass die Folge $\folge{n}$ nicht
konvergiert, denn diese Folge ist noch nicht einmal beschr\"ankt.  \eox
\pagebreak

\begin{Satz}[Summe konvergenter Folgen]
Sind $\folge{a_n}$ und $\folge{b_n}$ zwei Folgen, so dass
\\[0.2cm]
\hspace*{1.3cm}
$ \lim\limits_{n\rightarrow\infty} a_n = a \quad \wedge \quad \lim\limits_{n\rightarrow\infty} b_n = b $
\\[0.2cm]
gilt, dann konvergiert die Folge $\folge{a_n + b_n}$ gegen den Grenzwert $a+b$, in
Zeichen:
\\[0.2cm]
\hspace*{1.3cm}
$ \lim\limits_{n\rightarrow\infty} \bigl(a_n + b_n\bigr) = 
   \Bigl(\lim\limits_{n\rightarrow\infty} a_n\Bigr) +\Bigl(\lim\limits_{n\rightarrow\infty} b_n\Bigr).
$
\end{Satz}

\proof
Es sei $\varepsilon > 0$ \underline{fest} vorgegeben.  Wir suchen ein $K$, so dass
\\[0.2cm]
\hspace*{1.3cm}
$\forall n \in \mathbb{N}:\bigl( n \geq K \rightarrow \bigl| \bigl(a_n + b_n\bigr) - (a + b)\bigr| < \varepsilon \bigr)$
\\[0.2cm]
gilt.  Nach Voraussetzung gibt es f\"ur \underline{beliebi}g\underline{e} $\varepsilon' > 0$ ein $K_1$ und ein $K_2$, so dass
\\[0.2cm]
\hspace*{1.3cm}
$\forall n \in \mathbb{N}:\bigl(n \geq K_1 \rightarrow \bigl| a_n - a \bigr| < \varepsilon'\bigr)$ 
\quad \mbox{und} \quad
$\forall n \in \mathbb{N}:\bigl( n \geq K_2 \rightarrow \bigl| b_n - b \bigr| < \varepsilon'\bigr)$
\\[0.2cm]
gilt.  Wir setzen nun $\varepsilon' := \frac{1}{2} \cdot \varepsilon$.  Dann gibt es also $K_1$ und
$K_2$, so dass
\\[0.2cm]
\hspace*{1.3cm}
$\ds\forall n \in \mathbb{N}:\bigl(n \geq K_1 \rightarrow \bigl| a_n - a \bigr| < \frac{1}{2}\cdot\varepsilon\bigr)$ 
   \quad \mbox{und} \quad
\\[0.2cm]
\hspace*{1.3cm}
$\ds\forall n \in \mathbb{N}:\bigl(n \geq K_2 \rightarrow \bigl| b_n - b \bigr| < \frac{1}{2}\cdot\varepsilon\bigr)$
\\[0.2cm]
gilt.  Wir definieren $K := \max(K_1,K_2)$.  Damit gilt dann f\"ur alle $n \geq K$:
\\[0.2cm]
\hspace*{1.3cm}
$
\begin{array}{lcll}
  \bigl| \bigl(a_n + b_n\bigr) - (a + b) \bigr|  
 & = & \bigl| \bigl(a_n + b_n\bigr) - (a + b) \bigr| \\[0.3cm]
 & = & \bigl| \bigl(a_n - a\bigr) +  \bigl(b_n - b\bigr) \bigr| \\[0.3cm]
 & \leq & \bigl| \bigl(a_n - a\bigr) \bigr| +  \bigl|\bigl(b_n - b\bigr) \bigr| 
        & \mbox{(Dreiecksungleichung)} \\[0.3cm]
 & < & \ds\frac{1}{2}\cdot \varepsilon + \frac{1}{2} \cdot \varepsilon \\[0.3cm]
 & = &  \varepsilon. 
\end{array}
$
\\[0.2cm]
Damit ist die Behauptung gezeigt. \qed

\exercise
Zeigen Sie:
Sind $\folge{a_n}$ und $\folge{b_n}$ zwei Folgen, so dass
\\[0.2cm]
\hspace*{1.3cm}
$ \lim\limits_{n\rightarrow\infty} a_n = a \quad \wedge \quad \lim\limits_{n\rightarrow\infty} b_n = b $
\\[0.2cm]
gilt, dann konvergiert die Folge $\folge{a_n - b_n}$ gegen den Grenzwert $a-b$, in
Zeichen:
\\[0.2cm]
\hspace*{1.3cm}
$ \lim\limits_{n\rightarrow\infty} \bigl(a_n - b_n\bigr) = 
   \Bigl(\lim\limits_{n\rightarrow\infty} a_n\Bigr) -\Bigl(\lim\limits_{n\rightarrow\infty} b_n\Bigr).
$
\eox  

\begin{Satz}[Produkt konvergenter Folgen]
Sind $\folge{a_n}$ und $\folge{b_n}$ zwei Folgen, so dass
\\[0.2cm]
\hspace*{1.3cm}
$ \lim\limits_{n\rightarrow\infty} a_n = a \quad \wedge \quad \lim\limits_{n\rightarrow\infty} b_n = b $
\\[0.2cm]
gilt, dann konvergiert die Folge $\folge{a_n \cdot b_n}$ gegen den Grenzwert $a\cdot b$, in
Zeichen:
\\[0.2cm]
\hspace*{1.3cm}
$ \lim\limits_{n\rightarrow\infty} \bigl(a_n \cdot b_n\bigr) = 
   \Bigl(\lim\limits_{n\rightarrow\infty} a_n\Bigr) \cdot\Bigl(\lim\limits_{n\rightarrow\infty} b_n\Bigr).
$
\end{Satz}
\textbf{Beweis}:  Es sei $\varepsilon > 0$ \underline{fest} vorgegeben.  Wir suchen ein $K$, so dass
\\[0.2cm]
\hspace*{1.3cm}
$\forall n \in \mathbb{N}:\bigl(n \geq K \rightarrow \bigl|a_n \cdot b_n - a \cdot b\bigr| < \varepsilon\bigr)$
\\[0.2cm]
gilt.  Da die Folge $\folge{a_n}$ konvergent ist, ist diese Folge auch beschr\"ankt, es gibt
also eine Zahl $S$, so dass  
\\[0.2cm]
\hspace*{1.3cm}
$ |a_n| \leq S \quad \mbox{f\"ur alle $n\in\mathbb{N}$}$
\\[0.2cm]
gilt.
Nach Voraussetzung gibt es f\"ur beliebige $\varepsilon_1 > 0$ ein
$K_1$ und f\"ur beliebige $\varepsilon_2 >0$ ein $K_2$, so dass
\\[0.2cm]
\hspace*{1.3cm}
$\forall n \in \mathbb{N}:\bigl(n \geq K_1 \rightarrow \bigl| a_n - a \bigr| < \varepsilon_1\bigr)$
   \quad \mbox{und} \quad
$\forall n \in \mathbb{N}:\bigl(n \geq K_2 \rightarrow \bigl| b_n - b \bigr| < \varepsilon_2\bigr)$
\\[0.2cm]
gilt.  Wir setzen nun $\varepsilon_1 := \frac{\varepsilon}{2\cdot(|b| + 1)}$ und
$\varepsilon_2 := \frac{\varepsilon}{2\cdot S}$.  Dann gibt es also $K_1$ und
$K_2$, so dass
\\[0.2cm]
\hspace*{1.3cm}
$\ds\forall n \in \mathbb{N}:\Bigl(n \geq K_1 \rightarrow \bigl| a_n - a \bigr| < \frac{\varepsilon}{2\cdot(|b| + 1)}\Bigr)$
\quad \mbox{und} \quad
\\[0.2cm]
\hspace*{1.3cm}
$\ds\forall n \in \mathbb{N}:\Bigl(n \geq K_2 \rightarrow \bigl| b_n - b \bigr| < \frac{\varepsilon}{2\cdot S}\Bigr)$
\\[0.2cm]
gilt.  Wir definieren $K := \max(K_1,K_2)$.  Damit gilt dann f\"ur alle $n \geq K$:
\\[0.2cm]
\hspace*{1.3cm}
$
\begin{array}{lcll}
  \bigl| a_n \cdot  b_n - a \cdot  b \bigr|  
 & =    & \bigl| \bigl(a_n \cdot  b_n - a_n\cdot b \bigr) + \bigl(a_n\cdot b - a\cdot b\bigr) \bigr| \\[0.3cm]
 & \leq & \bigl| \bigl(a_n \cdot  b_n - a_n\cdot b \bigr) \bigr| + \bigl| \bigl(a_n\cdot b - a\cdot b\bigr) \bigr| 
        & \mbox{(Dreiecksungleichung)}  \\[0.3cm]
 & =    & \bigl|a_n\bigr| \cdot  \bigl| b_n - b \bigr| + \bigl| a_n - a \bigr| \cdot  |b| \\[0.3cm]
 & \leq & S \cdot  \bigl| b_n - b \bigr| + \bigl| a_n - a \bigr| \cdot  (|b| + 1) \\[0.3cm]
 & <    & \ds S \cdot  \frac{\varepsilon}{2\cdot S} + \frac{\varepsilon}{2\cdot (|b| + 1)} \cdot  (|b| + 1)  \\[0.3cm]
 & \leq & \ds\frac{\;\varepsilon\;}{2} + \frac{\;\varepsilon\;}{2}  \\[0.3cm]
 & =    & \varepsilon. 
\end{array}
$
\\[0.2cm]
Damit ist die Behauptung gezeigt. \qed


\exercise
Zeigen Sie:
{\em
Sind $\folge{a_n}$ und $\folge{b_n}$ zwei Folgen, so dass
\\[0.2cm]
\hspace*{1.3cm}
$ \lim\limits_{n\rightarrow\infty} a_n = a \quad \wedge \quad \lim\limits_{n\rightarrow\infty} b_n = b $
\\[0.2cm]
gilt und gilt $b_n \not= 0$ f\"ur alle $n \in \mathbb{N}$, sowie $b \not= 0$,
so konvergiert die Folge $\folge{a_n/b_n}$ gegen den Grenzwert $a/b$, in
Zeichen:}
\\[0.2cm]
\hspace*{1.3cm}
$\ds \lim\limits_{n\rightarrow\infty} \frac{a_n}{b_n} = 
   \frac{\Bigl(\lim\limits_{n\rightarrow\infty}
     a_n\Bigr)}{\Bigl(\lim\limits_{n\rightarrow\infty} b_n\Bigr)} =
   \frac{a}{b}.
$
\eox

\solution  
Zun\"achst k\"onnen wir das Problem vereinfachen, wenn wir die Folge
$\folge{a_n/b_n}$ als Folge von Produkten schreiben:
\\[0.2cm]
\hspace*{1.3cm}
$\ds \Folge{\frac{a_n}{b_n}} = \folge{a_n} \cdot  \Folge{\frac{1}{b_n}} $
\\[0.2cm]
Falls wir zeigen k\"onnen, dass 
\\[0.2cm]
\hspace*{1.3cm}
$\ds \lim\limits_{n\rightarrow\infty} \frac{1}{b_n} = \frac{1}{b} $
\\[0.2cm]
gilt, dann folgt die Behauptung aus dem Satz \"uber das Produkt konvergenter Folgen.
Bei unserer Suche nach einem Beweis starten wir damit, dass wir die Behauptung noch einmal
hinschreiben:
\begin{equation}
  \label{eq:ab0}  
\forall \varepsilon \in \mathbb{R}_+: \exists K \in \mathbb{R}:\forall n \in \mathbb{N}: \Bigl(
   n \geq K \rightarrow \left| \frac{1}{b_n} - \frac{1}{b} \right| < \varepsilon\Bigr)
\end{equation}
Wir m\"ussen also f\"ur alle $\varepsilon>0$ ein $K$ finden, so dass f\"ur alle nat\"urlichen
Zahlen  $n \geq K$ die Ungleichung
\begin{equation}
  \label{eq:ab1}
  \left| \frac{1}{b_n} - \frac{1}{b} \right| < \varepsilon 
\end{equation}
gilt.  Irgendwie m\"ussen wir die Voraussetzung, dass die Folge $\folge{b_n}$ gegen
$b$ konvergiert, ausnutzen.  Diese Voraussetzung lautet ausgeschrieben
\begin{equation}
  \label{eq:ab2}
  \forall \varepsilon' \in \mathbb{R}_+: \exists K' \in \mathbb{R}:\forall n \in \mathbb{N}: \Bigl(
   n \geq K' \rightarrow \bigl| b_n - b \bigr| < \varepsilon' \Bigr)
\end{equation}
Wir zeigen zun\"achst eine Absch\"atzung f\"ur die Betr\"age $|b_n|$, die wir sp\"ater brauchen.
Hier hilft uns die Voraussetzung, dass $b \not= 0$ ist.  Setzen wir in Ungleichung 
(\ref{eq:ab2}) f\"ur $\varepsilon'$ den Wert $\frac{1}{2} \cdot |b|$ ein, so erhalten wir eine Zahl
$K_1$, so dass f\"ur alle nat\"urlichen Zahlen $n \geq K_1$
\\[0.2cm]
\hspace*{1.3cm}
$\ds \bigl| b_n - b \bigr| < \frac{1}{2}\cdot |b| $
\\[0.2cm]
gilt.  Damit folgt:
\\[0.2cm]
\hspace*{1.3cm}
$\begin{array}{lrcl}
                & \bigl| b \bigr| & =    & \bigl| b - b_n + b_n \bigr|                   \\[0.2cm]
    \Rightarrow & \bigl| b \bigr| & \leq & \bigl| b - b_n \bigr| + \bigl| b_n \bigr|     \\[0.2cm]
    \Rightarrow & \bigl| b \bigr| & <    & \ds\frac{1}{2} \cdot \bigl|b\bigr| + \bigl| b_n \bigr| \\[0.2cm]
    \Rightarrow & \ds\frac{1}{2} \cdot \bigl|b\bigr| & < & \bigl| b_n \bigr| \\[0.3cm]
    \Rightarrow & \ds\frac{2}{\bigl|b\bigr|} & > & \ds\frac{1}{\bigl| b_n \bigr|}
  \end{array}
$
\\[0.2cm]
Damit wissen wir also, dass f\"ur alle $n>K_1$ die Ungleichung
\\[0.2cm]
\hspace*{1.3cm}
$\ds \frac{1}{|b_n|} < \frac{2}{|b|}$
\\[0.2cm]
gilt.  Um nun f\"ur ein gegebenes $\varepsilon > 0$ die Ungleichung (\ref{eq:ab1}) zu zeigen,  setzen wir
in der Voraussetzung (\ref{eq:ab2})   $\varepsilon' := \frac{1}{2}\cdot |b|^2 \cdot  \varepsilon$ 
und erhalten ein $K_2$, so dass f\"ur alle $n>K_2$ die Ungleichung
\begin{equation}
  \label{eq:ab3}
  \bigl|b - b_n\bigr| < \frac{1}{2} \cdot  |b|^2 \cdot  \varepsilon
\end{equation}
gilt.  Setzen wir $K := \max(K_1,K_2)$, so erhalten wir f\"ur alle $n>K$  die folgende
Ungleichungs-Kette: 
\\[0.2cm]
\hspace*{1.3cm}
$
\begin{array}{lcll}
       \ds\left| \frac{1}{b_n} - \frac{1}{b} \right| 
 & = & \ds\left| \frac{b-b_n}{b\cdot b_n} \right| \\[0.5cm]
 & = & \ds\frac{1}{|b|\cdot |b_n|} \cdot  \left| b-b_n\right| \\[0.5cm]
 & < & \ds\frac{2}{|b|\cdot |b|} \cdot  \left| b-b_n\right| 
     & \mbox{wegen $\ds\frac{2}{\bigl|b\bigr|} > \frac{1}{\bigl| b_n \bigr|}$} 
       \\[0.5cm]
 & < & \ds\frac{2}{|b|\cdot |b|} \cdot  \frac{1}{2}\cdot |b|^2 \cdot  \varepsilon 
     & \mbox{wegen (\ref{eq:ab3})} \\[0.5cm]
 & = & \varepsilon \\[0.3cm]
\end{array}
$
\\[0.2cm]
Damit haben wir f\"ur $n \geq K$ die Ungleichung
$\ds\left| \frac{1}{b_n} - \frac{1}{b} \right| < \varepsilon$
hergeleitet und der Beweis ist abgeschlossen. \qed

\remark
Die bis hierhin bewiesenen S\"atze zeigen, dass bei konvergenten Folgen die Berechnung von Grenzwerten mit den
arithmetischen Operationen vertauscht werden kann.  Damit erm\"oglichen uns diese S\"atze die Berechnung
von Grenzwerten.  Wir geben ein Beispiel:
\\[0.2cm]
\hspace*{1.3cm}
$
\begin{array}[b]{lcl}  
      \ds \lim\limits_{n\rightarrow\infty} \frac{n}{n+1}                                 
& = & \ds \lim\limits_{n\rightarrow\infty} \frac{1}{1+\frac{1}{n}}                                 \\[0.8cm]
& = & \ds \frac{\lim\limits_{n\rightarrow\infty} 1}{\lim\limits_{n\rightarrow\infty}1 +\frac{1}{n}} \\[0.8cm]
& = & \ds \frac{1}{\lim\limits_{n\rightarrow\infty}1 + \lim\limits_{n\rightarrow\infty} \frac{1}{n}} \\[0.8cm]
& = & \ds \frac{1}{1 + 0}                                                                             \\[0.3cm]
& = &  1
\end{array}
$
\eox



\begin{Satz}
Sind $\folge{a_n}$  
und $\folge{b_n}$ zwei konvergente Folgen, so dass
\\[0.2cm]
\hspace*{1.3cm}
$ \forall n \in \mathbb{N}: a_n \leq b_n $
\\[0.2cm]
gilt, dann gilt auch
\\[0.2cm]
\hspace*{1.3cm}
$ \lim\limits_{n\rightarrow\infty} a_n \leq \lim\limits_{n\rightarrow\infty} b_n. $
\end{Satz}

\exercise
Beweisen Sie den letzten Satz.  \eox


\begin{Definition}[monoton]
Eine Folge $\folge{a_n}$ ist \emph{monoton steigend} falls 
\\[0.2cm]
\hspace*{1.3cm}
$ \forall n \in \mathbb{N}: a_n \leq a_{n+1} $
\\[0.2cm]
gilt.  Analog hei\ss{}t eine Folge \emph{monoton fallend} falls
\\[0.2cm]
\hspace*{1.3cm}
$ \forall n \in \mathbb{N}: a_n \geq a_{n+1}. $
\eod
\end{Definition}
Ein Beispiel f\"ur eine monoton fallende Folge ist die Folge $\Folge{\frac{1}{n}}$, denn es
gilt
\\[0.2cm]
\hspace*{1.3cm}
$
\begin{array}{crcll}
            &  n+1          & \geq & n             & \ds\mid \cdot  \frac{1}{n} \\[0.3cm]
\Rightarrow & \ds\frac{n+1}{n} & \geq & 1             & \ds\mid \cdot  \frac{1}{n+1}\\[0.3cm]
\Rightarrow & \ds\frac{1}{n}   & \geq & \ds\frac{1}{n+1} \\[0.3cm]
\end{array}
$


\begin{Satz} \label{satz:monoton}
Ist die Folge $\folge{a_n}$ monoton fallend und beschr\"ankt, so ist die Folge auch konvergent.
\end{Satz}
\textbf{Beweis}:  Wir definieren zun\"achst die Menge $M$ als die Menge aller unteren Schranken der Folge $\folge{a_n}$
\\[0.2cm]
\hspace*{1.3cm}      
$M := \bigl\{ x \in \mathbb{Q} \mid \forall n \in \mathbb{N}: x \leq a_n \bigr\}$.
\\[0.2cm]
Weil wir vorausgesetzt haben, dass die Folge $\folge{a_n}$ beschr\"ankt ist,
ist die Menge $M$ sicher nicht leer und es ist offensichtlich, dass die Menge $M$ nach unten abgeschlossen ist.  
Au\ss{}erdem ist die Menge $M$ nach oben beschr\"ankt, eine obere Schranke ist das Folgenglied
$a_1$.  Da die Struktur $\langle \mathbb{R}, 0, 1, +, \cdot, \leq\rangle$ ein vollst\"andig geordneter
K�rper ist, hat die Menge $M$ ein Supremum und wir k\"onnen 
\\[0.2cm]
\hspace*{1.3cm}
$s := \sup(M)$
\\[0.2cm]
definieren.  Wir zeigen, dass
\\[0.2cm]
\hspace*{1.3cm}
$\lim\limits_{n\rightarrow\infty} a_n = s$
\\[0.2cm]
gilt.  Dazu m�ssen wir 
\\[0.2cm]
\hspace*{1.3cm}
$\forall \varepsilon \in \mathbb{R}_+: \exists K \in \mathbb{R}: \forall n \in \mathbb{N}: \bigl(n\geq k \rightarrow |a_n - s| < \varepsilon \bigr)$
\\[0.2cm]
zeigen.  Sei also $\varepsilon > 0$ gegeben.  Da 
\\[0.2cm]
\hspace*{1.3cm}
$s + \varepsilon > s$
\\[0.2cm]
ist und $s$ als das Supremum der Menge $M$ definiert ist, k\"onnen wir folgern, dass 
$s + \varepsilon \not\in M$ ist.  Nach Definition der Menge $M$ als Menge der unteren Schranken der
Folge $\folge{a_n}$ ist $s + \varepsilon$ dann keine untere Schranke der Folge $\folge{a_n}$.  Also
 gibt es eine Zahl $\widehat{n} \in \mathbb{N}$, so dass 
\\[0.2cm]
\hspace*{1.3cm}
$a_{\widehat{n}} < s + \varepsilon$ 
\\[0.2cm]
ist.  Da die Folge monoton fallend ist, gilt dann auch
\\[0.2cm]
\hspace*{1.3cm}
$a_n < s + \varepsilon$ \quad f\"ur alle $n \geq \widehat{n}$.
\\[0.2cm]
Andererseits ist $s - \frac{1}{2} \cdot \varepsilon < s$, so dass $s - \frac{1}{2} \cdot \varepsilon$ sicher ein Element der Menge $M$
ist, denn $M$ ist nach unten abgeschlossen.  Damit ist $s - \frac{1}{2} \cdot \varepsilon$ dann auch eine untere
Schranke der Folge $(a_n)_{n\in\mathbb{N}}$.  Folglich gilt f\"ur alle $n \in \mathbb{N}$
\\[0.2cm]
\hspace*{1.3cm}
$s - \varepsilon < s - \frac{1}{2} \cdot \varepsilon \leq a_n$.
\\[0.2cm]
Damit haben wir insgesamt
\\[0.2cm]
\hspace*{1.3cm}
$s - \varepsilon < a_n < s + \varepsilon$ \quad f\"ur alle $n \geq \widehat{n}$
\\[0.2cm]
und dies k\"onnen wir auch als
\\[0.2cm]
\hspace*{1.3cm}
$|a_n - s| < \varepsilon$ \quad f\"ur alle $n \geq \widehat{n}$
\\[0.2cm]
schreiben.  Setzen wir in der Definition des Grenzwerts $K := \widehat{n}$, so haben wir damit die Behauptung gezeigt.
\qed

\remark
Der soeben bewiesene Satz ist der erste Satz, bei dem wir ausgenutzt haben, dass die reellen Zahlen
vollst�ndig sind.
\eox

\exercise
Die Folge $\folge{a_n}$ sei monoton steigend und beschr\"ankt.  Zeigen Sie, dass der Grenzwert
\\[0.2cm]
\hspace*{1.3cm}
$\lim\limits_{n\rightarrow\infty} a_n$
\\[0.2cm]
existiert.  \eox


\begin{Definition}[Cauchy-Folge] \lb
Eine Folge $\folge{a_n}$ hei\ss{}t \emph{Cauchy-Folge} 
(\href{http://de.wikipedia.org/wiki/Augustin-Louis_Cauchy}{\textrm{Augustin-Louis Cauchy}}, 1789-1857), 
falls gilt:
\\[0.2cm]
\hspace*{1.3cm}
$\forall \varepsilon \in \mathbb{R}_+: \exists K \in \mathbb{R}: \forall m,n \in \mathbb{N}: \bigl(
  m \geq K \wedge n \geq K \rightarrow \bigl| a_m - a_n \bigr| < \varepsilon\bigr)
$. \eod
\end{Definition}

In einer Cauchy-Folge $\folge{a_n}$ liegen also die einzelnen Folgenglieder $a_n$ mit wachsendem $n$
immer dichter zusammen.  Wir werden sehen, dass eine Folge genau dann konvergent ist, wenn die Folge
eine Cauchy-Folge ist.  Den Nachweis dieser Behauptung spalten wir in mehrere S\"atze auf.


\begin{Satz}
Jede konvergente Folge $\folge{a_n}$ ist eine Cauchy-Folge.  
\end{Satz}


\noindent
\textbf{Beweis}:  Es sei $a := \lim\limits_{n\rightarrow\infty} a_n$. 
Sei $\varepsilon > 0$ gegeben.  Wir definieren $\varepsilon' := \frac{1}{2} \cdot \varepsilon$.
Aufgrund der Konvergenz der Folge $\folge{a_n}$ gibt es
dann ein $K$, so dass 
\\[0.2cm]
\hspace*{1.3cm}
$\ds\forall n \in \mathbb{N}:\Bigl( n \geq K \rightarrow \bigl|a_n-a\bigr| < \varepsilon' = \frac{\varepsilon}{2}\Bigr)$
\\[0.2cm]
gilt.  Damit gilt f\"ur alle $m,n\in\mathbb{N}$ mit $m \geq K$ und $n \geq K$ die folgende Absch\"atzung:
\\[0.2cm]
\hspace*{1.3cm}
$
   \begin{array}{lcl}
     |a_m - a_n| &   =  & \bigl|(a_m - a) + (a - a_n)\bigr| \\[0.2cm]
                 & \leq & \bigl|a_m - a\bigr| \;+\; \bigr|a - a_n\bigr| \\[0.2cm]
                 &   <  & \ds\frac{\varepsilon}{2} + \frac{\varepsilon}{2} \\[0.2cm]
                 &   =  & \varepsilon     
   \end{array}
$
\\[0.2cm]
Damit ist gezeigt, dass $\folge{a_n}$ eine Cauchy-Folge ist. \qed

\begin{Satz}
  Jede Cauchy-Folge ist beschr\"ankt.
\end{Satz}

\noindent
\textbf{Beweis}:  Wenn $\folge{a_n}$ eine Cauchy-Folge ist, dann finden wir eine Zahl
$K$, so dass f\"ur alle nat\"urlichen Zahlen $m,n$, die gr\"o\ss{}er-gleich $K$ sind, die Ungleichung
\\[0.2cm]
\hspace*{1.3cm}
$ \bigl| a_n - a_m | < 1 $
\\[0.2cm]
gilt.  Sei nun $h$ eine nat\"urliche Zahl, die gr\"o\ss{}er als $K$ ist.  Wir definieren
\\[0.2cm]
\hspace*{1.3cm}
$ S := \max\bigl\{ |a_1|, |a_2|, \cdots,  |a_h|, 1+ |a_h|\bigr\} $
\\[0.2cm]
und zeigen, dass $S$ eine  Schranke der Cauchy-Folge $\folge{a_n}$ ist, wir zeigen also
\\[0.2cm]
\hspace*{1.3cm}
$ \forall n \in \mathbb{N} : |a_n| \leq S. $
\\[0.2cm]
Falls $n \leq h$ ist, ist diese Ungleichung evident.  F\"ur alle $n>h$ haben wir die folgende
Absch\"atzung: 
\\[0.2cm]
\hspace*{1.3cm}
$ 
\begin{array}{lcl}
  \bigl| a_n \bigr| & =    & \bigl| a_n - a_h + a_h \bigr| \\[0.2cm]
                    & \leq & \bigl| a_n - a_h \bigr| + \bigl| a_h \bigr| \\[0.2cm]
                    & <    & 1 + \bigl| a_h \bigr| \\[0.2cm]
                    & \leq & S. \\[0.2cm]
\end{array}
$
\\[0.2cm]
Damit ist der Beweis abgeschlossen. \qed 

\exercise
In dem gleich folgenden Beweis der Tatsache, dass jede Cauchy-Folge konvergent ist, werden wir zwei
Eigenschaften des Supremums einer Menge $M$ benutzen, die zwar offensichtlich sind, die wir aber auch
formal beweisen sollten.  Nehmen Sie an, dass Folgendes gilt:
\begin{enumerate}
\item $M \subseteq \mathbb{R}$ ist nach unten abgeschlossen, es gilt also
      \\[0.2cm]
      \hspace*{1.3cm}
      $y < x \wedge x \in M \rightarrow y \in M$,
\item $s = \sup(M)$ \quad und
\item $\varepsilon > 0$.
\end{enumerate}
Beweisen Sie, dass dann
\\[0.2cm]
\hspace*{1.3cm}
$s - \varepsilon \in M$ \quad und \quad $s + \varepsilon \not\in M$
\\[0.2cm] 
gilt. \eox

\begin{Theorem}
Jede Cauchy-Folge ist konvergent.  
\end{Theorem}
\textbf{Beweis}: Der Beweis verl\"auft \"ahnlich wie der Nachweis, dass eine monotone und
beschr\"ankte Folge konvergent ist und zerf\"allt in zwei Teile: 
\begin{enumerate}
\item Zun\"achst definieren wir eine Menge $M$, die nicht leer und nach oben beschr\"ankt ist
      und definieren $s$ als das Supremum dieser Menge.
\item Anschlie\ss{}end zeigen wir, dass die Folge $\folge{a_n}$ gegen $s$ konvergiert.
\end{enumerate}
Wir definieren die Menge $M$ wie folgt:
\\[0.2cm]
\hspace*{1.3cm}
$M := \bigl\{ x \in \mathbb{R} \mid \exists K \in \mathbb{N}: \forall n \in \mathbb{N}:\bigl( n \geq K \rightarrow x \leq a_n\bigr) \bigr\}$.
\\[0.2cm]
Anschaulich ist $M$ die Menge aller unteren Grenzen f\"ur die Mehrheit der
Folgenglieder:  Ist $x \in M$, so m\"ussen von einem bestimmten Index $K$ an
alle weiteren Folgenglieder $a_n$ durch $x$ nach unten abgesch\"atzt werden.
Wir nennen $M$ daher die Menge der \emph{unteren Majorit\"ats-Schranken} der Folge
$(a_n)_{n\in\mathbb{N}}$, denn jedes Element aus $M$ ist eine untere Schranke f\"ur die Mehrheit der
Folgenglieder.  Genauer gilt f\"ur jedes $x \in M$, dass nur endlich viele der Folgenglieder $a_n$
kleiner als $x$ sind.  Mathematiker sagen an dieser Stelle, dass \emph{fast alle} Folgenglieder
kleiner als $x$ sind.

Es ist klar, dass $M$ nach unten abgeschlossen ist, es gilt 
\\[0.2cm]
\hspace*{1.3cm}
$y < x \wedge x \in M \rightarrow y \in M$,
\\[0.2cm]
denn wenn $x$ eine untere Schranke der Mehrheit aller Folgenglieder ist, dann ist sicher jede Zahl 
$y$, die kleiner als $x$ ist, ebenfalls eine untere Schranke der Mehrheit der Folgenglieder. 
Wir werden diese Eigenschaft sp\"ater ben\"otigen. 

Da die Folge $\folge{a_n}$ eine Cauchy-Folge ist, gibt es ein $S$, so dass die Folge durch $S$
beschr\"ankt ist, genauer gilt
\\[0.2cm]
\hspace*{1.3cm}
$\forall n \in \mathbb{N}: |a_n| \leq S$.
\\[0.2cm]
Daher ist die Menge $M$ nach oben durch $S$ beschr\"ankt, denn eine untere Grenze f\"ur die Mehrheit
aller Folgenglieder kann sicher nicht gr\"o\ss{}er als $S$ sein.
Weiter impliziert die Beschr\"anktheit der Cauchy-Folge, dass die Menge $M$ nicht leer ist,
denn wenn f\"ur alle $n \in \mathbb{N}$ die Ungleichung $|a_n| \leq S$ gilt,  dann gilt insbesondere
$-S \leq a_n$ und daraus folgt sofort $-S \in M$.  Als nicht-leere und nach oben
beschr\"ankte Menge hat $M$ ein Supremum, denn $\langle \mathbb{R}, \leq \rangle$ ist eine
vollst\"andige Ordnung.  Wir definieren  
\\[0.2cm]
\hspace*{1.3cm}
$s := \sup(M)$
\\[0.2cm]
und zeigen, dass mit dieser Definition
\\[0.2cm]
\hspace*{1.3cm}
$ \lim\limits_{n\rightarrow\infty} a_n = s $
\\[0.2cm]
gilt.  Sei $\varepsilon > 0$ gegeben.  Wir suchen eine Zahl $K$, so dass f\"ur alle 
nat\"urlichen Zahlen $n \geq K$ die Ungleichung
\\[0.2cm]
\hspace*{1.3cm}
$ \bigl|a_n - s \bigr| < \varepsilon $
\\[0.2cm]
gilt.  Wir betrachten zun\"achst die Zahl $s-\frac{\varepsilon}{2}$.  Wegen
$s-\frac{\varepsilon}{2} < s$ und $s = \sup(M)$ folgt $s-\frac{\varepsilon}{2} \in M$, denn $M$
ist nach unten abgeschlossen.  Damit existiert
dann nach Definition der Menge $M$ als Menge der unteren Majorit\"ats-Schranken
eine Zahl $K_1$, so dass f\"ur alle $n\in\mathbb{N}$ mit $n\geq K_1$ die Ungleichung
\begin{equation}
  \label{eq:ineq3}
  s-\frac{\varepsilon}{2} \leq a_n
\end{equation}
gilt.  
Da die Folge $\folge{a_n}$ eine Cauchy-Folge ist, gibt es eine Zahl $K_2$, so dass
f\"ur alle $m,n\in\mathbb{N}$ mit $m>K_2$ und $n>K_2$ die Ungleichung
\begin{equation}
  \label{eq:ineq4}
  \bigl| a_n - a_m \bigr| < \frac{\varepsilon}{2}
\end{equation}
gilt.  Wir setzen nun $K = \max(K_1,K_2)$ und betrachten die Zahl
$s+\frac{\varepsilon}{2}$, die wegen 
$s < s+\frac{\varepsilon}{2}$ sicher kein Element von $M$ mehr ist, denn sonst w\"are $s$ nicht das
Supremum von $M$.
Nach Definition von $M$ finden wir
dann eine nat\"urliche Zahl $m$, die gr\"o\ss{}er als $K$ ist, so dass
\begin{equation}
  \label{eq:ineq5}
  a_m < s + \frac{\varepsilon}{2} 
\end{equation}
gilt.  F\"ur diese Zahl $m$ gilt sicher auch die Ungleichung (\ref{eq:ineq3}), so dass wir
insgesamt
\\[0.2cm]
\hspace*{1.3cm}
$\ds s - \frac{\varepsilon}{2} \leq a_m < s + \frac{\varepsilon}{2} $
\\[0.2cm]
haben.  Daraus folgt sofort
\begin{equation}
  \label{eq:ineq6}
 \bigl| a_m - s \bigr| \leq \frac{\varepsilon}{2}.  
\end{equation}
Aufgrund der Ungleichung (\ref{eq:ineq4}) haben wir jetzt f\"ur alle nat\"urlichen Zahlen
$n>K$ die folgende Kette von Ungleichungen:
\\[0.2cm]
\hspace*{1.3cm}
$
\begin{array}{lcl}
  \bigl| a_n - s \bigr| &   =  & \bigl| (a_n - a_m) + (a_m - s) \bigr| \\[0.2cm] 
                        & \leq & \bigl| (a_n - a_m) \bigr| + \bigl| (a_m - s) \bigr| \\[0.2cm] 
                        &  <   & \ds\frac{\varepsilon}{2} + \frac{\varepsilon}{2}  \\[0.2cm] 
                        &  =   & \varepsilon  
\end{array}
$
\\[0.2cm]
Damit ist der Beweis abgeschlossen. \qed 



\section{Berechnung der Quadrat-Wurzel}
Wir pr\"asentieren nun eine Anwendung der bisher entwickelte Theorie und zeigen, wie die
Quadrat-Wurzel einer reellen Zahl berechnet werden kann.  Es sei eine reelle Zahl $a>0$
gegeben.  Gesucht ist eine reelle Zahl $b>0$, so dass $b^2 = a$ ist. Unsere Idee ist es,
die Zahl $b$ iterativ als L\"osung einer Fixpunkt-Gleichung zu berechnen.  
Wir definieren eine Folge $b_n$ induktiv wie folgt:
\begin{enumerate}
\item[I.A.:] $n=1$.  
      \\[0.2cm]
      \hspace*{1.3cm}      
      $b_1 := \left\{ \begin{array}{ll}
                      1 & \mbox{falls}\; a \leq 1, \\
                      a & \mbox{sonst}.
               \end{array}\right.
      $
\item[I.S.:] $n \mapsto n+1$.
      \\[0.2cm]
\hspace*{1.3cm}
$\ds b_{n+1} := \frac{1}{2}\cdot  \left(b_n + \frac{a}{b_n}\right). $
\end{enumerate}
Um diese Definition zu verstehen, nehmen wir zun\"achst an, dass der Grenzwert dieser Folge
existiert und den Wert $b \not= 0$ hat.  Dann gilt
\\[0.2cm]
\hspace*{1.3cm}
$ \begin{array}{lcl}
   b & = &  \lim\limits_{n\rightarrow\infty} b_n     \\[0.3cm]
     & = &\lim\limits_{n\rightarrow\infty} b_{n+1}   \\[0.3cm]
     & = & \ds\lim\limits_{n\rightarrow\infty} \frac{1}{2}\cdot  \left(b_n + \frac{a}{b_n}\right) \\[0.3cm]
     & = & \ds\frac{1}{2}\cdot  \Bigl(\lim\limits_{n\rightarrow\infty} b_n + \frac{a}{\lim\limits_{n\rightarrow\infty}b_n}\Bigr) \\[0.3cm]
     & = & \ds\frac{1}{2}\cdot  \left(b + \frac{a}{b}\right) \\[0.3cm]
\end{array}
$
\\[0.2cm]
Damit ist $b$ also eine L\"osung der Gleichung 
$b = \frac{1}{2}\cdot  \left(b + \frac{a}{b}\right)$.  
Wir formen diese Gleichung um:
\\[0.2cm]
\hspace*{1.3cm}
$
\begin{array}{lcll}
                & \ds b     = \frac{1}{2}\cdot  \left(b + \frac{a}{b}\right) & \mid \cdot  2 \\[0.4cm]
\Leftrightarrow & \ds 2 \cdot  b = b + \frac{a}{b}                           & \mid - b  \\[0.3cm]
\Leftrightarrow & \ds b = \frac{a}{b}                                   & \mid \cdot  b  \\[0.3cm]
\Leftrightarrow & b^2 = a                                            & \mid \sqrt{\;\;}  \\[0.2cm]
\Leftrightarrow & b  = \sqrt{a}                                      & 
\end{array}
$
\\[0.2cm]
Falls die oben definierte Folge $\folge{b_n}$ einen Grenzwert hat, dann ist dieser
Grenzwert also die Wurzel der Zahl $a$.  Wir werden die Konvergenz der Folge nachweisen,
indem wir zeigen, dass die Folge $\folge{b_n}$ einerseits monoton fallend und andererseits 
nach unten beschr\"ankt ist.  Dazu betrachten wir zun\"achst die Differenz $b_{n+1}^2 - a$:
\\[0.2cm]
\hspace*{1.3cm}
$
\begin{array}{lcl}
  b_{n+1}^2 - a & =    & \ds\frac{1}{4} \cdot  \left(b_n + \frac{a}{b_n}\right)^2 - a \\[0.4cm]
                & =    & \ds\frac{1}{4} \cdot  \left(b_n^2 + 2\cdot a + \frac{a^2}{b_n^2}\right) - a \\[0.4cm]
                & =    & \ds\frac{1}{4} \cdot  \left(b_n^2 - 2\cdot a + \frac{a^2}{b_n^2}\right) \\[0.4cm]
                & =    & \ds\frac{1}{4} \cdot  \left(b_n - \frac{a}{b_n}\right)^2 \\[0.4cm]
                & \geq & 0, 
\end{array}
$
\\[0.2cm]
denn das Quadrat einer reellen Zahl ist immer gr\"o\ss{}er-gleich Null.
Addieren wir auf beiden Seiten der Ungleichung
\\[0.2cm]
\hspace*{1.3cm}
$b_{n+1}^2 - a \geq 0$
\\[0.2cm]
die Zahl  $a$, so haben wir
\\[0.2cm]
\hspace*{1.3cm}
 $b_{n+1}^2 \geq a$ \quad und damit auch \quad $b_{n+1} \geq \sqrt{a}$ \quad f\"ur alle $n \in \mathbb{N}$
\\[0.2cm]
gezeigt.  Nach unserer Definition der Folge $\folge{b_n}$ gilt diese Ungleichung auch f\"ur
den ersten Wert $n=1$, so dass wir also insgesamt die Ungleichung
\\[0.2cm]
\hspace*{1.3cm}
$b_n^2 \geq a$ \quad und \quad $b_n \geq \sqrt{a}$ \quad f\"ur alle $n \in \mathbb{N}$
\\[0.2cm]
gezeigt haben.   Daraus folgt, dass $\sqrt{a}$ eine untere Schranke der Folge $\folge{b_n}$
ist.  Dividieren wir die Ungleichung $b_n^2 \geq a$ durch $b_{n}$, so folgt
\\[0.2cm]
\hspace*{1.3cm}
$\ds b_{n} \geq \frac{a}{b_{n}}.  $
\\[0.2cm]
Die Zahl $\frac{1}{2}\cdot \left(b_n + \frac{a}{b_n}\right)$ ist der arithmetische
Mittelwert der Zahlen $b_n$ und $\frac{a}{b_n}$ und muss daher zwischen diesen beiden
Zahlen liegen:
\\[0.2cm]
\hspace*{1.3cm}
$\ds b_{n} \geq \frac{1}{2}\cdot \left(b_n + \frac{a}{b_n}\right) \geq \frac{a}{b_n}. $
\\[0.2cm]
Dieser Mittelwert ist aber gerade $b_{n+1}$, es gilt also
\\[0.2cm]
\hspace*{1.3cm}
$\ds b_{n} \geq b_{n+1} \geq \frac{a}{b_n}. $
\\[0.2cm]
Dies zeigt, dass die Folge $\folge{b_n}$ monoton fallend ist und da wir oben gesehen haben, dass die
Folge durch $\sqrt{a}$ nach unten beschr\"ankt ist, konvergiert die Folge.
Wir hatten oben schon gezeigt, dass der Grenzwert dieser Folge dann den Wert $\sqrt{a}$ haben muss, es
gilt also
\\[0.2cm]
\hspace*{1.3cm}
$ \lim\limits_{n\rightarrow\infty} b_n = \sqrt{a} $
\\[0.2cm]
Abbildung \ref{fig:sqrt.stlx} auf Seite \pageref{fig:sqrt.stlx}
zeigt die Definition einer Prozedur \texttt{mySqrt}()
in \textsc{SetlX}, die die ersten 9 Glieder der Folge  
berechnet und dann jeweils mit Hilfe der Funktion $\texttt{nDecimalPlaces}()$ die ersten
100 Stellen der Werte ausgibt.  

Die von diesem Programm berechnete Ausgabe ist in
Abbildung \ref{fig:sqrt-output} gezeigt.  Sie k\"onnen sehen, dass die Folge sehr schnell
konvergiert.   $b_2$ stimmt auf 2 Stellen hinter dem Komma mit dem Ergebnis \"uberein,
bei $b_3$ sind es bereits 5 Stellen, bei $b_4$ sind es 11 Stellen, bei $b_5$ sind es 23
Stellen, bei $b_6$ sind es 47 Stellen, bei $b_7$ haben wir 96 Stellen und ab dem Folgeglied $b_8$ 
\"andern sich die ersten 100 Stellen hinter dem Komma nicht mehr.  

In modernen Mikroprozessoren wird \"ubrigens eine verfeinerte Version des in diesem
Abschnitt beschriebenen Verfahrens eingesetzt.  Die Verfeinerung besteht im wesentlichen
darin, dass zun\"achst ein guter Startwert $b_1$ in einer Tabelle nachgeschlagen wird, die
restlichen Folgeglieder werden dann in der Tat \"uber die Rekursionsformel 
\\[0.2cm]
\hspace*{1.3cm}
      $\ds b_{n+1} = \frac{1}{2}\cdot  \left(b_n + \frac{a}{b_n}\right)$
\\[0.2cm]
berechnet.

\begin{figure}[!ht]
  \centering
\begin{Verbatim}[ frame         = lines, 
                  framesep      = 0.3cm, 
                  labelposition = bottomline,
                  numbers       = left,
                  numbersep     = -0.2cm,
                  xleftmargin   = 1.3cm,
                  xrightmargin  = 1.3cm,
                ]
    mySqrt := procedure(a) {
        if (a <= 1) {
            b := 1; 
        } else {
            b := a; 
        }     
        for (n in [1 .. 9]) {
            b := 1/2 * (b + a/b);
            print("$n$: $nDecimalPlaces(b, 100)$");
        }
        return b;
    };
\end{Verbatim}
\vspace*{-0.3cm}
  \caption{Ein \textsl{SetlX}-Programm zur iterativen Berechnung der Quadrat-Wurzel.}
  \label{fig:sqrt.stlx}
\end{figure} %\$

\begin{figure}[!ht]
  \centering
{\footnotesize
\begin{Verbatim}[ frame         = lines, 
                  framesep      = 0.3cm, 
                  labelposition = bottomline,
                  numbers       = none,
                  numbersep     = -0.2cm,
                  xleftmargin   = -0.5cm,
                  xrightmargin  = 0.5cm,
                ]
1: 1.5000000000000000000000000000000000000000000000000000000000000000000000000000000000000000000000000000
2: 1.4166666666666666666666666666666666666666666666666666666666666666666666666666666666666666666666666666
3: 1.4142156862745098039215686274509803921568627450980392156862745098039215686274509803921568627450980392
4: 1.4142135623746899106262955788901349101165596221157440445849050192000543718353892683589900431576443402
5: 1.4142135623730950488016896235025302436149819257761974284982894986231958242289236217849418367358303565
6: 1.4142135623730950488016887242096980785696718753772340015610131331132652556303399785317871612507104752
7: 1.4142135623730950488016887242096980785696718753769480731766797379907324784621070388503875343276416016
8: 1.4142135623730950488016887242096980785696718753769480731766797379907324784621070388503875343276415727
9: 1.4142135623730950488016887242096980785696718753769480731766797379907324784621070388503875343276415727
\end{Verbatim}
}
\vspace*{-0.3cm}
  \caption{Berechnung der Quadrat-Wurzel mit Hilfe der Folge $b_{n+1} = \frac{1}{2}\cdot(b_n + \frac{a}{b_n})$.}
  \label{fig:sqrt-output}
\end{figure}


\exercise
\begin{enumerate}
\item[(a)] Es seien $a,b \in \mathbb{R}$.  Die Folge $(a_n)_{n\in\mathbb{N}}$ werde durch Induktion wie folgt
           definiert:
           \begin{enumerate}
           \item $a_1 := a$,
           \item $a_2 := b$,
           \item $a_{n+2} := \frac{1}{2} \cdot (a_n + a_{n+1})$.
           \end{enumerate}
           Zeigen Sie, dass die Folge $(a_n)_{n\in\mathbb{N}}$ konvergiert und berechnen Sie den Grenzwert
           $\lim\limits_{n\rightarrow\infty} a_n$  in Abh\"angigkeit von den Startwerten $a$ und $b$.

           \noindent
           \textbf{Hinweis}: Die Gleichung f\"ur $a_{n+2}$ ist eine Rekurrenz-Gleichung, die Sie \"uber den
           Ansatz $a_n = \lambda^n$ l\"osen k\"onnen.  Sie werden dabei f\"ur $\lambda$ zwei m\"ogliche Werte
           $\lambda_1$ und $\lambda_2$ finden, die Sie f\"ur die L\"osung in der Form
           \\[0.2cm]
           \hspace*{1.3cm}
           $a_n = c_1 \cdot \lambda_1^n + c_2 \cdot \lambda_2^n$
           \\[0.2cm]
           linear kombinieren m\"ussen.  Die Koeffizienten $c_1$ und $c_2$ k\"onnen Sie aus den Gleichungen f\"ur
           die Anfangswerte $a_1$ und $a_2$ bestimmen.  
\item[(b)] Es seien $a,b \in \mathbb{R}$ und zus\"atzlich gelte $a > 0$ und $b > 0$.  
           Die Folge $(a_n)_{n\in\mathbb{N}}$ werde durch Induktion wie folgt definiert:
           \begin{enumerate}
           \item $a_1 := a$,
           \item $a_2 := b$,
           \item $a_{n+2} := \sqrt{a_n \cdot a_{n+1}}$.
           \end{enumerate}
           Zeigen Sie, dass die Folge $(a_n)_{n\in\mathbb{N}}$ konvergiert und berechnen Sie den Grenzwert
           $\lim\limits_{n\rightarrow\infty} a_n$ in  Abh\"angigkeit von den Startwerten $a$ und $b$. \eox

           \noindent
           \textbf{Bemerkung}: F\"ur zwei positive Zahlen $a$ und $b$ wird die Zahl $c$ f\"ur die 
           \\[0.2cm]
           \hspace*{1.3cm}
           $c := \sqrt{a \cdot b\;}$
           \\[0.2cm]
           gilt, als das \href{http://de.wikipedia.org/wiki/Geometrisches_Mittel}{\emph{geometrische Mittel}} 
           von $a$ und $b$ bezeichnet. 

           \textbf{Hinweis}: Sie k\"onnen diese Teilaufgabe durch eine geeignete Transformation der
           Form 
           $x_n := f(a_n)$ in die Rekurrenz-Gleichung \"uberf\"uhren, die Sie in Teil (a) dieser Aufgabe
           bereits gel\"ost haben.  

\item[(c)] Es seien $a,b \in \mathbb{R}$ und zus\"atzlich gelte $a > 0$ und $b > 0$.  
           Die Folge $(a_n)_{n\in\mathbb{N}}$ werde durch Induktion wie folgt definiert:
           \begin{enumerate}
           \item $a_1 := a$,
           \item $a_2 := b$,
           \item $\ds\frac{1}{a_{n+2}} := \frac{1}{2} \cdot \left(\frac{1}{a_n} + \frac{1}{a_{n+1}}\right)$.
           \end{enumerate}
           Zeigen Sie, dass die Folge $(a_n)_{n\in\mathbb{N}}$ konvergiert und berechnen Sie den Grenzwert
           $\lim\limits_{n\rightarrow\infty} a_n$ in  Abh\"angigkeit von den Startwerten $a$ und $b$. \eox

           \noindent
           \textbf{Bemerkung}: F\"ur zwei positive Zahlen $a$ und $b$ wird die Zahl $c$ f\"ur die 
           \\[0.2cm]
           \hspace*{1.3cm}
           $\ds\frac{1}{c} := \frac{1}{2} \cdot \left(\frac{1}{a} + \frac{1}{b}\right)$
           \\[0.2cm]
           gilt, als das \href{http://de.wikipedia.org/wiki/Harmonisches_Mittel}{\emph{harmonische Mittel}} von $a$ und $b$ bezeichnet.
\end{enumerate}



\section{Reihen}
\begin{Definition}[Reihe]
Ist $\folge{a_n}$ eine Folge, so definieren wir die Folge der \emph{Partial-Summen} $\folge{s_n}$
durch die Festsetzung
\\[0.2cm]
\hspace*{1.3cm}
$\ds s_n := \sum\limits_{i=1}^n a_i. $
\\[0.2cm]
Diese Folge bezeichnen wir auch als unendliche  \emph{Reihe}.  Die Folge $\folge{a_n}$
bezeichnen wir als die 
der Reihe $\folge{\sum_{i=1}^n a_i}$ \emph{zugrunde liegende Folge}.
Falls die Folge der Partial-Summen konvergiert, so schreiben wir den Grenzwert als
\\[0.2cm]
\hspace*{1.3cm}
$\ds \sum\limits_{i=1}^\infty a_i := \lim\limits_{n\rightarrow\infty} \sum\limits_{i=1}^n a_i. $
\eod
\end{Definition}

\noindent
Gelegentlich treten in der Praxis Folgen $\folge{a_n}$ auf, f\"ur welche die Folgenglieder
$a_i$ erst ab einem Index $k>1$ definiert sind.  
Um auch aus solchen Folge bequem Reihen bilden zu k\"onnen, definieren wir in einem solchen
Fall die Partial-Summen $s_n$ durch
\\[0.2cm]
\hspace*{1.3cm}
$\ds s_n = \sum\limits_{i=k}^n a_i, $
\\ 
wobei wir vereinbaren, dass $\ds\sum_{i=k}^n a_i = 0$ ist, falls $k > n$ ist.

\begin{Satz}[Bernoullische Ungleichung]
Es sei $x \in \mathbb{R}$, $n \in \mathbb{N}_0$ und es gelte $x \geq -1$.   Dann gilt 
\\[0.2cm]
\hspace*{1.3cm}
$(1 + x)^n \geq 1 + n \cdot x$.
\\[0.2cm]
Diese Ungleichung wird als \emph{Bernoullische Ungleichung}
(\href{http://en.wikipedia.org/wiki/Jacob_Bernoulli}{Jakob Bernoulli}, 1655-1705)
bezeichnet.
\end{Satz}

\proof
Wir beweisen die Ungleichung durch vollst\"andige Induktion f\"ur alle $n \in \mathbb{N}_0$.
\begin{enumerate}
\item[I.A.:] $n = 0$.  Es gilt
             \\[0.2cm]
             \hspace*{1.3cm}
             $(1 + x)^0 = 1 \geq 1 = 1 + 0 \cdot x$. \quad  $\checkmark$
\item[I.S.:] $n \mapsto n + 1$.   Nach Induktions-Voraussetzung gilt
             \begin{equation}
               \label{eq:Bernoulli}
             (1 + x)^n \geq 1 + n \cdot x.
             \end{equation}
             Da $x \geq -1$ ist, folgt $1 + x \geq 0$, so dass wir die Ungleichung
             \ref{eq:Bernoulli} mit $1 + x$ multiplizieren k\"onnen.  Dann erhalten wir die 
             folgende Ungleichungs-Kette
             \\[0.2cm]
             \hspace*{1.3cm}
             $
             \begin{array}[t]{lcl}
             (1 + x)^{n+1} & \geq & (1 + n \cdot x) \cdot (1 + x)     \\[0.2cm]
                           & =    & 1 + (n + 1) \cdot x + n \cdot x^2 \\[0.2cm]
                           & \geq & 1 + (n + 1) \cdot x 
             \end{array}
             $
             \\[0.2cm]
             Also haben wir insgesamt
             \\[0.2cm]
             \hspace*{1.3cm}
             $(1 + x)^{n+1} \geq 1 + (n + 1) \cdot x$ 
             \\[0.2cm]
             gezeigt und das ist die Behauptung f\"ur $n+1$. $\checkmark$ \qed
\end{enumerate}

\begin{Satz}
Es sei $q \in \mathbb{R}$ mit $|q| < 1$.  Dann gilt
\\[0.2cm]
\hspace*{1.3cm}
$\ds\lim\limits_{n\rightarrow\infty} q^n = 0$.
\end{Satz}

\proof
Wir nehmen zun\"achst an, dass $q$ positiv ist. Aus $q < 1$ folgt dann 
\\[0.2cm]
\hspace*{1.3cm}
$\ds 1 < \frac{1}{q}$ \quad und damit \quad $\ds 0 < \frac{1}{q} - 1$. 
\\[0.2cm]
Wir definieren nun
\\[0.2cm]
\hspace*{1.3cm}
$\ds x := \frac{1}{q} - 1$. 
\\[0.2cm]
Mit Hilfe der Bernoullischen Ungleichung sehen wir nun, dass Folgendes gilt:
\\[0.2cm]
\hspace*{1.3cm}
$
\begin{array}[t]{lcl}
  \ds\bruch{1}{q^n} & =    & \ds\left(1 + \left(\bruch{1}{q} - 1\right)\right)^n  \\[0.4cm]
                 & \geq & \ds 1 + n \cdot \left(\bruch{1}{q} - 1\right)         \\[0.4cm]
                 & =    & 1 + n \cdot x.
\end{array}
$
\\[0.2cm]
Durch Invertierung dieser Ungleichung erhalten wir
\\[0.2cm]
\hspace*{1.3cm}
$\ds q^n \leq \frac{1}{1 + n \cdot x}$
\\[0.2cm]
Ist nun ein $\varepsilon > 0$ gegeben, so definieren wir 
\\[0.2cm]
\hspace*{1.3cm}
$\ds K := \left(\frac{1}{\varepsilon} - 1\right) \cdot \frac{1}{x} + 1$.
\\[0.2cm]
Dann gilt f\"ur alle $n \geq K$:
\\[0.2cm]
\hspace*{1.3cm}
$
\begin{array}[t]{lrcl}
            & \ds\left(\frac{1}{\varepsilon} - 1\right) \cdot \frac{1}{x} + 1 & \leq & n             \\[0.4cm]
\Rightarrow & \ds\left(\frac{1}{\varepsilon} - 1\right) \cdot \frac{1}{x} & < & n             \\[0.4cm]
\Rightarrow & \ds\left(\frac{1}{\varepsilon} - 1\right)                    & < & n \cdot x     \\[0.4cm]
\Rightarrow & \ds\frac{1}{\varepsilon}                                     & < & 1 + n \cdot x \\[0.4cm]
\Rightarrow & \ds\frac{1}{1 + n \cdot x} & < & \varepsilon
\end{array}
$
\\[0.2cm]
Insgesamt haben wir nun f\"ur alle $n \geq K$ gezeigt, dass
\\[0.2cm]
\hspace*{1.3cm}
$\ds 0 < q^n \leq \frac{1}{1 + n \cdot x} < \varepsilon$
\\[0.2cm]
gilt, also haben wir f\"ur $n \geq K$
\\[0.2cm]
\hspace*{1.3cm}
$\left| q^n \right| < \varepsilon$.
\\[0.2cm]
F\"ur $q = 0$ ist diese Ungleichung offenbar auch g\"ultig und wenn $q$ negativ ist, gilt $-q > 0$, so dass die
Ungleichung f\"ur $-q$ gilt:
\\[0.2cm]
\hspace*{1.3cm}
$\left| (-q)^n \right| < \varepsilon$.
\\[0.2cm]
Wegen $\left| (-q)^n \right| = \left| q^n \right|$ folgt daraus also, dass f\"ur alle $q$ die Ungleichung
\\[0.2cm]
\hspace*{1.3cm}
$\left| q^n \right| < \varepsilon$ \quad f\"ur $n \geq K$
\\[0.2cm]
g\"ultig ist und damit ist die Behauptung bewiesen.  \qed

\noindent
Wir pr\"asentieren nun einige Beispiele f\"ur konvergente Reihen:
\begin{enumerate}
\item Wir betrachten die Folge $\left(\frac{1}{n\cdot (n+1)}\right)_{n\in\mathbb{N}}$.

      F\"ur die Partial-Summen zeigen wir durch Induktion \"uber $n$, dass 
      \begin{equation}
        \label{eq:seq0}        
      \sum\limits_{i=1}^n \frac{1}{i\cdot (i+1)} = 1 - \frac{1}{n+1}
      \end{equation}
      gilt.
      \begin{enumerate}
      \item (Induktions-Anfang) $n=1$: Einerseits haben wir f\"ur $n=1$
           \\[0.2cm]
           \hspace*{1.3cm}      
           $\ds\sum\limits_{i=1}^n \frac{1}{i\cdot (i+1)} = \sum\limits_{i=1}^1 \frac{1}{i\cdot (i+1)}
            = \frac{1}{1\cdot (1+1)} = \frac{1}{2}$,
           \\[0.2cm]
           andererseits gilt
           \\[0.2cm]
           \hspace*{1.3cm}      
           $\ds 1 - \frac{1}{n+1} = 1 - \frac{1}{1+1} = 1 - \frac{1}{2} = \frac{1}{2}$. $\checkmark$
           \pagebreak

      \item (Induktions-Schritt) $n \mapsto n+1$:  
            \\[0.2cm]
            \hspace*{1.3cm}      
            $
            \begin{array}{lcl}
              \ds\sum\limits_{i=1}^{n+1} \frac{1}{i\cdot (i+1)} 
              &               =  & \ds\sum\limits_{i=1}^{n} \frac{1}{i\cdot (i+1)} + \frac{1}{(n+1)\cdot (n+2)} \\[0.3cm]
              & \stackrel{IV}{=} & \ds 1 - \frac{1}{(n+1)} + \frac{1}{(n+1)\cdot (n+2)}                     \\[0.3cm]
              &               =  & \ds 1 - \frac{n+2 -1}{(n+1)\cdot (n+2)}                                  \\[0.3cm]
              &               =  & \ds 1 - \frac{n+1}{(n+1)\cdot (n+2)}                                     \\[0.3cm]
              &               =  & \ds 1 - \frac{1}{n+2}\quad \checkmark                                               
            \end{array}
            $
      \end{enumerate}
      Damit haben wir Gleichung (\ref{eq:seq0}) durch vollst\"andige Induktion nachgewiesen.
      Aus Gleichung (\ref{eq:seq0}) folgt nun 
      \\[0.2cm]
      \hspace*{1.3cm}      
      $\ds\sum\limits_{i=1}^\infty \frac{1}{i\cdot (i+1)} = \lim\limits_{n\rightarrow\infty} \sum\limits_{i=1}^n \frac{1}{i\cdot (i+1)} = 
       \lim\limits_{n\rightarrow\infty} \Bigl(1 - \frac{1}{n+1} \Bigr) = 1$.
\item Wir betrachten die Folge $\folge{q^n}$ f\"ur eine Zahl $q \in \mathbb{R}$.
      F\"ur die Partial-Summen gilt
      \\[0.2cm]
      \hspace*{1.3cm}      
      $\ds s_n = \sum\limits_{i=0}^n q^i$.
      \\[0.2cm]
      Wir betrachten den Ausdruck $(1-q) \cdot  s_n$: 
      \\[0.2cm]
      \hspace*{1.3cm}      
      $
      \begin{array}{lcl}
       (1-q) \cdot s_n & = & \ds(1 - q) \cdot  \sum\limits_{i=0}^n q^i \\[0.4cm]
                   & = & \ds\sum\limits_{i=0}^n q^i \;-\; q \cdot  \sum\limits_{i=0}^n q^i \\[0.4cm]
                   & = & \ds\sum\limits_{i=0}^n q^i \;-\; \sum\limits_{i=0}^n q^{i+1} \\[0.4cm]
                   & = & \ds\sum\limits_{i=0}^n q^i \;-\; \sum\limits_{i=1}^{n+1} q^{i} \\[0.5cm]
                   & = & \ds\left(q^0 + \sum\limits_{i=1}^n q^i\right) \;-\; \left(\sum\limits_{i=1}^{n} q^{i} + q^{n+1}\right) \\[0.5cm]
                   & = & \ds q^0 - q^{n+1} \\[0.3cm]
                   & = & \ds 1 - q^{n+1} 
      \end{array}
      $
      \\[0.2cm]
      Es gilt also 
      \\[0.2cm]
      \hspace*{1.3cm}      
      $\ds (1-q) \cdot  \sum\limits_{i=0}^{n} q^i = 1 - q^{n+1}$
      \\[0.2cm]
      Dividieren wir diese Gleichung durch $(1-q)$, so erhalten wir f\"ur die Partial-Summen
      den Ausdruck
      \\[0.2cm]
      \hspace*{1.3cm}      
      \colorbox{orange}{$\ds \sum\limits_{i=0}^{n} q^i = \frac{1 - q^{n+1}}{1-q}$}.
      \\[0.2cm]
      Falls $|q| < 1$ ist, konvergiert die Folge $\folge{q^n}$ gegen $0$.  Damit gilt
      f\"ur $|q| < 1$
      \\[0.2cm]
      \hspace*{1.3cm}
      \colorbox{orange}{$\ds\sum\limits_{i=0}^{\infty} q^i = \frac{1}{1-q}$}.
      \\[0.2cm]
      Die Reihe
      \\[0.2cm]
      \hspace*{1.3cm}
      $\ds\left(\sum\limits_{i=0}^{n} q^i\right)_{n\in\mathbb{N}}$ 
      \\[0.2cm]
      wird als die
      \href{http://de.wikipedia.org/wiki/Geometrische_Reihe}{\emph{geometrische Reihe}} bezeichnet.
      Diese Reihe ist mit Abstand die wichtigste Reihe, die Ihnen in der Informatik begegnen wird.
      \eox
\end{enumerate}

\exercise
Berechnen Sie den Wert der Reihe $\ds\sum\limits_{n=0}^\infty \frac{3}{n^2 + 3 \cdot n + 2}$. \eox

\begin{Definition}[Alternierende Reihe] 
Hat eine Reihe die Form 
\\[0.2cm]
\hspace*{1.3cm}
$\ds\Folge{\sum\limits_{i=1}^{n}  (-1)^i \cdot  a_i} $
\\[0.2cm]
und gilt entweder
\\[0.2cm]
\hspace*{1.3cm}
$ \forall i \in \mathbb{N}: a_i \geq 0 \quad \mbox{oder} \quad \forall i \in \mathbb{N}: a_i \leq 0 $,
\\[0.2cm]
so haben aufeinander folgende Glieder der Reihe ein unterschiedliches Vorzeichen (es sei denn, dass
die Glieder den Wert $0$ haben).  In diesem Fall sprechen wir daher von einer \emph{alternierenden Reihe}.  
\eod
\end{Definition}

\noindent
\textbf{Beispiel}: Die Reihe
\\[0.2cm]
\hspace*{1.3cm}
$\ds\Folge{\sum\limits_{i=1}^{n} \frac{(-1)^{i+1}}{i}}$
\\[0.2cm]
ist eine alternierende Reihe, denn die Vorzeichen aufeinander folgender Reihenglieder sind alternierend positiv und negativ:
\\[0.2cm]
\hspace*{1.3cm}
$\ds\sum\limits_{i=1}^{\infty} \frac{(-1)^{i+1}}{i} = \frac{1}{1} - \frac{1}{2} + \frac{1}{3} - \frac{1}{4} \pm \cdots$.
\\[0.2cm]
Wir werden sp\"ater sehen, dass diese Reihe gegen den Wert $\ln(2)$ konvergiert.
\eox


\begin{Definition}[Null-Folge]
Die Folge $\folge{a_n}$ ist eine \emph{Null-Folge} wenn gilt:
\\[0.2cm]
\hspace*{1.3cm}
$ \lim\limits_{n\rightarrow\infty} a_n = 0. $ \eod
\end{Definition}

\exercise
Die Folge $\folge{a_n}$ sei eine monoton fallende \emph{Null-Folge}.  Zeigen Sie, dass dann
\\[0.2cm]
\hspace*{1.3cm}
$\forall n \in \mathbb{N}: 0 \leq a_n$
\\[0.2cm]
gilt.  

\hint
F�hren Sie den Beweis indirekt.
\eox



\begin{Satz}[Leibniz-Kriterium, (\href{http://de.wikipedia.org/wiki/Leibniz}{Gottfried Wilhelm Leibniz}, 1646-1716)] \lb
Wenn die Folge $\folge{a_n}$ eine monoton fallende Null-Folge ist, dann 
konvergiert die alternierende Reihe
\\[0.2cm]
\hspace*{1.3cm}
$\ds \left(\sum\limits_{i=1}^{n} (-1)^i \cdot  a_i\right)_{n\in\mathbb{N}.} $
\end{Satz}
\pagebreak  

\noindent
\textbf{Beweis}:  Die  Partial-Summen $s_n$ sind durch
\\[0.2cm]
\hspace*{1.3cm}
$\ds s_n = \sum\limits_{i=1}^{n} (-1)^i \cdot  a_i $
\\[0.2cm]
definiert.  Wir betrachten zun\"achst die Folge der Partial-Summen mit geraden Indizes, wir
betrachten also die Folge $\folge{s_{2 \cdot n}}$ und zeigen, dass diese Folge monoton fallend
ist.  Es gilt
\begin{equation}
  \label{eq:seq1}
 s_{2\cdot(n+1)} = s_{2\cdot n} + (-1)^{2\cdot n+1}\cdot a_{2 \cdot n+1} + (-1)^{2 \cdot n+2} \cdot  a_{2\cdot n+2} 
                 = s_{2 \cdot n} - a_{2 \cdot n+1} + a_{2 \cdot n+2}. 
\end{equation}
Daraus folgt
\\[0.2cm]
\hspace*{1.3cm}
$
\begin{array}{lrcl}
                 &   s_{2\cdot(n+1)} & \leq & s_{2\cdot n}       \\[0.2cm]
 \Leftrightarrow & s_{2\cdot n} - a_{2\cdot n+1} + a_{2\cdot n+2} & \leq &s_{2\cdot n}   \\[0.2cm]
 \Leftrightarrow &  a_{2\cdot n+2} & \leq & a_{2\cdot n+1}        
\end{array}
$
\\[0.2cm]
Die letzte Ungleichung ist aber nichts anderes als die Monotonie der Folge $\folge{a_n}$.

Als n\"achstes zeigen wir durch vollst\"andige Induktion, dass die Folge der Partial-Summen
nach unten beschr\"ankt ist, genauer gilt
\\[0.2cm]
\hspace*{1.3cm}
$s_{n} \geq - a_1$ \quad f\"ur alle $n \in \mathbb{N}$.
\\[0.2cm]
Um diese Ungleichung nachzuweisen, zeigen wir zun�chst durch vollst\"andige Induktion, dass f\"ur alle
$n\in\mathbb{N}_0$ gilt:
      \\[0.2cm]
      \hspace*{1.3cm}      
      $s_{2\cdot n+1} \geq - a_1$.
\begin{enumerate}
\item[I.A.:] $n=0$.
      \\[0.2cm]
      \hspace*{1.3cm} $s_{2\cdot 0+1} = s_1 = -a_1 \geq -a_1$.
\item[I.S.:] $n \mapsto n+1$ 
             \\[0.2cm]
\hspace*{1.3cm}
$ 
             \begin{array}{lcll}
               s_{2\cdot(n+1)+1} & = &  s_{2\cdot n+1} + a_{2\cdot n+2} - a_{2\cdot n+3}      \\[0.2cm]
                            & \geq & - a_1 + a_{2\cdot n+2} - a_{2\cdot n+3} & \mbox{nach Induktions-Voraussetzung}   \\[0.2cm]
                            & \geq & - a_1 & \mbox{wegen $a_{2\cdot n+2} \geq a_{2\cdot n+3}$.} \\[0.2cm]
                          
             \end{array}
             $
\\[0.2cm]
\end{enumerate}
Nun gilt f\"ur $n \in \mathbb{N}$
      \\[0.2cm]
      \hspace*{1.3cm}      
      $s_{2 \cdot n} = s_{2 \cdot n-1} + a_{2 \cdot n} \geq s_{2 \cdot n-1} \geq - a_1$.
      \\[0.2cm]
Da wir nun gezeigt haben, dass die Folge $\folge{s_{2 \cdot n}}$ sowohl monoton fallend als auch
nach unten beschr\"ankt ist, folgt aus Satz \ref{satz:monoton}, dass diese Folge konvergent ist.
Der Grenzwert dieser Folge sei $s$:
\\[0.2cm]
\hspace*{1.3cm}
$ s := \lim\limits_{n\rightarrow\infty} s_{2 \cdot n}. $
\\[0.2cm]
Dann konvergiert auch die Folge $\folge{s_{n}}$ gegen $s$. Dies sehen wir wie folgt:
Sei $\varepsilon > 0$ gegeben.  Weil   $\folge{s_{2 \cdot n}}$ gegen $s$ konvergiert, gibt es eine
Zahl $K_1$, so dass f\"ur alle $n \geq K_1$ die Ungleichung
\begin{equation}
  \label{eq:seq2}
  \bigl| s_{2 \cdot n} - s \bigr| < \frac{1}{2}\cdot\varepsilon
\end{equation}
erf\"ullt ist.  Weil $\folge{a_{n}}$ eine Null-Folge ist, gibt es au\ss{}erdem eine Zahl $K_2$,
so dass f\"ur alle $n \geq K_2$ die Ungleichung
\begin{equation}
  \label{eq:seq3}
  \bigl| a_n - 0 \bigr| < \frac{1}{2} \cdot \varepsilon
\end{equation}
gilt.  Wir setzen $K:= \max(2 \cdot K_1 + 1,K_2)$ und zeigen, dass f\"ur alle $n \geq K$ die Ungleichung
\\[0.2cm]
\hspace*{1.3cm}
$ \bigl| s_n - s \bigr| < \varepsilon $
\\[0.2cm]
gilt.  Wir erbringen diesen Nachweis \"uber eine Fall-Unterscheidung:
\begin{enumerate}
\item $n$ ist gerade, also gilt $n= 2 \cdot m$.
      \\[0.2cm]
\hspace*{1.3cm}
$ \begin{array}{lcl}
        \bigl| s_n - s \bigr| & = & \bigl| s_{2 \cdot m} - s \bigr| \\[0.2cm]
                              & < & \ds\frac{1}{2}\cdot\varepsilon \\[0.2cm]
                              & < & \varepsilon,
         \end{array}
      $
\\[0.2cm]
      denn aus $n=2 \cdot m$ und $n \geq K$ folgt $m \geq K_1$.
\item $n$ ist ungerade, also gilt $n= 2 \cdot m+1$.
      \\[0.2cm]
\hspace*{1.3cm}
$ \begin{array}{lcl}
        \bigl| s_n - s \bigr| & = & \bigl| s_{2 \cdot m+1} - s \bigr| \\[0.2cm]
                              & = & \bigl| s_{2 \cdot m+1} - s_{2 \cdot m} + s_{2 \cdot m} - s \bigr| \\[0.2cm]
                              & \leq & \bigl| s_{2 \cdot m+1} - s_{2 \cdot m} \bigr| + \bigl| s_{2 \cdot m} - s \bigr| \\[0.2cm]
                              & < & \ds\bigl| a_{2 \cdot m+1} \bigr| + \frac{1}{2}\cdot\varepsilon \\[0.2cm]
                              & < & \ds\frac{1}{2}\cdot\varepsilon + \frac{1}{2}\cdot\varepsilon \\[0.2cm]
                              & = & \varepsilon,
         \end{array}
       $
\\[0.2cm]
       denn aus $n=2\cdot m+1$ und $n \geq K$ folgt $m \geq K_1$ und $n \geq K_2$.
\end{enumerate}
Damit ist der Beweis abgeschlossen. \qed 

\begin{Satz}[Cauchy'sches Konvergenz-Kriterium f\"ur Reihen] 
Die Reihe $\ds\folge{\sum_{i=0}^{n} a_i}$ ist genau dann konvergent, wenn es f\"ur alle 
$\varepsilon>0$ eine Zahl $K$ gibt, so dass 
\\[0.2cm]
\hspace*{1.3cm}
$\ds \forall n,l \in \mathbb{N}:\bigl( n \geq K \rightarrow \left|\sum\limits_{i=n+1}^{n+l} a_i \right| < \varepsilon\bigr)$ 
\\[0.2cm]
gilt.
\end{Satz}

\noindent
\textbf{Beweis}:  Nach den S\"atzen aus dem Abschnitt \"uber Folgen ist die Folge
$\folge{s_n}$ der durch
\\[0.2cm]
\hspace*{1.3cm}
$\ds s_n = \sum\limits_{i=0}^{n} a_i $
\\[0.2cm]
definierten Partial-Summen genau dann konvergent, wenn $\folge{s_n}$ eine Cauchy-Folge
ist, wenn also gilt:
\\[0.2cm]
\hspace*{1.3cm}
$ \forall \varepsilon \in \mathbb{R}_+: \exists K \in \mathbb{R}: \forall m,n \in \mathbb{N}: 
   \bigl(m \geq K \wedge n \geq K \rightarrow \bigl| s_m - s_n \bigr| < \varepsilon\bigr).
$
\\[0.2cm]
In der letzten Formel k\"onnen wir ohne Einschr\"ankung der Allgemeinheit annehmen,
dass $n \leq m$ gilt.  Dann ist $m = n + l$ f\"ur eine nat\"urliche Zahl $l$.  Setzen wir hier
die Definition der Partial-Summen ein, so erhalten wir
\\[0.2cm]
\hspace*{1.3cm}
$ \begin{array}{lcl}
   \left| s_{m} - s_n \right| & = & \left| s_{n+l} - s_n \right| \\[0.2cm]
                              & = & \ds \left| \sum\limits_{i=1}^{n+l} a_i - \sum\limits_{i=1}^n a_i \right| \\[0.4cm]
                              & = & \ds \left| \sum\limits_{i=n+1}^{n+l} a_i \right| \\[0.2cm]
   \end{array}
$
\\[0.2cm]
und damit ist klar, dass die Ungleichung des Satzes \"aquivalent dazu ist,
dass die Folge der Partial-Summen eine Cauchy-Folge ist.
\qed

\begin{Korollar} \lb
Wenn die Reihe $\ds\Reihe{a_i}$ konvergent ist, dann ist die Folge $\folge{a_n}$ eine Null-Folge.  
\end{Korollar}

\noindent
\textbf{Beweis}: Nach dem Cauchy'schen Konvergenz-Kriterium gilt \\[0.2cm]
\hspace*{1.3cm}      
$\ds\forall \varepsilon \in \mathbb{R}_+:\exists K \in \mathbb{R}:\forall n,l \in \mathbb{N}: \Bigl(n \geq K \rightarrow \left|\sum\limits_{i=n+1}^{n+l} a_i \right| < \varepsilon\Bigr)$.
\\[0.2cm]
Setzen wir hier $l=1$ so haben wir insbesondere
\\[0.2cm]
\hspace*{1.3cm}
$\ds\forall \varepsilon \in \mathbb{R}_+:\exists K \in \mathbb{R}:\forall n \in \mathbb{N}: \Bigl(n \geq K \rightarrow \left|\sum\limits_{i=n+1}^{n+1} a_i \right| < \varepsilon\Bigr)$. 
\\[0.2cm]
Wegen \\[0.2cm]
      \hspace*{1.3cm}      
      $\ds\left|\sum\limits_{i=n+1}^{n+1} a_i \right| = |a_{n+1}|$
      \\[0.2cm]
folgt also 
\\[0.2cm]
\hspace*{1.3cm}
$\forall \varepsilon \in \mathbb{R}_+:\exists K \in \mathbb{R}:\forall n \in \mathbb{N}: \bigl(n \geq K \rightarrow |a_{n+1}| < \varepsilon\bigr)$.      
\\[0.2cm]
Diese Formel dr\"uckt aus, dass $\folge{a_n}$ eine Null-Folge ist.
\qed

\remark
Mit Hilfe des letzten Satzes k\"onnen wir zeigen, dass die 
\href{http://de.wikipedia.org/wiki/Harmonische_Reihe}{\emph{harmonische Reihe}}
\\[0.2cm]
\hspace*{1.3cm}
$\ds\Folge{\sum\limits_{i=1}^{n} \frac{1}{i}} $
\\[0.2cm]
divergiert.  W\"are diese Reihe konvergent, so g\"abe es nach dem Cauchy'schen
Konvergenz-Kriterium eine Zahl $K$, so dass f\"ur alle $n \geq K$ und alle $l$ die Ungleichung
\\[0.2cm]
\hspace*{1.3cm}
$\ds \sum\limits_{i=n+1}^{n+l} \frac{1}{i} < \frac{1}{2} $
\\[0.2cm]
gilt.  Insbesondere w\"urde diese Ungleichung dann f\"ur $l=n$ gelten. 
F\"ur beliebige $n$ gilt aber die folgende Absch\"atzung:
\\[0.2cm]
\hspace*{1.3cm}
$\ds\sum\limits_{i=n+1}^{n+n} \frac{1}{i}  \geq  \sum\limits_{i=n+1}^{2 \cdot n} \frac{1}{2 \cdot n} = n \cdot  \frac{1}{2 \cdot n} = \frac{1}{2}$
\\[0.2cm]
Damit erf\"ullt die harmonische Reihe das Cauchy'sche Konvergenz-Kriterium nicht. \eox

\begin{Satz}[Majoranten-Kriterium]
F\"ur die Folgen $\folge{a_n}$  und  $\folge{b_n}$  gelte:
\begin{enumerate}
\item $\forall n \in \mathbb{N}: 0 \leq a_n \leq b_n$.
\item Der Grenzwert $\ds\sum\limits_{i=1}^\infty b_i$ existiert.
\end{enumerate}
Dann existiert auch der Grenzwert $\ds\sum\limits_{i=1}^\infty a_i$.
\end{Satz}


\noindent
\textbf{Beweis}:  Es gilt
\\[0.2cm]
\hspace*{1.3cm}
$\ds \sum\limits_{i=1}^n a_i \leq \sum\limits_{i=1}^n b_i \leq \sum\limits_{i=1}^\infty b_i. $
\\[0.2cm]
Also ist die Folge $\Folge{\sum\limits_{i=1}^{n} a_i}$ monoton wachsend und beschr\"ankt und
damit konvergent.
\qed

\remark
Oft wird im Majoranten-Kriterium die Voraussetzung
\\[0.2cm]
\hspace*{1.3cm}
$\forall n \in \mathbb{N}: 0 \leq a_n \leq b_n$ 
\\[0.2cm] 
abgeschw\"acht zu 
\\[0.2cm]
\hspace*{1.3cm}      
$\forall n \in \mathbb{N}: \bigl(n \geq K \rightarrow 0 \leq a_n \leq b_n\bigr)$.
\\[0.2cm]
Hierbei ist $K$ dann eine geeignet gew\"ahlte Schranke.  Die G\"ultigkeit dieser Form des
Majoranten-Kriteriums folgt aus der Tatsache, dass das Ab\"andern endlich vieler
Glieder einer Reihe f\"ur die Frage, ob eine Reihe konvergent ist, unbedeutend ist.

\example
Wir zeigen  mit dem Majoranten-Kriterium, dass die Reihe 
$\ds\Folge{\sum\limits_{i=1}^{n} \frac{1}{i^2}}$ konvergiert.  Es gilt
\\[0.2cm]
\hspace*{1.3cm}
$
\begin{array}{lcl}
            & i + 1\geq i & \mid\; \cdot  (i+1) \\[0.2cm]
\Rightarrow & \ds (i+1)^2 \geq i\cdot (i + 1) & \mid \frac{1}{\cdot} \\[0.2cm]
\Rightarrow & \ds \frac{1}{(i+1)^2} \leq \frac{1}{i\cdot (i + 1)} 
\end{array}
$
\\[0.2cm]
Damit ist die Reihe $\ds\Folge{\sum\limits_{i=1}^{n}\frac{1}{i\cdot (i+1)}}$
eine konvergente Majorante der Reihe $\ds\Folge{\sum\limits_{i=1}^{n}\frac{1}{(i+1)^2}}$.  Wegen
\\[0.2cm]
\hspace*{1.3cm}
$\ds \sum\limits_{i=1}^{\infty} \frac{1}{i^2} = \frac{1}{1^2} + \sum\limits_{i=1}^{\infty} \frac{1}{(i+1)^2} $
\\[0.2cm]
folgt die Konvergenz aus dem Majoranten-Kriterium. \qed


\remark
Wir werden sp\"ater in dem Kapitel \"uber Fourier-Reihen zeigen, dass 
\\[0.2cm]
\hspace*{1.3cm}
$\ds\sum\limits_{i=1}^{\infty} \frac{1}{i^2} = \frac{\;\pi^2}{6\;} \quad\mbox{gilt.} $ \eox

\begin{Satz}[Minoranten-Kriterium]
F\"ur die Folgen $\folge{a_n}$  und  $\folge{b_n}$  gelte:
\begin{enumerate}
\item $\forall n \in \mathbb{N}: 0 \leq a_n \leq b_n$.
\item Der Grenzwert $\ds\sum\limits_{i=1}^\infty a_i$ existiert nicht.
\end{enumerate}
Dann existiert auch der Grenzwert $\ds\sum\limits_{i=1}^\infty b_i$ nicht.
\end{Satz}

\noindent
\textbf{Beweis}:  Wir f\"uhren den Beweis indirekt und nehmen an, dass der Grenzwert
$\sum\limits_{i=1}^\infty b_i$
existiert.  Nach dem Majoranten-Kriterium m\"usste dann auch der Grenzwert
$\sum\limits_{i=1}^\infty a_i$
existieren und dass steht im Widerspruch zur Voraussetzung. 
\qed

\example
Wir zeigen, dass die Reihe 
\\[0.2cm]
\hspace*{1.3cm}
 $\ds\Reihe{\frac{1}{\sqrt{i\,}}}$ \\[0.2cm]
nicht konvergiert.  Dazu
benutzen wir das Minoranten-Kriterium und zeigen, dass die
Reihe
\\[0.2cm]
\hspace*{1.3cm}
 $\ds\Reihe{\frac{1}{i}}$ 
\\[0.2cm]
eine divergente Minorante der Reihe
\\[0.2cm]
\hspace*{1.3cm}
  $\ds\Reihe{\frac{1}{\sqrt{i}}}$ 
\\[0.2cm]
ist, denn es gilt: 
\\[0.2cm]
\hspace*{1.3cm}
$
\begin{array}{llcll}
                 & \ds\frac{1}{i}   & \leq & \frac{1}{\sqrt{i}}  & \ds\mid \; \frac{1}{\cdot}\\[0.3cm]
\Leftrightarrow  & i              & \geq & \sqrt{i}             & \mid \;\cdot^2  \\[0.3cm]
\Leftrightarrow  & i^2            & \geq & i                    & \ds\mid \; \cdot \frac{1}{i}  \\[0.3cm]
\Leftrightarrow  & i              & \geq & 1                    & \mid \; 
\end{array}
$
\\[0.2cm]
Da die letzte Ungleichung offenbar f\"ur alle $i \in \mathbb{N}$ wahr ist, ist der Beweis abgeschlossen.
\qed

\begin{Satz}[Quotienten-Kriterium]
Es sei $\folge{a_n}$ eine Folge und $q\in\mathbb{R}$ eine Zahl, so dass gilt:
\begin{enumerate}
\item $0 \leq q < 1$
\item $\forall n \in \mathbb{N}: 0 \leq a_n$
\item $\forall n \in \mathbb{N}: a_{n+1} \leq q \cdot  a_n$
\end{enumerate}
Dann konvergiert die Reihe $\Reihe{a_i}$.
\end{Satz}

\noindent 
\textbf{Beweis}:  Wir zeigen, dass die geometrische Reihe
\\[0.2cm]
\hspace*{1.3cm}
 $\ds\Reihe{a_1\cdot q^i}$ 
\\[0.2cm]
eine konvergente Majorante der Reihe
\\[0.2cm]
\hspace*{1.3cm}
 $\ds\Reihe{a_i}$ 
\\[0.2cm]
ist.  Dazu zeigen wir durch Induktion \"uber $n$, dass folgendes gilt:
\\[0.2cm]
\hspace*{1.3cm}
$\forall n \in \mathbb{N}: a_{n} \leq a_1 \cdot q^{n-1}$
\begin{enumerate}
\item[I.A.]: $n = 1$.  Wegen $q^0 = 1$ gilt trivialerweise
      \\[0.2cm]
      \hspace*{1.3cm} $a_1  \leq a_1 \cdot  q^0$.
\item[I.S.]: $n \mapsto n+1$.  Es gilt: 
      \\[0.2cm]
      \hspace*{1.3cm}      
      $
      \begin{array}{lclll}
        a_{n+1} & \leq & q \cdot  a_n        & \mbox{nach Voraussetzung} \\[0.2cm]
                & \leq & q \cdot  a_1 \cdot  q^{n-1}  & \mbox{nach Induktions-Voraussetzung} \\[0.2cm]
                & =    & a_1 \cdot  q^{n}. & &\hspace*{2cm} \Box
              \end{array}
      $
\end{enumerate}

\remark
Beim Quotienten-Kriterium sind eigentlich nur die Betr\"age der
Folgenglieder $|a_n|$ wichtig, denn es l\"asst sich folgende Versch\"arfung des
Quotienten-Kriteriums zeigen:  
Ist $\folge{a_n}$ eine Folge, $q\in\mathbb{R}$ und $K \in \mathbb{R}$, so dass 
\\[0.2cm]
\hspace*{1.3cm}
$0 \leq q < 1$ \quad und \quad $\ds\forall n \in \mathbb{N}: \left(n \geq K \rightarrow \left|\frac{a_{n+1}}{a_n}\right| \leq q\right)$
\\[0.2cm]
gilt.  Dann konvergiert die Reihe $\Reihe{a_i}$.

\example
Wir zeigen mit Hilfe des Quotienten-Kriteriums, dass die Reihe
\\[0.2cm]
\hspace*{1.3cm}
$\ds\Reihe{\frac{z^i}{i!}}$ 
\\[0.2cm]
f\"ur alle $z \in \mathbb{C}$ konvergiert.  F\"ur $z=0$ ist die Konvergenz der
Reihe trivial und sonst betrachten wir
den Quotienten $a_{n+1}/a_n$ f\"ur diese Reihe, setzen $K = 2\cdot |z|$ und $\ds q = \frac{1}{2}$
und zeigen, dass das Quotienten-Kriterium erf\"ullt ist, denn f\"ur alle $n\geq K$ gilt:

\hspace*{1.3cm}      
$\ds\left|\frac{a_{n+1}}{a_n}\right| =  \frac{\;\frac{|z^{n+1}|}{\;(n+1)!\;}\;}{\frac{\;|z^{n}|\;}{n!}} 
                            =  \frac{\;|z^{n+1}| \cdot  n!\;}{\;|z^{n}| \cdot  (n+1)!\;} 
                            =  \frac{\;|z|\;}{\;n+1\;} 
                            \leq  \frac{\;|z|\;}{\;K\;} 
                            = \frac{\;|z|\;}{\;2\cdot |z|\;} 
                            =  \frac{\;1\;}{\;2\;} 
$.
\\[0.2cm]
Also gilt
\\[0.2cm]
\hspace*{1.3cm}
$\ds|a_{n+1}| \leq \frac{\;1\;}{\;2\;} \cdot |a_n|$
\\[0.2cm]
und damit ist das Quotienten-Kriterium erf\"ullt. \eox

\remark
Wir werden sp\"ater zeigen, dass die in dem letzten Beispiel pr\"asentierte Reihe die
Exponentialfunktion berechnet, denn es gilt  
\\[0.2cm]
\hspace*{1.3cm}
\colorbox{orange}{$\ds e^z = \sum\limits_{n=0}^{\infty} \frac{z^n}{n!}$} \quad f\"ur alle $z \in \mathbb{C}$.
\eox


\begin{Satz}[Wurzel-Kriterium]
Es sei $\folge{a_n}$ eine Folge und $q\in\mathbb{R}$ eine Zahl, so dass 
\begin{enumerate}
\item $0 \leq q < 1$
\item $\forall n \in \mathbb{N}: 0 \leq a_n$
\item $\ds\forall n \in \mathbb{N}: \sqrt[n]{a_n} \leq q$
\end{enumerate}
gilt.  Dann konvergiert die Reihe $\ds\Reihe{a_i}$.
\end{Satz}

\noindent 
\textbf{Beweis}: Auch hier k\"onnen wir den Nachweis der Konvergenz dadurch f\"uhren indem wir
zeigen, dass die geometrische Reihe $\Reihe{q^i}$ eine konvergente Majorante ist: F\"ur
$n>0$ gilt
\\[0.2cm]
\hspace*{1.3cm} $a_n \leq q^n \;\Leftrightarrow\; \sqrt[n]{a_n} \leq q$.
\qed

\remark
Beim Wurzel-Kriterium sind eigentlich nur die Betr\"age der
Folgenglieder $|a_n|$ wichtig, denn es l\"asst sich folgende Versch\"arfung des
Wurzel-Kriteriums zeigen:  
Ist $\folge{a_n}$ eine Folge, $q\in\mathbb{R}$ und $K \in \mathbb{R}$, so dass 
\\[0.2cm]
\hspace*{1.3cm}
$\ds 0 \leq q < 1 \quad \wedge \quad \forall n \in \mathbb{N}: \bigl(n \geq K \rightarrow \sqrt[n]{|a_n|} \leq q\bigr)$
\\[0.2cm]
gilt.  Dann konvergiert die Reihe $\Reihe{a_i}$.


\noindent
\textbf{Beispiel}: Wir zeigen mit dem Wurzel-Kriterium, dass die Reihe
$\ds\Reihe{\frac{1}{i!}}$konvergiert.  Wir setzen
$K = 4$ und $q = \frac{1}{2}$.  Zun\"achst k\"onnen Sie mit vollst\"andiger Induktion
leicht zeigen, dass f\"ur alle nat\"urlichen Zahlen $n\geq 4$ die Ungleichung
$n! \geq 2^n$ gilt.  Damit haben wir f\"ur $n\geq 4$:
\\[0.2cm]
\hspace*{1.3cm}
$\ds \sqrt[n]{\frac{1}{n!}} \leq \frac{1}{2}     \quad \Leftrightarrow \quad
   \frac{1}{n!} \leq \left(\frac{1}{2}\right)^n  \quad \Leftrightarrow \quad
   n! \geq 2^n. 
$ \eox


\exercise
Im Folgenden sei $\folge{a_n}$ eine monoton fallende Folge nicht-negativer Zahlen, es gelte also
\\
\hspace*{1.3cm}
$a_n \geq a_{n+1} \geq 0$ \quad f\"ur alle $n \in \mathbb{N}$.
\begin{enumerate}[(a)]
\item Nehmen Sie an, dass die Reihe
      \\[0.2cm]
      \hspace*{1.3cm}
      $\sum\limits_{n=0}^\infty 2^n \cdot a_{2^n}$ 
      \\[0.2cm]
      konvergiert.  Zeigen Sie, dass dann auch die Reihe
      \\[0.2cm]
      \hspace*{1.3cm}
      $\ds\sum\limits_{n=1}^\infty a_n$
      \\[0.2cm]
      konvergiert.

      \textbf{Hinweis}:  Zeigen Sie zun\"achst, dass
      \\[0.2cm]
      \hspace*{1.3cm}
      $\ds\sum\limits_{i=1}^{2^n} a_i \leq \sum\limits_{i=0}^n 2^i \cdot a_{2^i}$       
      \\[0.2cm]
      gilt.  Der Nachweis dieser Ungleichung funktioniert \"ahnlich wie der Nachweis der Divergenz der
      harmonischen Reihe.
\item Nehmen Sie an, dass die Reihe
      \\[0.2cm]
      \hspace*{1.3cm}
      $\ds\sum\limits_{n=1}^\infty a_n$
      \\[0.2cm]
      konvergiert.  Zeigen Sie, dass dann auch die Reihe
      \\[0.2cm]
      \hspace*{1.3cm}
      $\ds\sum\limits_{n=0}^\infty 2^n \cdot a_{2^n}$ 
      \\[0.2cm]
      konvergiert.

      \textbf{Hinweis}:  Zeigen Sie die Ungleichung
      \\[0.2cm]
      \hspace*{1.3cm}
      $\ds\sum\limits_{i=0}^n 2^i \cdot a_{2^i} \leq 2 \cdot \sum\limits_{i=1}^{2^n} a_i $.
      
      \textbf{Bemerkung}:  Die in (a) und (b) bewiesenen Aussagen k\"onnen wir zu dem
      \href{http://de.wikipedia.org/wiki/Cauchysches_Verdichtungskriterium}{\emph{Kondensations-Kriterium von Cauchy}} 
      zusammenfassen:  Die Reihe 
      \\[0.2cm]
      \hspace*{1.3cm}
      $\ds\sum\limits_{n=1}^\infty a_n$ \quad konvergiert genau dann, wenn die Reihe \quad
      $\ds\sum\limits_{n=0}^\infty 2^n \cdot a_{2^n}$ \quad konvergiert.
\item Untersuchen Sie die Konvergenz der Reihe
      \\[0.2cm]
      \hspace*{1.3cm}
      $\ds\sum\limits_{n=2}^{\infty} \frac{1}{n \cdot \log_2(n)}$.
\item Untersuchen Sie die Konvergenz der Reihe
      \\[0.2cm]
      \hspace*{1.3cm}
      $\ds\sum\limits_{n=2}^{\infty} \frac{1}{n \cdot \bigl(\log_2(n)\bigr)^2}$. \eox
\end{enumerate}
\pagebreak

\subsection{Absolute Konvergenz}
\begin{Definition}[Absolute Konvergenz]
Eine Reihe
\\[0.2cm]
\hspace*{1.3cm}
 $\ds\Reihe{a_i}$ 
\\[0.2cm]
hei\ss{}t \colorbox{orange}{\emph{absolut konvergent}} genau dann, wenn die Reihe der Absolutbetr\"age
\\[0.2cm]
\hspace*{1.3cm}
 $\ds\Reihe{|a_i|}$ 
\\[0.2cm]
konvergiert.  Falls die Reihe
 $\Reihe{a_i}$
konvergiert, aber die Reihe der Absolutbetr\"age
$\Reihe{|a_i|}$ 
divergent ist,  dann sagen wir, dass die Reihe
 $\Reihe{a_i}$ 
\colorbox{orange}{\emph{bedingt konvergent}} ist.  \eod
\end{Definition}

\examples
\begin{enumerate}[(a)]
\item Die Reihe
      \\[0.2cm]
      \hspace*{1.3cm}
      $\ds\Reihe{\frac{(-1)^{i+1}}{i}}$
      \\[0.2cm]
      ist nach dem Leibniz-Kriterium bedingt konvergent, aber nicht absolut konvergent, denn die Reihe der Absolutbetr\"age ist
      die harmonische Reihe und die harmonische Reihe ist divergent.
\item Die Reihe
      \\[0.2cm]
      \hspace*{1.3cm}
      $\ds\Reihe{\frac{(-1)^i}{i^2}}$
      \\[0.2cm]
      ist absolut konvergent, denn der Grenzwert $\ds\sum\limits_{n=1}^\infty \frac{1}{n^2}$ existiert. \eox
\end{enumerate}

\begin{Satz}
  Falls die Reihe $\Reihe{a_i}$ absolut konvergent ist, dann ist sie auch konvergent.
\end{Satz}

\proof
Nach Voraussetzung wissen wir, dass der Grenzwert
\\[0.2cm]
\hspace*{1.3cm}
$\ds s := \lim\limits_{n\rightarrow\infty} \sum\limits_{i=1}^{n} |a_i|$
\\[0.2cm]
existiert.  An der Konvergenz \"andert sich auch nichts, wenn wir die Reihe mit $2$ multiplizieren.
Weiter gilt f\"ur jedes $i \in \mathbb{N}$ die Ungleichung
\\[0.2cm]
\hspace*{1.3cm}
$0 \leq a_i + |a_i| \leq 2 \cdot |a_i|$,
\\[0.2cm]
denn wenn $a_i$ positiv ist, gilt $a_i + |a_i| = 2 \cdot |a_i|$ und wenn $a_i$ negativ ist, dann
gilt $a_i + |a_i| = 0$.  Also ist die Reihe 
\\[0.2cm]
\hspace*{1.3cm}
$\ds \sum\limits_{i=1}^{n} 2 \cdot |a_i|$ \quad eine konvergente Majorante der Reihe \quad
$\ds \sum\limits_{i=1}^{n} \bigl(a_i + |a_i|\bigr)$
\\[0.2cm]
und folglich konvergiert die Reihe $\sum\limits_{i=1}^{n} \bigl(a_i + |a_i|\bigr)$ nach dem Majoranten-Kriterium.
Nun gilt aber
\\[0.2cm]
\hspace*{1.3cm}
$\ds \lim\limits_{n\rightarrow\infty} \sum\limits_{i=1}^{n} a_i =
     \lim\limits_{n\rightarrow\infty}\sum\limits_{i=1}^{n} \bigl(a_i + |a_i|\bigr) 
   - \lim\limits_{n\rightarrow\infty} \sum\limits_{i=1}^{n} |a_i|$
\\[0.2cm]
und da die Grenzwerte auf der rechten Seite dieser Gleichung existieren, existiert auch der
Grenzwert auf der linken Seite dieser Gleichung.  \qed

\exercise
Zeigen Sie, dass der Grenzwert
\\[0.2cm]
\hspace*{1.3cm}
$\ds\sum\limits_{n=1}^{\infty} \frac{\;1 + 2 \cdot \sin(3 \cdot n + 4)\;}{n^2}$
\\[0.2cm]
existiert.  \qed


\section{Potenz-Reihen}
Es bezeichne $x$ eine Variable und $\folge{a_n}$ sei eine Folge von Zahlen.  Dann
bezeichnen wir den Ausdruck
\\[0.2cm]
\hspace*{1.3cm}
$\ds\sum\limits_{n=0}^\infty a_n \cdot  x^n$      
\\[0.2cm]
als \emph{formale Potenz-Reihe}.  Wichtig ist hier, dass $x$ keine feste Zahl ist, sondern
eine Variable, f\"ur die wir sp\"ater reelle (oder auch komplexe) Zahlen einsetzen.  Wenn wir
in einer Potenz-Reihe f\"ur $x$ eine feste Zahl einsetzen, wird aus der Potenz-Reihe eine
gew\"ohnliche Reihe.  Der Begriff der Potenz-Reihen kann als eine Verallgemeinerung des
Begriffs des   Polynoms aufgefasst werden. 


\examples
\begin{enumerate}
\item $\ds\sum\limits_{n=0}^\infty \frac{x^n}{n!}$ ist eine formale Potenz-Reihe.  
      Wir haben oben mit Hilfe des Quotienten-Kriteriums gezeigt, dass diese Reihe f\"ur
      beliebige reelle und komplexe Zahlen konvergiert.
\item $\ds\sum\limits_{n=1}^\infty \frac{x^n}{n}$ ist eine formale Potenz-Reihe.  
      Setzen wir f\"ur $x$ den Wert $1$ ein, so erhalten wir die divergente harmonische
      Reihe.  F\"ur $x=-1$ erhalten wir eine alternierende Reihe, die nach dem Leibniz-Kriterium
      konvergent ist. \eox
\end{enumerate}

In der Theorie der Potenz-Reihen ist die Frage entscheidend, welche Zahlen wir f\"ur die
Variable $x$ einsetzen k\"onnen, so dass die Reihe konvergiert.  Diese Frage wird durch die
folgenden S\"atze beantwortet.

\begin{Satz}\label{satz:konvergenz-radius}
Wenn die Potenz-Reihe 
\\[0.2cm]
\hspace*{1.3cm}
$\sum\limits_{n=0}^\infty a_n\cdot x^n$ 
\\[0.2cm]
f\"ur einen Wert $u \in \mathbb{C}$ konvergiert, dann konvergiert die Reihe auch f\"ur alle $v
\in \mathbb{C}$, f\"ur die $|v| < |u|$ ist.
\end{Satz}

\noindent 
\textbf{Beweis}:  Da die Reihe
\\[0.2cm]
\hspace*{1.3cm}
 $\ds\sum\limits_{n=0}^\infty a_n\cdot u^n$ 
\\[0.2cm]
konvergiert, folgt aus dem
Korollar zum Cauchy'schen Kon\-vergenz-Kriterium, dass die Folge \\ $\folge{a_n\cdot u^n}$ eine
Null-Folge ist.  Also gibt es eine Zahl $K$, so dass f\"ur alle $n \geq K$ die Ungleichung 
\\[0.2cm]
\hspace*{1.3cm}      
$\left|a_n\cdot u^n\right| \leq 1$
\\[0.2cm]
gilt.  Wir definieren 
\\[0.2cm]
\hspace*{1.3cm}
$\ds q := \left|\frac{v}{u}\right|$. 
\\[0.2cm]
 Aus $|v| < |u|$ folgt $q < 1$.
Dann haben wir f\"ur alle $n \geq K$ die folgende Absch\"atzung:
      \\[0.2cm]
      \hspace*{1.3cm}      
      $\ds\left|a_n\cdot v^n\right| = 
      \left|a_n\cdot u^n\right| \cdot  \left|\frac{v^n}{u^n}\right| =
      \left|a_n\cdot u^n\right| \cdot q^n \leq 1 \cdot q^n = q^n$.
      \\[0.2cm]
Diese Absch\"atzung zeigt, dass die geometrische Reihe eine konvergente Majorante der Reihe
\\[0.2cm]
\hspace*{1.3cm}
$\ds\sum\limits_{n=1}^\infty a_n\cdot v^n$ 
\\[0.2cm]
ist.  Damit folgt die Konvergenz dieser Reihe aus dem
Majoranten-Kriterium. \qed 

\begin{Satz}
Wenn die Potenz-Reihe
\\[0.2cm]
\hspace*{1.3cm}
 $\ds\sum\limits_{n=0}^\infty a_n\cdot x^n$ 
\\[0.2cm]
f\"ur einen Wert $u \in \mathbb{C}$
divergiert, dann divergiert die Reihe auch f\"ur alle $v \in \mathbb{C}$, f\"ur die
$|u| < |v|$ ist.
\end{Satz}

\noindent
\textbf{Beweis}:  W\"urde die Reihe
\\[0.2cm]
\hspace*{1.3cm}
 $\ds\sum\limits_{n=0}^\infty a_n\cdot v^n$ 
\\[0.2cm]
konvergieren,
dann m\"usste nach Satz \ref{satz:konvergenz-radius} auch die Reihe 
\\[0.2cm]
\hspace*{1.3cm}
$\ds\sum\limits_{n=0}^\infty a_n\cdot u^n$
\\[0.2cm]
konvergieren, was im Widerspruch zur Voraussetzung steht. \qed

\noindent
Die letzten beiden S\"atze erm\"oglichen es nun, den Begriff des \colorbox{orange}{\emph{Konvergenz-Radius}} zu
definieren.  Es sei 
\\[0.2cm]
\hspace*{1.3cm}
$\ds\sum\limits_{n=0}^\infty a_n\cdot x^n$ 
\\[0.2cm]
eine formale Potenz-Reihe.
Wenn diese Reihe f\"ur alle $x\in\mathbb{C}$ konvergiert, dann sagen wir, dass der
Konvergenz-Radius den Wert $\infty$ hat.  Andernfalls definieren wir den Konvergenz-Radius
als
\\[0.2cm]
\hspace*{1.3cm}      
$R := \sup \biggl\{ |u| \;\biggm|\; u \in \mathbb{C} \;\wedge\;
                   \ds\sum\limits_{n=0}^\infty a_n\cdot u^n \;\mathrm{konvergiert} 
         \biggr\}
$.
\\[0.2cm]
Aus den letzten beiden S\"atzen folgt dann:
\begin{enumerate}
\item $\forall z \in \mathbb{C}:\Bigl( |z| < R \rightarrow \ds\sum\limits_{n=0}^\infty a_n\cdot z^n\Bigr)$ konvergiert. 
\item $\forall z \in \mathbb{C}:\Bigl( |z| > R \rightarrow \ds\sum\limits_{n=0}^\infty a_n\cdot z^n\Bigr)$ divergiert. 
\end{enumerate}
In der Gau\ss{}'schen Zahlen-Ebene ist die Menge 
$\bigl\{ z \in \mathbb{C} \;\big|\; |z| < R \bigr\}$ das Innere eines Kreises mit dem
Radius $R$ um den Nullpunkt.
Der folgende Satz gibt uns eine effektive M\"oglichkeit, den Konvergenz-Radius zu berechnen.

\begin{Satz}
Es sei 
\\[0.2cm]
\hspace*{1.3cm}
$\ds\sum\limits_{n=0}^\infty a_n\cdot z^n$ 
\\[0.2cm]
eine formale Potenz-Reihe und die Folge
\\[0.2cm]
\hspace*{1.3cm}
$\ds\Folge{\frac{|a_n|}{|a_{n+1}|}}$
\\[0.2cm]
 konvergiere.   Dann ist der Konvergenz-Radius $R$ durch
folgende Formel gegeben:
      \\[0.2cm]
      \hspace*{1.3cm}      
      $\ds R = \lim\limits_{n\rightarrow\infty} \left|\frac{a_n}{\;a_{n+1}\;}\right|$.  
\end{Satz}

\noindent
\textbf{Beweis}:
Es sei $u \in \mathbb{C}$ mit $|u| < R$.  In diesem Fall m\"ussen wir zeigen, dass die Reihe
\\[0.2cm]
\hspace*{1.3cm}
$\ds\sum\limits_{n=0}^\infty a_n\cdot u^n$
\\[0.2cm]
 konvergiert.  Wir werden diesen Nachweis mit Hilfe des
Quotienten-Kriteriums erbringen.  Wir setzen
\\[0.2cm]
\hspace*{1.3cm}
 $\ds q := \frac{|u|}{\frac{1}{2} \cdot (R + |u|)}$ 
\\[0.2cm] 
und zeigen, dass $q < 1$ ist:
\\[0.2cm]
\hspace*{1.3cm}
$
\begin{array}[t]{cl}
                & q < 1 \\[0.2cm]
\Leftrightarrow &\ds \frac{|u|}{\frac{1}{2} \cdot (R + |u|)} < 1  \\[0.8cm] 
\Leftrightarrow & 2 \cdot |u| < R + |u|                        \\[0.2cm]  
\Leftrightarrow & |u| < R                         
\end{array}
$
\\[0.2cm]
und die letzte Ungleichung ist nach Wahl von $u$ wahr.  Da wir nur \"Aquivalenzumformungen benutzt
haben, ist damit auch die Formel $q < 1$ wahr.  Wir zeigen weiter,  dass f\"ur
alle hinreichend gro\ss{}en $n$ die Ungleichung 
      \\[0.2cm]
      \hspace*{1.3cm}      
      $\ds\frac{|a_{n+1}\cdot u^{n+1}|}{|a_{n}\cdot u^{n}|} \leq q$
      \\[0.2cm]
erf\"ullt ist.  Um diesen Beweis zu f\"uhren, m\"ussen wir etwas ausholen.  Zun\"achst folgt aus 
      \\[0.2cm]
      \hspace*{1.3cm}      
       $\ds R = \lim\limits_{n\rightarrow\infty} \left|\frac{a_n}{\;a_{n+1}\;}\right|$,
      \\[0.2cm]
dass es f\"ur beliebige $\varepsilon >0$ eine Zahl $K$ gibt, so dass f\"ur alle $n \geq K$
die Ungleichung
\\[0.2cm]
\hspace*{1.3cm}      
$\ds\left| \Bigl|\frac{a_{n}}{a_{n+1}}\Bigr| - R \right| < \varepsilon$
\\[0.2cm]
gilt.  Wir setzen $\varepsilon := \frac{1}{2}\cdot(R - |u|)$. Wir zeigen, dass dann f\"ur alle 
$n \geq K$ die Ungleichung
\\
 $\ds \left|\frac{a_{n}}{a_{n+1}}\right| > \frac{1}{2}(R + |u|)$ gilt:
\\[0.2cm]
\hspace*{1.3cm}
$
\begin{array}{cl}
 &\ds \left|\;\bigl|\frac{a_{n}}{a_{n+1}}\bigr| + \bigl(R - \bigl|\frac{a_{n}}{a_{n+1}}\bigr|\bigr)\;\right| = |R| = R  \\[0.5cm]
\Rightarrow     & \ds\bigl|\frac{a_{n}}{a_{n+1}}\bigr| + \bigl|\; \bigl(R - \bigl|\frac{a_{n}}{a_{n+1}}\bigr|\bigr)\;\bigr| \geq R  \\[0.5cm]
\Rightarrow     & \ds\bigl|\frac{a_{n}}{a_{n+1}}\bigr| + \varepsilon > R  \\[0.5cm]
\Rightarrow     & \ds\bigl|\frac{a_{n}}{a_{n+1}}\bigr| + \frac{1}{2}(R - |u|) > R  \\[0.5cm]
\Rightarrow     & \ds\bigl|\frac{a_{n}}{a_{n+1}}\bigr|   > R - \frac{1}{2}(R - |u|)  \\[0.5cm]
\Rightarrow     & \ds\bigl|\frac{a_{n}}{a_{n+1}}\bigr|   > \frac{1}{2}(R + |u|).  \\[0.5cm]
\end{array}
$
\\[0.2cm]
Jetzt k\"onnen wir zeigen, dass die Reihe
\\[0.2cm]
\hspace*{1.3cm}
 $\ds\sum\limits_{n=0}^\infty a_n\cdot u^n$ \\[0.2cm]
das
Quotienten-Kriterium erf\"ullt, denn f\"ur alle $n \geq K$ gilt:
\\[0.2cm]
\hspace*{1.3cm}
$\ds \left|\frac{a_{n+1}\cdot u^{n+1}}{a_n\cdot u^n}\right| = 
   \left|\frac{a_{n+1}}{a_n}\right| \cdot  |u| = 
   \frac{|u|}{\left|\frac{a_{n}}{a_{n+1}}\right|} < 
   \frac{|u|}{\frac{1}{2}\cdot(R + |u|)} = q
$
\\[0.2cm]
Um den Beweis abzuschlie\ss{}en m\"ussen wir noch zeigen, die Reihe
\\[0.2cm]
\hspace*{1.3cm}
 $\ds\sum\limits_{n=0}^\infty a_n\cdot u^n$
\\[0.2cm]
divergiert wenn  $R < |u|$ ist.   Dies folgt aus der Tatsache, dass die Folge
$\folge{a_n\cdot u^n}$ f\"ur $|u|>R$ keine Null-Folge ist.  Die Details bleiben dem Leser
\"uberlassen. 
\qed 

\exercise
Vervollst�ndigen Sie den Beweis des letzten Satzes. \eox

\remark
Der obige Satz bleibt auch richtig, wenn
      \\[0.2cm]
      \hspace*{1.3cm}      
      $\ds\lim\limits_{n\rightarrow\infty} \left|\frac{a_n}{\;a_{n+1}\;}\right| = \infty$
      \\[0.2cm]
ist, denn dann ist die Potenz-Reihe $\ds\sum\limits_{n=0}^\infty a_n\cdot u^n$ f\"ur alle $u\in\mathbb{C}$
konvergent.

\remark
Die Potenz-Reihe $\ds\sum_{n=1}^\infty \frac{x^n}{n}$ hat den Konvergenz-Radius $R=1$, denn
es gilt 
\\[0.2cm]
\hspace*{1.3cm}
$\ds
      \lim\limits_{n\rightarrow\infty} \left|\frac{\frac{1}{n}}{\;\frac{1}{n+1}\;}\right| = 
       \lim\limits_{n\rightarrow\infty} \frac{n+1}{n} = 
       1 + \lim\limits_{n\rightarrow\infty}  \frac{1}{n} = 1 + 0 = 1.
$  \eox



\begin{Satz}[\href{https://de.wikipedia.org/wiki/Jacques_Hadamard}{Jacques Hadamard(1865-1963)}]
Es sei $\sum_{n=0}^\infty a_n\cdot x^n$ eine Potenz-Reihe und die Folge
$\Folge{\sqrt[n]{|a_n|}}$ konvergiere.   Dann gilt 
      \\[0.2cm]
      \hspace*{1.3cm}      
      $\ds\frac{1}{R} = \lim\limits_{n\rightarrow\infty} \sqrt[n]{|a_n|}$.  
\end{Satz}

\remark
Setzen wir $\frac{1}{\infty} = 0$, so  bleibt die Formel 
      \\[0.2cm]
      \hspace*{1.3cm}      
      $\frac{1}{R} = \lim\limits_{n\rightarrow\infty} \sqrt[n]{|a_n|}$.  
      \\[0.2cm]
auch in dem Fall
$\lim\limits_{n\rightarrow\infty} \sqrt[n]{|a_n|} = 0$ richtig, denn dann gilt $R = \infty$. \eox


\example
Die Potenz-Reihe
\\[0.2cm]
\hspace*{1.3cm}
 $\ds\sum\limits_{n=1}^\infty \frac{x^n}{n^n}$ 
\\[0.2cm]
hat den Konvergenz-Radius $R=\infty$, denn
es gilt 
\\[0.2cm]
\hspace*{1.3cm}
$\ds \lim\limits_{n\rightarrow\infty} \sqrt[n]{\frac{1}{n^n}} =  
   \lim\limits_{n\rightarrow\infty} \frac{1}{n} = 0.
$ \eox

\exercise
\begin{enumerate}[(a)]
\item Der \href{http://de.wikipedia.org/wiki/Binomialkoeffizient}{\emph{Binomial-Koeffizient}} $n \choose k$
      ist durch die Festlegung
      \\[0.2cm]
      \hspace*{1.3cm}
      $\ds {n \choose k} = \frac{n!}{k! \cdot (n-k)!}$
      \\[0.2cm]
      definiert.  Beweisen Sie die Gleichung \quad
      $\ds {n+1 \choose k+1} = {n \choose k} + {n \choose k+1}$.
\item Zeigen Sie durch vollst\"andige Induktion, dass f\"ur alle nat\"urlichen Zahlen $n$ die Formel
      \\[0.2cm]
      \hspace*{1.3cm}
      $\ds (a + b)^n =\sum\limits_{k=0}^n {n \choose k} \cdot a^k \cdot b^{n-k}$
      \\[0.2cm]
      gilt.  Diese Gleichung wird als der 
      \href{http://de.wikipedia.org/wiki/Binomischer_Lehrsatz}{\emph{binomischer Lehrsatz}}
      bezeichnet.
\item \"Uberlegen Sie sich, wie Sie f\"ur zwei Potenz-Reihen 
      \\[0.2cm]
      \hspace*{1.3cm}
      $\ds f(x) := \sum\limits_{n=0}^\infty a_n \cdot x^n$ \quad und \quad
      $\ds g(x) := \sum\limits_{n=0}^\infty b_n \cdot x^n$ 
      \\[0.2cm]
      das Produkt $f(x) \cdot g(x)$ als Potenz-Reihe darstellen k\"onnen.  Wir suchen also eine
      Potenz-Reihe $\sum\limits_{n=0}^\infty c_n \cdot x^n$, so dass die Formel
      \\[0.2cm]
      \hspace*{1.3cm}
      $\ds f(x) \cdot g(x) = \sum\limits_{n=0}^\infty c_n \cdot x^n$ \quad g\"ultig ist.
\item F\"ur alle $x \in \mathbb{C}$ ist die Exponentialfunktion als
      \\[0.2cm]
      \hspace*{1.3cm}
      $\ds \exp(x) := \sum\limits_{n=0}^\infty \frac{x^n}{n!}$
      \\[0.2cm]
      definiert.  Zeigen Sie, dass $\exp(x) \cdot \exp(x) = \exp(2 \cdot x)$ gilt. \eox
\item Zeigen Sie, dass f�r alle $x,y \in \mathbb{C}$ die Gleichung
      \\[0.2cm]
      \hspace*{1.3cm}
      $\exp(x) \cdot \exp(y) = \exp(x + y)$
      \\[0.2cm]
      gilt.
\end{enumerate}

\section{Berechnung der Fibonacci-Zahlen}
Die Folge $\folge{a_n}$ der
\href{http://de.wikipedia.org/wiki/Fibonacci-Folge}{\emph{Fibonacci-Zahlen}} wird induktiv durch die
Festlegung  
\\[0.2cm]
\hspace*{1.3cm}
$a_0 := 0$, \quad $a_1 := 1$ \quad und \quad $a_{n+2} := a_{n+1} + a_n$ f\"ur alle $n \in \mathbb{N}_0$
\\[0.2cm]
definiert.  Wir hatten bereits im ersten Semester eine explizite Formel f\"ur $a_n$ hergeleitet.  In
diesem Abschnitt zeigen wir einen anderen Weg auf, mit dem dieselbe Formel hergeleitet werden kann.
Dazu definieren wir zun\"achst die der Folge $\folge{a_n}$ zugeordnete 
\href{http://de.wikipedia.org/wiki/Erzeugende_Funktion}{\emph{erzeugende Funktion}} $f(x)$ als die Potenz-Reihe
\\[0.2cm]
\hspace*{1.3cm}
$\ds f(x) := \sum\limits_{n=0}^\infty a_n \cdot x^n = 0 \cdot x^0 + 1 \cdot x + 1 \cdot x^2 + 2 \cdot x^3 + 3 \cdot x^4 + 5 \cdot x^5 + \cdots$.
\\[0.2cm]
Die Gleichung $a_{n+2} = a_{n+1} + a_n$ motiviert uns dazu, den Ausdruck
 $x \cdot f(x) + x^2 \cdot f(x)$ zu berechnen.
Zun\"achst gilt
\\[0.2cm]
\hspace*{1.3cm}
$\ds x \cdot f(x) = \sum\limits_{n=0}^\infty a_n \cdot x^{n+1} = \sum\limits_{n=1}^\infty a_{n-1}
\cdot x^n$.  \hspace*{\fill} (1)
\\[0.2cm]
Multiplizieren wir diese Gleichung ein weiteres Mal mit $x$, so erhalten wir
\\[0.2cm]
\hspace*{1.3cm}
$\ds x^2 \cdot f(x) = \sum\limits_{n=1}^\infty a_{n-1} \cdot x^{n+1} = \sum\limits_{n=2}^\infty
a_{n-2} \cdot x^n$.  \hspace*{\fill} (2)
\\[0.2cm]
Addieren wir die Gleichungen (1) und (2), so erhalten wir
\\[0.2cm]
\hspace*{1.3cm}
$(x + x^2) \cdot f(x) = a_0 \cdot x^1 + \sum\limits_{n=2}^\infty \bigl(a_{n-1} + a_{n-2}\bigr) \cdot x^n$.
\\[0.2cm]
An dieser Stelle k\"onnen Sie sehen, warum wir $f(x)$ mit $x + x^2$ multipliziert haben, denn wenn wir
nun ber\"ucksichtigen, dass $a_{n-1} + a_{n-2} = a_n$ gilt, so k\"onnen wir die obige Gleichungen zu
\\[0.2cm]
\hspace*{1.3cm}
$(x + x^2) \cdot f(x) = \sum\limits_{n=0}^\infty a_{n} \cdot x^n - a_1 \cdot x$ 
\\[0.2cm]
vereinfachen.  Die Summe, die hier auftritt, ist aber gerade die erzeugende Funktion $f(x)$.
Da $a_1 = 1$ ist, haben wir die Gleichung
\\[0.2cm]
\hspace*{1.3cm}
$(x + x^2) \cdot f(x) = f(x) - x$ 
\\[0.2cm]
gefunden, die wir zu
\\[0.2cm]
\hspace*{1.3cm}
$x = (1 - x - x^2) \cdot f(x)$ 
\\[0.2cm]
umschreiben.  Teilen wir hier noch durch $(1-x-x^2)$, so haben wir f\"ur die erzeugende Funktion
$f(x)$ die Gleichung
\\[0.2cm]
\hspace*{1.3cm}
$\ds f(x) = \frac{x}{1 - x - x^2}$
\\[0.2cm]
gefunden.  Der Trick besteht nun darin, eine 
\href{http://de.wikipedia.org/wiki/Partialbruchzerlegung}{\emph{Partialbruch-Zerlegung}} 
der rechten Seite zu finden.  Indem Sie die Nullstellen des Polynoms $1 - x - x^2$ bestimmen, k\"onnen
sie den obigen Bruch nach einer kurzen Rechnung in der Form
\\[0.2cm]
\hspace*{1.3cm}
$\ds \frac{x}{1 - x - x^2} = \frac{1}{\sqrt{5}} \cdot \left(\frac{1}{1 - x \cdot \varphi} - \frac{1}{1 - x \cdot \overline{\varphi}}\right)$
\\[0.2cm]
\hspace*{1.3cm}
mit $\varphi = \frac{1}{2} \cdot (1 + \sqrt{5})$ \quad und \quad  $\overline{\varphi} = 1
-\varphi = \frac{1}{2} \cdot (1 - \sqrt{5})$
\\[0.2cm]
darstellen.  Ersetzen wir die Br\"uche auf der rechten Seite dieser Gleichung durch geometrische
Reihen, so erhalten wir die Gleichung
\\[0.2cm]
\hspace*{1.3cm}
$\ds f(x) = \frac{1}{\sqrt{5}} \cdot \left(\sum\limits_{n=0}^\infty (x \cdot \varphi)^n - \sum\limits_{n=0}^\infty (x \cdot \overline{\varphi})^n \right)$.
\\[0.2cm]
Setzen wir nun f\"ur $f(x)$ die Reihe $\sum\limits_{n=0}^\infty a_n \cdot x^n$ ein und fassen die die
beiden Summen auf der rechten Seite zusammen, so erhalten wir die Gleichung
\\[0.2cm]
\hspace*{1.3cm}
$\ds \sum\limits_{n=0}^\infty a_n \cdot x^n = \sum\limits_{n=0}^\infty \frac{1}{\sqrt{5}} \cdot \bigl(\varphi^n - \overline{\varphi}^n\bigr) \cdot x^n$.
\\[0.2cm]
Da diese Gleichungen f\"ur alle $x$ gelten soll, f\"ur die diese Reihen konvergieren, m\"ussen die
Koeffizienten der Potenzen von  $x^n$ auf beiden Seiten dieser Gleichung \"ubereinstimmen.  Damit
haben wir f\"ur die Fibonacci-Zahlen die explizite Formel
\\[0.2cm]
\hspace*{1.3cm}
$\ds a_n = \frac{1}{\sqrt{5}} \cdot \bigl(\varphi^n - \overline{\varphi}^n\bigr)$ 
\quad mit $\varphi= \frac{1}{2} \cdot (1 + \sqrt{5})$ 
\quad und \quad  $\overline{\varphi} = 1 -\varphi = \frac{1}{2} \cdot (1 - \sqrt{5})$
\\[0.2cm]
gefunden.

\exercise
\begin{enumerate}[(a)]
\item L\"osen Sie die Rekurrenz-Gleichung
      \\[0.2cm]
      \hspace*{1.3cm}
      $a_0 = 0$, $a_1 = 1$, $a_{n+2} = 3 \cdot a_{n+1} - 2 \cdot a_n$
      \\[0.2cm]
      mit dem in diesem Abschnitt vorgestellten Verfahren.
\item L\"osen Sie die Rekurrenz-Gleichung
      \\[0.2cm]
      \hspace*{1.3cm}
      $a_0 = 0$, $a_1 = 1$, $a_{n+2} = 2 \cdot a_{n+1} - 1 \cdot a_n$
      \\[0.2cm]
      mit dem in diesem Abschnitt vorgestellten Verfahren.

      \textbf{Hinweis}:  Stellen Sie das Produkt
      \\[0.2cm]
      \hspace*{1.3cm}
      $\biggl(\sum\limits_{n=0}^\infty x^n\biggr) \cdot \biggl(\sum\limits_{n=0}^\infty x^n\biggr)$
      \\[0.2cm]
      als Potenz-Reihe dar.  \eox
\end{enumerate}

%%% Local Variables: 
%%% mode: latex
%%% TeX-master: "analysis"
%%% End: 
