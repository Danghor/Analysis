\chapter{Die reellen Zahlen}
Bevor wir mit der eigentlichen Analysis beginnen m\"ussen wir kl\"aren, was genau reelle Zahlen
\"uberhaupt sind.  Anschaulich werden reelle Zahlen zur Angabe von L\"angen ben\"otigt, denn in der Geometrie
reicht es nicht, mit den rationalen Zahlen zu arbeiten.  Das liegt daran, dass die Diagonale eines
Quadrats der Seitenl\"ange 1 nach dem 
\href{http://de.wikipedia.org/wiki/Satz_des_Pythagoras}{Satz des Pythagoras} die L\"ange $\sqrt{2}$ hat.
Wir hatten im ersten Semester aber gesehen, dass es keine rationale Zahl $r$ gibt,
so dass $r^2 = 2$ ist, denn $\sqrt{2}$ ist irrational.  Folglich reichen die rationalen Zahlen f\"ur die Geometrie
nicht aus.  Es gibt mehrere Wege, die Menge $\mathbb{R}$ der reellen Zahlen so zu konstruieren, so dass die Gleichung 
\\[0.2cm]
\hspace*{1.3cm}
$r^2 = 2$
\\[0.2cm]
in $\mathbb{R}$ eine L\"osung hat.  Bevor wir mit dieser Konstruktion beginnen, wollen wir die Menge
der reellen Zahlen axiomatisch charakterisieren.  Dazu definieren wir den Begriff des
\\[0.2cm]
\hspace*{1.3cm}
\colorbox{orange}{\emph{vollst\"andig geordneten Kr\"opers}.}
\\[0.2cm]
Hierbei handelt es sich um ein System von Axiomen, aus dem
sich \underline{alle} weiteren Eigenschaften der reellen Zahlen ableiten lassen.  Es l\"asst sich n\"amlich zeigen,
dass diese Axiomatisierung in dem folgenden Sinne vollst\"andig ist:
Ist $\mathbb{K}$ ein vollst\"andig angeordneter K\"orper, so ist $\mathbb{K}$ \colorbox{orange}{\emph{isomorph}} zu
$\mathbb{R}$: Im Klartext hei\ss{}t dies, dass es eine Abbildung
\\[0.2cm]
\hspace*{1.3cm}
$\varphi:\mathbb{K} \rightarrow \mathbb{R}$
\\[0.2cm]
gibt, die jedem Element $\alpha \in \mathbb{K}$ genau ein Element $\varphi(\alpha) \in \mathbb{R}$
zuordnet und die mit der Addition, der Multiplikation und der Ordnungs-Relation vertr\"aglich ist, f\"ur alle $x,y \in
\mathbb{K}$ gilt also 
\begin{enumerate}
\item $\varphi(x + y) = \varphi(x) + \varphi(y)$,
\item $\varphi(x \cdot y) = \varphi(x) \cdot \varphi(y)$,
\item $\varphi(x) < \varphi(y) \;\leftrightarrow\; x < y$.
\end{enumerate}
Die rein axiomatische Charakterisierung der
reellen Zahlen ist philosophisch unbefriedigend, denn dabei bleiben zwei Fragen offen:
\begin{enumerate}
\item Gibt es \"uberhaupt eine Struktur $\mathbb{K}$, die den Axiomen eines vollst\"andig geordneten
      K\"orpers gen\"ugt?

      Solange wir nicht sicher sind, dass diese Frage mit ``Ja'' beantwortet wird, steht unsere
      gesamte Theorie auf wackeligen F\"u\ss{}en, denn es k\"onnte dann sein, dass es sich um die Theorie
      der leeren Menge handelt. 
\item Was genau sind reelle Zahlen?

      Es zeigt sich, dass wir die zweite Frage zuerst beantworten m\"ussen, bevor wir die Beantwortung
      der ersten Frage in Angriff nehmen k\"onnen. 
\end{enumerate}
Wir werden im zweiten Abschnitt zeigen, wie sich die reellen Zahlen aus den rationalen Zahlen mit
Hilfe von \href{https://de.wikipedia.org/wiki/Dedekindscher_Schnitt}{\emph{Dedekind-Schnitten}}
erzeugen lassen.  Da diese Konstruktion jedoch technisch aufwendig ist, werden wir diesen Abschnitt
im Rahmen der Vorlesung aus Zeitgr\"unden nicht in voller L\"ange diskutieren.  Die dort
pr\"asentierten Details sind zwar eine gute \"Ubung zur Mengenlehre,  werden aber f\"ur den weiteren
Verlauf der Vorlesung nicht mehr ben\"otigt.


\section{Axiomatische Charakterisierung der reellen Zahlen}
Wir erinnern zun\"achst an die im ersten Semester gegebene Definition eines K\"orpers. 
\begin{Definition}[K\"orper]
Eine Struktur $\mathcal{K} = \langle K, 0, 1, +, \cdot \rangle$ ist ein \colorbox{gold}{\emph{K\"orper}}, falls gilt:
\begin{enumerate}
\item $K$ ist eine Menge.
\item $0 \in K$ und $1 \in K$.
\item $+$ und $\cdot$ sind bin\"are Operatoren auf $K$, wir haben
      \\[0.2cm]
      \hspace*{1.3cm}
      $+: K \times K \rightarrow K$ \quad und \quad $\cdot: K \times K \rightarrow K$.      
\item $\langle K, 0, + \rangle$ ist eine kommutative Gruppe.
      
      Im Detail hei\ss{}t dies, dass die folgenden Axiome gelten:
      \begin{enumerate}
        \item $(x + y) + z = x + (y + z)$,
        \item $x + y = y + x$,
        \item $0 + x = x$,
        \item $\exists y \in K: y + x = 0$.

              Dasjenige $y \in K$, f\"ur welches $y + x = 0$ gilt, wird als das \emph{additive Inverse} von $x$ 
              bezeichnet und ist, wie wir in der Vorlesung zur linearen Algebra gesehen haben,
              eindeutig bestimmt.  Wir schreiben das additive Inverse als $-x$. 
        \end{enumerate}
\item $\langle K \backslash \{ 0 \}, 1, \cdot \rangle$ ist ebenfalls eine kommutative Gruppe,

      Im Detail sind also die folgenden Axiome erf\"ullt:
      \begin{enumerate}
      \item $(x \cdot y) \cdot z = x \cdot (y \cdot z)$,
      \item $x \cdot y = y \cdot x$,
      \item $1 \cdot x = x$,
      \item $x \not= 0 \rightarrow \exists y \in K: y \cdot x = 1$.

            Falls $x \not= 0$ ist, so ist das $y \in K$, f\"ur welches $y \cdot x = 1$ gilt, eindeutig
            bestimmt.  Es wird als das \emph{multiplikative Inverse} von $x$   
            bezeichnet.  Wir schreiben das multiplikative Inverse als $x^{-1}$. 
            
      \end{enumerate}
\item Es gilt das Distributiv-Gesetz: F\"ur alle $x, y,z \in K$ haben wir
      \\[0.2cm]
      \hspace*{1.3cm} 
      $x \cdot (y + z) = x \cdot y + x \cdot z$.  \edx
\end{enumerate}
\end{Definition}

\noindent
Weiter ben\"otigen wir die Definition einer linearen Ordnung, die wir ebenfalls wiederholen.

\begin{Definition}[Lineare Ordnung]
  Ein Paar $\langle M, \leq \rangle$ ist eine \colorbox{gold}{\emph{lineare Ordnung}}, falls gilt:
  \begin{enumerate}
  \item $M$ ist eine Menge und $\leq$ ist eine bin\"are Relation auf $M$, es gilt also
        \\[0.2cm]
        \hspace*{1.3cm}
        $\leq\; \subseteq M \times M$.
  \item Zus\"atzlich gelten die folgenden Axiome:
        \begin{enumerate}
        \item $x \leq x$,                 \hspace*{\fill} (Reflexivit\"at)
        \item $x \leq y \wedge y \leq x \rightarrow x = y$,  \hspace*{\fill} (Anti-Symmetrie)
        \item $x \leq y \wedge y \leq z \rightarrow x \leq z$,  \hspace*{\fill} (Transitivit\"at)
        \item $x \leq y \vee y \leq x$. \quad  $\diamond$   \hspace*{\fill} (Linearit\"at)
        \end{enumerate}
  \end{enumerate} 
\end{Definition} 

Die n\"achste Definition kombiniert die algebraischen Eigenschaften eines K\"orpers mit den
Anordnungs-Axiomen einer linearen Ordnung.

\begin{Definition}[Geordneter K\"orper]  \hspace*{\fill} \linebreak
  Eine Struktur $\mathcal{K} = \langle K, 0, 1, +, \cdot, \leq \rangle$ ist ein 
  \colorbox{gold}{\emph{geordneter K\"orper}}, falls 
  \begin{enumerate}
  \item $\langle K, 0, 1, +, \cdot, \rangle$ ein K\"orper und
  \item $\langle K, \leq \rangle$
  \end{enumerate}
  eine lineare Ordnung ist, die mit den arithmetischen Operationen $+$ und $\cdot$ 
  \emph{vertr\"aglich} ist.  Dazu m\"ussen die beiden folgenden Bedingungen erf\"ullt sein:
  \begin{enumerate}
  \item $\forall x,y,z \in K: \bigl(x \leq y \rightarrow x + z \leq y + z\bigr)$,  \hspace*{\fill} (A1)
  \item $\forall x,y \in K:\bigl(0 \leq x \wedge 0 \leq y \rightarrow 0 \leq x \cdot y\bigr)$.  \hspace*{\fill} (A2)
  \end{enumerate}
\end{Definition}

Bevor wir ein Beispiel eines geordneten K\"orpers pr\"asentieren, geben wir einige Eigenschaften an, die
aus den Axiomen eines geordneten K\"orpers gefolgert werden k\"onnen.  Das folgende Lemma zeigt, dass
sich eine Ungleichung bei Multiplikation mit $-1$ umdreht.

\begin{Lemma} \label{lemma:l4}
  Es sei  $\mathcal{K} = \langle K, 0, 1, +, \cdot, \leq \rangle$ ein \emph{geordneter K\"orper}.
  Dann gilt
  \\[0.2cm]
  \hspace*{1.3cm} $x \leq y \rightarrow -y \leq -x$  \quad f\"ur alle $x,y \in K$.
\end{Lemma}

\proof
Wir gehen davon aus, dass
\\[0.2cm]
\hspace*{1.3cm}
$x \leq y$
\\[0.2cm]
gilt und zeigen, dass daraus $-y \leq -x$ folgt.  Addieren wir auf beiden Seiten der Ungleichung 
$x \leq y$ den Wert $-y$, was nach dem Axiom (A1) erlaubt ist, so erhalten wir die Ungleichung
\\[0.2cm]
\hspace*{1.3cm}
$x - y \leq 0$.
\\[0.2cm]
Addieren wir auf beiden Seiten dieser Ungleichung den Wert $-x$, so erhalten wir
\\[0.2cm]
\hspace*{1.3cm}
$-y \leq -x$
\\[0.2cm]
und das ist die Behauptung.  \qed

Das n\"achste Lemma zeigt, dass in einem geordneten K\"orper die Addition mit der Relation $\leq$
vertr\"aglich ist.

\begin{Lemma}
  Es sei  $\mathcal{K} = \langle K, 0, 1, +, \cdot, \leq \rangle$ ein \emph{geordneter K\"orper}.
  Dann gilt
  \\[0.2cm]
  \hspace*{1.3cm} $x_1 \leq y_1 \wedge x_2 \leq y_2 \rightarrow x_1 + x_2 \leq y_1 + y_2$  \quad f\"ur alle $x_1,x_2,y_1,y_2 \in K$.
\end{Lemma}

\proof
Wir setzen voraus, dass sowohl $x_1 \leq y_1$ als auch $x_2 \leq y_2$ gilt und zeigen, dass daraus
$x_1 + x_2 \leq y_1 + y_2$ folgt.  Addieren wir auf beiden Seiten der Ungleichung
\\[0.2cm]
\hspace*{1.3cm}
$x_1 \leq y_1$
\\[0.2cm]
den Wert $x_2$, so erhalten wir die Ungleichung
\\[0.2cm]
\hspace*{1.3cm}
$x_1 + x_2 \leq y_1 + x_2$.  \hspace*{\fill} (1)
\\[0.2cm]
Addieren wir auf beiden Seiten der Ungleichung
\\[0.2cm]
\hspace*{1.3cm}
$x_2 \leq y_2$
\\[0.2cm]
den Wert $y_1$, so erhalten wir die Ungleichung
\\[0.2cm]
\hspace*{1.3cm}
$y_1 + x_2 \leq y_1 + y_2$.  \hspace*{\fill} (2)
\\[0.2cm]
Setzen wir die Ungleichungen (1) und (2) zusammen, so folgt aus der Transitivit\"at der Relation
$\leq$, dass 
\\[0.2cm]
\hspace*{1.3cm}
$x_1 + x_2 \leq y_1 + y_2$
\\[0.2cm]
gilt und das war zu zeigen.  \qed

Das n\"achste Lemma zeigt, dass die Relation $\leq$ mit der Multiplikation nicht-negativer Zahlen
vertr\"aglich ist. 

\begin{Lemma} \label{lemma:l6}
  Es sei  $\mathcal{K} = \langle K, 0, 1, +, \cdot, \leq \rangle$ ein \emph{geordneter K\"orper}.
  Dann gilt
  \\[0.2cm]
  \hspace*{1.3cm} $x \leq y \wedge 0 \leq z \rightarrow x \cdot z \leq y \cdot z$  \quad f\"ur alle $x,y,z \in K$.
\end{Lemma}

\proof
Wir setzen voraus, dass  $x \leq y$ und $0 \leq z$ gilt und zeigen, dass in einem geordneten
K\"orper dann auch $x \cdot z \leq y \cdot z$ gelten muss.  Addieren wir auf beiden Seiten der Ungleichung
\\[0.2cm]
\hspace*{1.3cm}
$x \leq y$
\\[0.2cm]
den Wert $-x$, so erhalten wir die Ungleichung
\\[0.2cm]
\hspace*{1.3cm}
$0 \leq y - x$.
\\[0.2cm]
Zusammen mit der Ungleichung $0 \leq z$ zeigt das Axiom (A2) nun, dass
\\[0.2cm]
\hspace*{1.3cm}
$0 \leq (y - x) \cdot z$
\\[0.2cm]
gilt, was wegen dem Distributiv-Gesetz zu
\\[0.2cm]
\hspace*{1.3cm}
$0 \leq y \cdot z - x \cdot z$
\\[0.2cm]
\"aquivalent ist.  Addieren wir auf beiden Seiten dieser Ungleichung den Wert $x \cdot z$, so erhalten wir
\\[0.2cm]
\hspace*{1.3cm}
$x \cdot z \leq y \cdot z$.  \qed

\begin{Lemma} \label{lemma:l7}
  Es sei  $\mathcal{K} = \langle K, 0, 1, +, \cdot, \leq \rangle$ ein \emph{geordneter K\"orper}.
  Dann gilt
  \\[0.2cm]
  \hspace*{1.3cm} $0 \leq x_1 \leq y_1 \wedge 0 \leq x_2 \leq y_2 \rightarrow x_1 \cdot x_2 \leq y_1 \cdot y_2$  
  \quad f\"ur alle $x_1,x_2,y_1,y_2 \in K$.
\end{Lemma}

\proof
Wir setzen voraus, dass $0 \leq x_1 \leq y_1$ und $0 \leq x_2 \leq y_2$ gilt und zeigen, dass daraus 
$x_1 \cdot x_2 \leq y_1 \cdot y_2$ folgt.  Multiplizieren wir die Ungleichung $x_1 \leq y_1$ mit
$x_2$, so sehen wir unter Anwendung des gerade bewiesenen Lemmas, dass
\\[0.2cm]
\hspace*{1.3cm}
$x_1 \cdot x_2 \leq y_1 \cdot x_2$  \hspace*{\fill} (1)
\\[0.2cm]
gilt.  Multiplizieren wir statt dessen die Ungleichung $x_2 \leq y_2$ mit $y_1$, so erhalten wir die
Ungleichung 
\\[0.2cm]
\hspace*{1.3cm}
$y_1 \cdot x_2 \leq y_1 \cdot y_2$. \hspace*{\fill} (2)
\\[0.2cm]
Aus den beiden Ungleichungen (1) und (2) folgt mit der Transitivit\"at der Relation $\leq$ nun
\\[0.2cm]
\hspace*{1.3cm}
$x_1 \cdot x_2 \leq y_1 \cdot y_2$
\\[0.2cm]
und das ist die Behauptung.  \qed

Das n\"achste Lemma zeigt, dass sich Ungleichungen bei Multiplikation mit nicht positiven Zahlen
umdrehen. 

\begin{Lemma} 
  Es sei  $\mathcal{K} = \langle K, 0, 1, +, \cdot, \leq \rangle$ ein \emph{geordneter K\"orper}.
  Dann gilt
  \\[0.2cm]
  \hspace*{1.3cm} $x \leq y \wedge z \leq 0 \rightarrow y \cdot z \leq x \cdot z$  
  \quad f\"ur alle $x,y,z \in K$.
\end{Lemma}

\proof
Aus der Voraussetzung $z \leq 0$ folgt mit Lemma \ref{lemma:l4}
\\[0.2cm]
\hspace*{1.3cm}
$-0 \leq -z$, \quad also $0 \leq -z$.
\\[0.2cm]
Mit Lemma \ref{lemma:l6} folgt aus $x \leq y$ und $0 \leq -z$ die Ungleichung
\\[0.2cm]
\hspace*{1.3cm}
$x \cdot (-z) \leq y \cdot (-z)$,
\\[0.2cm]
was wir auch als
\\[0.2cm]
\hspace*{1.3cm}
$-x \cdot z \leq -y \cdot z$
\\[0.2cm]
schreiben k\"onnen.  Daraus folgt mit Lemma \ref{lemma:l4} die Ungleichung
\\[0.2cm]
\hspace*{1.3cm}
$y \cdot z \leq x \cdot z$. \qed

Nun zeigen wir, dass ein Quadrat nie negativ ist.

\begin{Lemma} 
  Es sei  $\mathcal{K} = \langle K, 0, 1, +, \cdot, \leq \rangle$ ein \emph{geordneter K\"orper}.
  Dann gilt
  \\[0.2cm]
  \hspace*{1.3cm} $0 \leq x \cdot x$  
  \quad f\"ur alle $x \in K$.
\end{Lemma}

\proof
Da $\leq$ eine lineare Ordnung ist, haben wir
\\[0.2cm]
\hspace*{1.3cm}
$0 \leq x \vee x \leq 0$.
\\[0.2cm] 
Wir f\"uhren eine Fallunterscheidung durch.
\begin{enumerate}
\item Fall: $0 \leq x$.

      Multiplizieren wir die Ungleichung $0 \leq x$ mit der Ungleichung $0 \leq x$, 
      so folgt aus dem Axiom (A2) sofort die Behauptung
      \\[0.2cm]
      \hspace*{1.3cm}
      $0 \leq x \cdot x$.
\item Fall: $x \leq 0$.

      Nach Lemma \ref{lemma:l4} folgt dann
      \\[0.2cm]
      \hspace*{1.3cm}
      $0 \leq -x$.
      \\[0.2cm]
      Multiplizieren wir diese Ungleichung mit sich selbst, so folgt aus dem Axiom (A2)
      \\[0.2cm]
      \hspace*{1.3cm}
      $0 \leq (-x) \cdot (-x)$
      \\[0.2cm]
      und wegen $(-x) \cdot (-x) = x \cdot x$ ist das die Behauptung.  \qed
\end{enumerate}

Das n\"achste Lemma zeigt, in welcher Weise die Relation $\leq$ mit dem multiplikativen Inversen
vertr\"aglich ist.

\begin{Lemma} 
  Es sei  $\mathcal{K} = \langle K, 0, 1, +, \cdot, \leq \rangle$ ein \emph{geordneter K\"orper}.
  Dann gilt
  \\[0.2cm]
  \hspace*{1.3cm} $0 \leq x \wedge x \not= 0 \rightarrow 0 \leq x^{-1}$
  \quad f\"ur alle $x \in K$.
\end{Lemma}

\proof
Es gelte $0 \leq x$.  Aus der Tatsache, dass $x \not= 0$ ist, folgt, dass $x^{-1}$ definiert ist.
Nach dem letzten Lemma gilt daher
\\[0.2cm]
\hspace*{1.3cm}
$0 \leq x^{-1} \cdot x^{-1}$ 
\\[0.2cm]
Multiplizieren wir diese Ungleichung mit $x$, so zeigt Lemma \ref{lemma:l6}, dass dann auch
\\[0.2cm]
\hspace*{1.3cm}
$0 \cdot x \leq x^{-1} \cdot x^{-1} \cdot x$
\\[0.2cm]
gilt und wegen $0 \cdot x = 0$ und $x^{-1} \cdot x^{-1} \cdot x$ ist diese Aussage
\"aquivalent zu der Behauptung. \qed

\begin{Lemma} 
  Es sei  $\mathcal{K} = \langle K, 0, 1, +, \cdot, \leq \rangle$ ein \emph{geordneter K\"orper}.
  Dann gilt
  \\[0.2cm]
  \hspace*{1.3cm} $0 \leq x \leq y \wedge x \not= 0 \rightarrow y^{-1} \leq x^{-1}$
  \quad f\"ur alle $x, y \in K$.  
\end{Lemma}

\exercise
Beweisen Sie diese Behauptung.  \eox

\example
Die Struktur $\langle \mathbb{Q}, 0, 1, +, \cdot, \leq \rangle$ ist ein geordneter K\"orper, wenn wir
die Operationen $+$, $\cdot$ und $\leq$ wie im ersten Semester vorgef\"uhrt auf den rationalen Zahlen
definieren.  
\eox

\exercise
Zeigen Sie, dass es nicht m\"oglich ist, auf der Menge $\mathbb{C}$ der komplexen Zahlen eine Relation
$\leq$ so zu definieren, dass $\langle \mathbb{C}, 0, 1, +, \cdot, \leq \rangle$ ein geordneter
K\"orper ist. \eox

\begin{Definition}[Absolut-Betrag]
  Es sei  $\mathcal{K} = \langle K, 0, 1, +, \cdot, \leq \rangle$ ein \emph{geordneter K\"orper}.
  Dann definieren wir den \colorbox{gold}{\emph{Absolut-Betrag}} $|x|$ eines Elements $x \in K$ wie folgt:
  \\[0.2cm]
  \hspace*{1.3cm}
  $|x| := \left\{
  \begin{array}{rl}
     x & \mbox{falls $0 \leq x$};  \\[0.1cm]
    -x & \mbox{sonst.}
  \end{array}
  \right.
  $\eox
\end{Definition}

\exercise
Es sei  $\mathcal{K} = \langle K, 0, 1, +, \cdot, \leq \rangle$ ein \emph{geordneter K\"orper}.
Weisen Sie die folgenden Eigenschaften des Absolut-Betrags nach:
\begin{enumerate}[(a)]
\item $0 \leq |x|$
\item $x \leq |x|$ \quad und \quad $-x \leq |x|$
\item $|-x| = |x|$
\item $|x| = 0 \;\leftrightarrow\; x = 0$
\item $|x \cdot y| = |x| \cdot |y|$
\item $|x+y| \leq |x| + |y|$ \eox
\end{enumerate}


Die Struktur $\langle \mathbb{Q}, 0, 1, +, \cdot, \leq \rangle$ ist  f\"ur die Zwecke der Analysis nicht ausreichend,
da sie in einer noch n\"aher zu spezifizierenden Weise \emph{unvollst\"andig} ist.  Zur Pr\"azisierung dieser Aussage
ben\"otigen wir den Begriff einer \emph{vollst\"andigen Ordnung}, die ihrerseits
auf dem Begriff des \emph{Supremums} basiert, den wir jetzt einf\"uhren.


\begin{Definition}[Supremum]
Es sei $\langle M, \leq \rangle$ eine lineare Ordnung.
Wir nennen die Menge $A \subseteq M$  \linebreak
\colorbox{gold}{\emph{nach oben beschr\"ankt}}, falls es ein
 $y \in M$ gibt, so dass gilt:
\\[0.2cm]
\hspace*{1.3cm}
$\forall x \in A:  x \leq y$.
\\[0.2cm]
Dieses $y$ bezeichnen wir dann als eine \colorbox{gold}{\emph{obere Schranke}} der Menge $A$.
Ein Element $s \in M$ ist das \colorbox{gold}{\emph{Supremum}} der Menge $A$, wenn $s$ die 
\colorbox{gold}{\emph{kleinste obere Schranke}} der Menge $A$ ist, wenn also 
\\[0.2cm]
\hspace*{1.3cm}
$\forall x \in A: x \leq s$ \quad \mbox{und} \quad
$\forall y \in M: \Bigl(\bigl(\forall x \in A: x \leq y \bigr)\rightarrow s \leq y \Bigr)$
\\[0.2cm]
gilt.  In diesem Fall schreiben wir
\\[0.2cm]
\hspace*{1.3cm}
$s = \sup(A)$.
\edx
\end{Definition}

\begin{Definition}[Vollst\"andige Ordnung]
  Ein Paar $\pair(M, \leq)$ bestehend aus einer Menge $M$ und einer Relation $\leq\; \subseteq M \times M$ 
  ist eine \colorbox{gold}{\emph{vollst\"andige}} Ordnung genau denn, wenn folgendes gilt:
  \begin{enumerate}
  \item Das Paar $\pair(M, \leq)$ ist eine lineare Ordnung.
  \item Zu jeder nicht-leeren und nach oben beschr\"ankten Menge $A \subseteq M$ existiert ein
        Supremum in $M$. \edx
  \end{enumerate}
\end{Definition}
\vspace*{-0.2cm}


\noindent
\textbf{Bemerkung}:
Das Paar $\langle \mathbb{Q}, \leq \rangle$ ist \underline{keine} vollst\"andige Ordnung, denn wenn wir die Menge
$A$ als
\\[0.2cm]
\hspace*{1.3cm}
$A := \{ r \in \mathbb{Q} \mid r^2 \leq 2 \}$
\\[0.2cm]
definieren, so ist die Menge $A$ nach oben beschr\"ankt, weil die Zahl $2$ eine obere
Schranke von $A$ ist.  Die Menge $A$ hat aber keine kleinste obere Schranke.  Anschaulich liegt das
daran, dass die kleinste obere Schranke  der Menge $A$ die Zahl $\sqrt{2}$ ist, aber aus dem ersten
Semester wissen wir, dass $\sqrt{2}$ keine rationale Zahl ist.  \eox


\noindent
Analog zum Begriff des Supremums k\"onnen wir auch den Begriff des Infimums definieren.

\begin{Definition}[Infimum]
Es sei $\pair(M, \leq)$ eine lineare Ordnung.  Wir nennen die Menge $B \subseteq M$ \linebreak
\colorbox{gold}{\emph{nach unten beschr\"ankt}}, falls es ein
 $y \in M$ gibt, so dass 
\\[0.2cm]
\hspace*{1.3cm}
$\forall x \in B:  y \leq x$
\\[0.2cm]
gilt.  Ein solches $y$ bezeichnen wir als \colorbox{gold}{\emph{untere Schranke}}  von $B$.
Ein Element $i \in M$ ist das \colorbox{gold}{\emph{Infimum}} einer Menge $B$, wenn $i$ die gr\"o\ss{}te
untere Schranke von $B$ ist, wenn also 
\\[0.2cm]
\hspace*{1.3cm}
$\forall x \in B: i \leq x$ \quad \mbox{und} \quad
$\forall y \in M: \Bigl(\bigl(\forall x \in B: y \leq x \bigr)\rightarrow y \leq i \Bigr)$
\\[0.2cm]
gilt.  In diesem Fall schreiben wir
\\[0.2cm]
\hspace*{1.3cm}
$i = \inf(B)$.
\edx
\end{Definition}


\exercise
Es sei $\pair(M, \leq)$ eine vollst\"andige lineare Ordnung.  Zeigen Sie, dass f\"ur jede Teilmenge
$B \subseteq M$, die nicht-leer und nach unten beschr\"ankt ist, ein Infimum existiert.
\vspace*{0.2cm}

\noindent
\textbf{Hinweis}: Betrachten Sie die Menge
\\[0.2cm]
\hspace*{1.3cm}
$A := \{ x \in M \mid \forall y \in B: x \leq y \}$
\\[0.2cm]
der unteren Schranken von $B$.  Diese Menge ist nicht leer, denn nach Voraussetzung ist $B$ nach
unten beschr\"ankt.  Au\ss{}erdem ist $A$ nach oben beschr\"ankt, denn jedes Element aus der Menge $B$ ist
eine obere Schranke von $A$.  Da nach Voraussetzung das Paar $\pair(M, \leq)$ eine vollst\"andige
lineare Ordnung ist, besitzt $A$ also ein Supremum.  Zeigen Sie, dass dieses Supremum auch das
Infimum von $B$ ist. \eox

Wir kommen nun zur zentralen Definition dieses Abschnitts.
\begin{Definition}[Vollst\"andig angeordneter K\"orper] \hspace*{\fill} \linebreak
  Eine Struktur  $\mathcal{K} = \langle K, 0, 1, +, \cdot, \leq \rangle$  ist ein 
  \colorbox{gold}{\emph{vollst\"andig angeordneter K\"orper}} genau dann, wenn die Struktur $\mathcal{K}$
  ein geordneter K\"orper ist und zus\"atzlich das Paar $\langle K, \leq \rangle$ eine vollst\"andige
  Ordnung ist. \edx
\end{Definition}

\begin{Theorem}
  Die Struktur $\langle \mathbb{R}, 0, 1, +, \cdot, \leq \rangle$ der reellen Zahlen ist ein  vollst\"andig geordneter K\"orper.   \edx
\end{Theorem}

Einen Beweis diese Behauptung k\"onnen wir jetzt noch nicht erbringen, denn wir haben bisher noch gar nicht
gesagt, wie die Menge $\mathbb{R}$ der reellen Zahlen formal definiert wird.
Die Definition der reellen Zahlen werden wir im n\"achsten Abschnitt mit Hilfe der sogenannten
\colorbox{gold}{\emph{Dedekind-Schnitte}} liefern.  f\"ur praktische Rechnungen ist diese 
Konstruktion viel zu schwerf\"allig.  Wir skizzieren  daher zum Abschluss dieses
Abschnitts noch eine andere Konstruktion der reellen Zahlen als 
\colorbox{gold}{\emph{unendliche Dezimalbr\"uche}}, die
auf \href{http://en.wikipedia.org/wiki/Simon_Stevin}{Simon Stevin} (1548-1620) zur\"uck geht.  Um
diese Konstruktion verstehen zu k\"onnen,
betrachten wir eine positive reelle Zahl $x \in \mathbb{R}$ abstrakt als gegeben und untersuchen, wie wir $x$
in der Form
\\[0.2cm]
\hspace*{1.3cm}
$\ds x = m + \sum\limits_{k=1}^\infty b_k \cdot \frac{1}{10^k}$
\\[0.2cm]
so darstellen k\"onnen, dass folgendes gilt:
\begin{enumerate}
\item $m \in \mathbb{N}_0$ \quad und
\item $b_k \in \{0,\cdots,9\}$.
\end{enumerate}
Die Zahl $b_k$ kann damit als die $k$-te Stelle von $x$ hinter dem Komma interpretiert werden.
Die Berechnung der Zahlen $m$ und $b_k$ verl\"auft f\"ur ein gegebenes positives $x \in \mathbb{R}$ wie folgt:
\begin{enumerate}
\item $m := \max(\{ n \in \mathbb{N} \mid n \leq x \})$.

      $m$ ist also die gr\"o\ss{}te ganze Zahl, die nicht gr\"o\ss{}er als $x$ ist.
\item Die Zahlen $b_k$ werden f\"ur alle $k \in \mathbb{N}$ induktiv definiert.
      \begin{enumerate}
      \item[I.A.:] $k=1$.  Wir setzen
                   \\[0.2cm]
                   \hspace*{1.3cm}
                   $\ds b_1 := \max\Bigl(\Bigl\{ z \in \{ 0, 1, \cdots, 9 \} \Bigm| m + z \cdot \frac{1}{10} \leq x\Bigr\}\Bigr)$.
      \item[I.S.:] $k \mapsto k+1$.  Nach Induktions-Voraussetzung sind die Zahlen $b_1$, $\cdots$, $b_k$
                   bereits definiert.  Daher k\"onnen wir $b_{k+1}$ als
                   \\[0.2cm]
                   \hspace*{1.3cm}
                   $\ds b_{k+1} := \max\Bigl(\Bigl\{ z \in \{ 0, 1, \cdots, 9 \} \Bigm| m + \sum\limits_{i=1}^{k} b_i \cdot \frac{1}{10^i} + z \cdot \frac{1}{10^{k+1}}\leq x\Bigr\}\Bigr)$.
                   \\[0.2cm]
                   definieren.
      \end{enumerate}
\end{enumerate}
Insgesamt gilt mit dieser Konstruktion
\\[0.2cm]
\hspace*{1.3cm}
$\ds x = m + \sum\limits_{k=1}^{\infty} b_k \cdot \frac{1}{10^{k}}$.
\\[0.2cm]
Diese Behauptung k\"onnen wir allerdings zum jetzigen Zeitpunkt nicht beweisen, denn wir haben ja noch gar nicht formal
definiert, welchen Wert wir einer unendlichen Reihe der Form
\\[0.2cm]
\hspace*{1.3cm}
$\ds \sum\limits_{k=1}^{\infty} b_k \cdot \frac{1}{10^{k}}$
\\[0.2cm]
zuordnen.  Statt dessen k\"onnen wir sagen, dass
\\[0.2cm]
\hspace*{1.3cm}
$\ds x = \sup\Bigl(\Bigl\{ m + \sum\limits_{i=1}^{k} b_i \cdot \frac{1}{10^{i}} \Bigm| k \in \mathbb{N} \Bigr\}\Bigr)$
\\[0.2cm]
gilt, denn die Menge
\\[0.2cm]
\hspace*{1.3cm}
$\ds \Bigl\{ m + \sum\limits_{i=1}^{k} b_i \cdot \frac{1}{10^{i}} \Bigm| k \in \mathbb{N} \Bigr\}$
\\[0.2cm]
ist offenbar durch $x$ nach oben beschr\"ankt und es l\"asst sich zeigen, dass es keine 
obere Schranke $y$ f\"ur diese Menge gibt, die  kleiner als $x$ ist.  f\"ur den Rest der Vorlesung
reicht es aus, wenn Sie sich die reellen Zahlen wie oben skizziert als unendliche Dezimalbr\"uche vorstellen.



\section{Die formale Konstruktion der reellen Zahlen$^*$}
Bis jetzt haben wir so getan, als w\"ussten wir schon, was reelle Zahlen sind und haben Ihre
Eigenschaften in Form von Axiomen angegeben.  Wir wissen  aber noch gar nicht, ob es tats\"achlich
eine Struktur gibt, die diesen Axiomen gen\"ugt.  Diesen Nachweis werden wir jetzt mit Hilfe der
Mengenlehre erbringen.  Dabei gehen wir von folgendem Beispiel aus:  Wir wissen, dass $\sqrt{2}$
kein Element der rationalen Zahlen ist.  Wir k\"onnen aber $\sqrt{2}$ durch zwei Mengen rationaler
Zahlen beschreiben, wenn wir
\\[0.2cm]
\hspace*{1.3cm}
$M_1 := \{ q \in \mathbb{Q} \mid q \leq 0 \vee q^2 < 2 \}$ \quad und \quad 
$M_2 := \{ q \in \mathbb{Q} \mid q \geq 0 \wedge 2 < q^2 \}$,
\\[0.2cm]
definieren, denn dann liegt $\sqrt{2}$ gerade zwischen $M_1$ und $M_2$.  
Versuchen wir die diesem Beispiel zugrunde liegende Idee zu pr\"azisieren, so kommen wir zur nun
folgende Definition eines \emph{Dedekind'schen-Schnittes}.

\begin{Definition}[Dedekind-Schnitt] \lb
Ein Paar $\pair(M_1,M_2)$ hei\ss{}t \href{http://de.wikipedia.org/wiki/Dedekindscher_Schnitt}{\emph{Dedekind-Schnitt}}
(\href{http://de.wikipedia.org/wiki/Richard_Dedekind}{\textrm{Richard Dedekind}}, 1831-1916)
falls folgendes gilt:
\begin{enumerate}
\item $M_1 \subseteq  \mathbb{Q}$, \quad $M_2 \subseteq \mathbb{Q}$.
\item $M_1 \not= \emptyset$, \quad $M_2 \not= \emptyset$.
\item $\forall x_1 \in M_1: \forall x_2 \in M_2: x_1 < x_2$.

       Diese Bedingung besagt, dass alle Elemente aus $M_1$ kleiner als alle Elemente aus $M_2$
       sind.  Diese Bedingung bezeichnen wir als die \colorbox{gold}{\emph{Trennungs-Eigenschaft}}.
\item $M_1 \cup M_2 = \mathbb{Q}$.
\item $M_1$ hat kein Maximum.

      Da alle Elemente aus $M_1$ kleiner als alle Elemente von $M_2$ sind und da dar\"uber hinaus 
      $M_2 \not= \emptyset$ ist, ist $M_1$ sicher nach oben beschr\"ankt.  Aber wenn f\"ur ein $y$
      \\[0.2cm]
      \hspace*{1.3cm}
      $\forall x \in M_1: x \leq y$
      \\[0.2cm]
      gilt, dann darf $y$ eben kein Element von $M_1$ sein.  Als Formel schreibt sich das als
      \\[0.2cm]
      \hspace*{1.3cm}
      $\forall y \in \mathbb{Q}: \bigl((\forall x \in M_1: x \leq y) \rightarrow y \not\in M_1\bigr)$. \edx
\end{enumerate}
\end{Definition}

\example 
Definieren wir
\\[0.2cm]
\hspace*{1.3cm} 
$M_1 := \{ x \in \mathbb{Q} \mid x \leq 0 \vee x^2 < 2 \}$ \quad und \quad
$M_2 := \{ x \in \mathbb{Q} \mid x > 0 \wedge x^2 > 2 \}$,
\\[0.2cm]
so enth\"alt $M_1$ alle die Zahlen, die kleiner als $\sqrt{2}$ sind, w\"ahrend
$M_2$ alle Zahlen enth\"alt, die gr\"o\ss{}er als $\sqrt{2}$ sind. Das Paar $\pair(M_1,M_2)$ ist dann ein
Dedekind-Schnitt.  Intuitiv spezifiziert dieser Dedekind-Schnitt die Zahl $\sqrt{2}$.
\eox



Bei einem Dedekind-Schnitt $\pair(M_1,M_2)$ ist die Menge $M_2$ durch die Angabe von
$M_1$ bereits vollst\"andig festgelegt, denn aus der Gleichung $M_1 \cup M_2 = \mathbb{Q}$ folgt
sofort $M_2 = \mathbb{Q} \backslash M_1$.  Die Frage ist nun, welche Eigenschaften eine Menge $M$
haben muss, damit umgekehrt das Paar $\pair(M, \mathbb{Q} \backslash M)$ ein Dedekind-Schnitt ist.
Die Antwort auf diese Frage wird in der nun folgenden Definition einer \emph{Dedekind-Menge} gegeben.


\begin{Definition}[Dedekind-Menge]
Eine Menge $M \subseteq \mathbb{Q}$ ist eine \colorbox{gold}{\emph{Dedekind-Menge}} genau dann, wenn die
folgenden Bedingungen erf\"ullt sind.
\begin{enumerate}
\item $M \not= \emptyset$,
\item $M \not= \mathbb{Q}$,
\item $\forall x, y \in \mathbb{Q}: \bigl(y < x \wedge x \in M \rightarrow y \in M)$.

      Die letzte Bedingung besagt, dass $M$ \emph{nach unten abgeschlossen} ist:  Wenn eine
      Zahl $x$ in $M$ liegt, dann liegt auch jede Zahl, die kleiner als $x$ ist, in $M$.
\item Die Menge $M$ hat kein Maximum, es gibt also kein $m \in M$, so dass
      \\[0.2cm]
      \hspace*{1.3cm}
      $x \leq m$ \quad f\"ur alle $x \in M$ gilt.
      \\[0.2cm]
      Diese Bedingung k\"onnen wir auch etwas anders formulieren:  Wenn $x \in M$ ist, dann finden wir
      immer ein $y \in M$, dass noch gr\"o\ss{}er als $x$ ist, denn sonst w\"are $x$ ja das Maximum von $M$.
      Formal k\"onnen wir das als
      \\[0.2cm]
      \hspace*{1.3cm}
      $\forall x \in M: \exists y \in M: x < y$
      \\[0.2cm]
      schreiben.
\end{enumerate}
\end{Definition}

\exercise
Zeigen Sie, dass eine Menge $M \subseteq \mathbb{Q}$ genau dann eine Dedekind-Menge ist, wenn das Paar $\pair(M,\mathbb{Q} \backslash M)$
ein Dedekind-Schnitt ist. \eox

\solution
Da es sich bei der zu beweisenden Aussage um eine \"Aquivalenz-Aussage handelt,  zerf\"allt der Beweis
in zwei Teile.
\begin{enumerate}
\item[``$\Rightarrow$'':] Zun\"achst nehmen wir an, dass $M \subseteq \mathbb{Q}$ eine Dedekind-Menge ist.  Wir haben zu
      zeigen, dass dann $\pair(M,\mathbb{Q} \backslash M)$ ein Dedekind-Schnitt ist.  Von den zu
      \"uberpr\"ufenden Eigenschaften ist nur die Trennungs-Eigenschaft nicht offensichtlich.
      Sei als $x \in M$ und $y \in \mathbb{Q} \backslash M$.  Wir haben zu zeigen, dass dann
      \\[0.2cm]
      \hspace*{1.3cm}
      $x < y$
      \\[0.2cm]
      gilt.  Wir f\"uhren diesen Nachweis indirekt und nehmen an, dass $y \leq x$.  Da $M$ nach unten
      abgeschlossen ist, folgt daraus aber $y \in M$, was im Widerspruch zu $y \in \mathbb{Q} \backslash M$ steht.
      Dieser Widerspruch zeigt, dass $x < y$ ist und das war zum Nachweis der Trennungs-Eigenschaft
      zu zeigen.
\item[``$\Leftarrow$'':] Nun nehmen wir an, dass $\pair(M, \mathbb{Q} \backslash M)$ ein Dedekind-Schnitt ist und
     zeigen, dass dann $M$ eine Dedekind-Menge sein muss.  Von den zu \"uberpr\"ufenden Eigenschaften
     ist nur Tatsache, dass $M$ nach unten abgeschlossen ist, nicht offensichtlich.  Sei also $x \in M$
     und $y < x$.  Wir haben zu zeigen, dass dann $y$ ebenfalls ein Element von $M$ ist.  Wir f\"uhren
     diesen Nachweis indirekt und nehmen $y \in \mathbb{Q} \backslash M$ an.  Aufgrund der
     Trennungs-Eigenschaft des Dedekind-Schnitts $\pair(M, \mathbb{Q} \backslash M)$ muss dann
     \\[0.2cm]
     \hspace*{1.3cm}
     $x < y$
     \\[0.2cm]
     gelten, was im Widerspruch zu $y < x$ steht.  Dieser Widerspruch zeigt, dass $y \in M$ gilt
     und das war zu zeigen.  \qed
\end{enumerate}

Die letzte Aufgabe hat gezeigt, dass Dedekind-Schnitte und Dedekind-Mengen zu einander \"aquivalent
sind.  Daher werden wir im Folgenden mit Dedekind-Mengen
arbeiten,  denn das macht die Notation einfacher.  Wir definieren dazu
$\mathcal{D}$ als die Menge aller rationalen Dedekind-Mengen, wir setzen also
\\[0.2cm]
\hspace*{1.3cm}
\colorbox{orange}{
$\mathcal{D} := \bigl\{ M \in 2^{\mathbb{Q}} \mid \mbox{$M$ ist Dedekind-Menge} \bigr\}$.}
\\[0.2cm]
Wir wollen zeigen, dass wir die Menge $\mathbb{R}$ der reellen Zahlen als die Menge $\mathcal{D}$
definieren k\"onnen.  Dazu m\"ussen wir nun zeigen, wie sich auf der Menge $\mathcal{D}$ die
arithmetischen Operationen Addition, Subtraktion, Multiplikation und Division definieren 
lassen und wie die Relation $\leq$ f\"ur zwei Dedekind-Schnitte festgelegt
werden kann.  Zus\"atzlich m\"ussen wir nachweisen, dass wir mit diesen Definitionen einen vollst\"andig
angeordneten K\"orper konstruieren.  Wir beginnen mit der Definition der Relation $\leq$.

\exercise
Auf der Menge $\mathcal{D}$ definieren wir eine bin\"are Relation $\leq$ durch die Festsetzung
\\[0.2cm]
\hspace*{1.3cm}
$A \leq B \;\stackrel{\mathrm{def}}{\Longleftrightarrow}\; A \subseteq B$ \quad f\"ur alle $A,B \in \mathcal{D}$.
\\[0.2cm]
Zeigen Sie, dass die so definierte Relation $\leq$ eine lineare Ordnung auf der Menge $\mathcal{D}$ ist.
\pagebreak

\solution
Es ist zu zeigen, dass die Relation $\leq$ reflexiv, anti-symmetrisch und transitiv ist und dass au\ss{}erdem die Linearit\"ats-Eigenschaft
\\[0.2cm]
\hspace*{1.3cm} $A \leq B \vee B \leq A$ \quad f\"ur alle Dedekind-Mengen $A,B \in \mathcal{D}$
\\[0.2cm]
gilt.  Die Reflexivit\"at, Anti-Symmetrie und Transitivit\"at der Relation $\leq$ folgen sofort aus der Reflexivit\"at,
Anti-Symmetrie und Transitivit\"at der Teilmengen-Relation $\subseteq$.  Es bleibt, den Nachweis der
Linearit\"ats-Eigenschaft zu f\"uhren.  Seien also $A,B \in \mathcal{D}$ gegeben.  Falls $A = B$ ist, gilt sowohl
$A \subseteq B$ als auch $B \subseteq A$, woraus sofort $A \leq B$ und $B \leq A$ folgt.  Wir nehmen
daher an, dass $A \not= B$ ist.  Dann gibt es zwei M\"oglichkeiten:
\begin{enumerate}
\item Fall: Es existiert ein $x \in \mathbb{Q}$ mit $x \in A$ und $x \not\in B$.

            Wir zeigen, dass dann $B \subseteq A$, also $B \leq A$ gilt.  Zum Nachweis der Beziehung
            $B \subseteq A$ nehmen wir an, dass $y \in B$ ist und m\"ussen $y \in A$ zeigen.

            Wir behaupten, dass $y < x$ ist und f\"uhren den Beweis dieser Behauptung indirekt, nehmen also
            $x \leq y$ an.  Da die Dedekind-Menge $B$ nach unten abgeschlossen ist und $y \in B$ ist, w\"urde daraus
            \\[0.2cm]
            \hspace*{1.3cm}
            $x \in B$
            \\[0.2cm]
            folgen, was im Widerspruch zu der in diesem Fall gemachten Annahme $x \not\in B$ steht.  Also
            haben wir
            \\[0.2cm]
            \hspace*{1.3cm}
            $y < x$.
            \\[0.2cm]
            Da die Menge $A$ als Dedekind-Menge nach unten abgeschlossen ist und $x \in A$ ist, folgt
            \\[0.2cm]
            \hspace*{1.3cm}
            $y \in A$,
            \\[0.2cm]
            so dass wir $B \subseteq A$ gezeigt haben.
\item Fall: Es existiert ein $x \in \mathbb{Q}$ mit $x \in B$ und $x \not\in A$.

            Dieser Fall ist offenbar analog zum ersten Fall.
            \qed
\end{enumerate}


\exercise
Zeigen Sie, dass jede nicht-leere und in $\mathcal{D}$ nach oben beschr\"ankte Menge
$\mathcal{M} \subseteq \mathcal{D}$ ein Supremum $S \in \mathcal{D}$ hat.  
\vspace*{0.2cm}

\noindent
\textbf{Hinweis}: Sie k\"onnen das Supremum von $\mathcal{M}$ als die Vereinigung aller Mengen aus
$\mathcal{M}$ definieren.  Es gilt also
\\[0.2cm]
\hspace*{1.3cm}
$\sup(\mathcal{M}) = \bigcup \mathcal{M} := \{ x \in \mathbb{Q} \mid \exists A \in \mathcal{M}: x \in A \}$.
\eox

\noindent
Als n\"achstes definieren wir eine Addition auf Dedekind-Mengen.


\begin{Definition}[Addition von Dedekind-Mengen] \hspace*{\fill} \linebreak
Es seien $A$ und $B$ Dedekind-Mengen.  Dann definieren wir die Summe $A + B$ wie folgt:
\\[0.2cm]
\hspace*{1.3cm}
$A + B := \{ x + y \mid x \in A \wedge y \in B \}$. \edx
\end{Definition}

\exercise
Es seien $A,B \in \mathcal{D}$.  Zeigen Sie, dass dann auch $A + B \in \mathcal{D}$ ist.
\eox

\solution
Wir zeigen, dass $A + B$ eine Dedekind-Menge ist.
\begin{enumerate}
\item Wir zeigen $A + B \not= \{\}$.

      Da $A$ eine Dedekind-Menge ist, gibt es ein Element $a \in A$ und da $B$ ebenfalls eine
      Dedekind-Menge ist, gibt es auch ein Element $b \in B$.  Nach Definition von $A + B$ folgt dann
      $a + b \in A + B$ und damit gilt $A + B \not= \{\}$.
\pagebreak

\item Wir zeigen $A + B \not= \mathbb{Q}$.

      Da $A$ und $B$ Dedekind-Mengen sind, gilt $A \not= \mathbb{Q}$ und $B \not= \mathbb{Q}$.  Also gibt es 
      $x,y \in \mathbb{Q}$ mit $x \not\in A$ und $y \not\in B$.  Wir zeigen, dass dann  $x + y \not\in A + B$
      ist und f\"uhren diesen Nachweis indirekt.  Wir nehmen also an, dass
      \\[0.2cm]
      \hspace*{1.3cm}
      $x + y \in A + B$
      \\[0.2cm]
      gilt.  Nach Definition der Menge $A + B$ gibt es dann ein $a \in A$ und ein $b \in B$, so dass
      \\[0.2cm]
      \hspace*{1.3cm}
      $x + y = a + b$
      \\[0.2cm]
      ist.  Aus $x \not\in A$ und $a \in A$ folgt, dass 
      \\[0.2cm]
      \hspace*{1.3cm}
      $a < x$ 
      \\[0.2cm]
      ist, denn da $A$ eine Dedekind-Menge ist, w\"urde aus $x \leq a$ sofort $x \in A$ folgern.  Weil $B$ eine
      Dedekind-Menge ist, gilt dann auch
      \\[0.2cm]
      \hspace*{1.3cm}
      $b < y$.
      \\[0.2cm]
      Addieren wir diese beiden Ungleichungen, so erhalten wir
      \\[0.2cm]
      \hspace*{1.3cm}
      $a + b < x + y$,
      \\[0.2cm]
      was im Widerspruch zu der Gleichung $x + y = a + b$ steht.
\item Wir zeigen, dass die Menge $A + B$ nach unten abgeschlossen ist.

      Sei also $x \in A + B$ und $y < x$.  Nach Definition von $A + B$ gibt es dann ein $a \in A$ und ein
      $b \in B$, so dass $x = a + b$ gilt.  Wir definieren
      \\[0.2cm]
      \hspace*{1.3cm} 
      $c := x - y$, \quad $u := a - \bruch{1}{2} \cdot c$ \quad und \quad
      $v := b - \bruch{1}{2} \cdot c$.
      \\[0.2cm]
      Aus $y < x $ folgt zun\"achst $c > 0$ und daher gilt $u < a$ und $v < b$.  Da $a \in A$ ist und die
      Menge $A$ als Dedekind-Menge nach unten abgeschlossen ist, folgt $u \in A$.  Analog sehen wir, dass
      auch $v \in B$ ist.  Insgesamt folgt dann
      \\[0.2cm]
      \hspace*{1.3cm}
      $u + v \in A + B$.
      \\[0.2cm]
      Wir haben aber
      \\[0.2cm]
      \hspace*{1.3cm}
      $
      \begin{array}[t]{lcll}
        u + v & = & a - \bruch{1}{2} \cdot c + b - \bruch{1}{2} \cdot c \\[0.2cm]
              & = & a + b - c                                           \\[0.2cm]
              & = & a + b - (x - y)                                     
                  & \mbox{denn $c = x - y$}                             \\[0.2cm]
              & = & x - (x - y)                                     
                  & \mbox{denn $x = a + b$}                             \\[0.2cm]
              & = & y
      \end{array}
      $
      \\[0.2cm]
      Wegen $u + v \in A + B$ haben wir insgesamt $y \in A + B$ nachgewiesen, was zu zeigen war.
\item Wir zeigen, dass die Menge $A + B$ kein Maximum enth\"alt.

      Wir f\"uhren den Beweis indirekt und nehmen an, dass ein $m \in A + B$ existiert, so dass
      $m = \max(A + B)$ gilt.  Nach Definition der Menge $A + B$ gibt es dann ein $a \in A$ und ein 
      $b \in B$ so dass $m = a + b$ ist.  Sei nun $u \in A$.  Wir wollen zeigen, dass $u \leq a$
      ist. w\"are $u > a$, dann w\"urde auch 
      \\[0.2cm]
      \hspace*{1.3cm}
      $u + b > a + b$
      \\[0.2cm]
      gelten, und da $u + b \in A + B$ ist, k\"onnte $m$ dann nicht das Maximum der Menge $A + B$ sein.
      Also gilt $u \leq a$.  Dann ist aber $a$ das Maximum der Menge $A$ und au\ss{}erdem in $A$ enthalten.
      Dies ist ein Widerspruch zu der Voraussetzung, dass $A$ eine Dedekind-Menge ist. \qed
\end{enumerate}

\exercise
Zeigen Sie, dass die Menge 
\\[0.2cm]
\hspace*{1.3cm}
$O := \{ x \in \mathbb{Q} \mid x < 0 \}$
\\[0.2cm]
eine Dedekind-Menge ist.
\eox

\solution
Wir zeigen, dass $O$ eine Dedekind-Menge ist, indem wir die einzelnen Eigenschaften einer
Dedekind-Menge nachweisen. 
\begin{enumerate}
\item $O \not= \{\}$, denn es gilt $-1 \in O$.
\item $O \not= \mathbb{Q}$, denn es gilt $1 \not\in O$.
\item Die Menge $O$ ist nach unten abgeschlossen.

      Sei $x \in O$ und $y < x$.  Nach Definition von $O$ haben wir
      $x < 0$ und aus $y < x$ und $x < 0$ folgt $y < 0$, also gilt nach Definition der Menge $O$ 
      auch $y \in O$.
\item Die Menge $O$ enth\"alt kein Maximum, denn falls $m$ das Maximum der Menge $O$ w\"are,
      dann w\"are $m < 0$ und daraus folgt sofort $\bruch{1}{2} \cdot m < 0$.  Damit w\"are dann nach
      Definition der Menge $O$ auch
      \\[0.2cm]
      \hspace*{1.3cm}
      $\bruch{1}{2} \cdot m \in O$.
      \\[0.2cm]
      Da $m < 0$ ist gilt andererseits aber
      \\[0.2cm]
      \hspace*{1.3cm}
      $m < \bruch{1}{2} \cdot m$
      \\[0.2cm]
      und dann kann $m$ nicht das Maximum der Menge $O$ sein.   Folglich hat die Menge $O$ kein Maximum.
      \qed
\end{enumerate}
\renewcommand{\labelenumi}{\arabic{enumi}.}
\vspace*{-0.3cm}

\exercise 
Zeigen Sie, dass die Menge $O$ das links-neutrale Element bez\"uglich der Addition von
Dedekind-Mengen ist.  \eox


\solution
Wir zeigen, dass
\\[0.2cm]
\hspace*{1.3cm}
$O + A = A$
\\[0.2cm]
gilt.  Wir spalten den Nachweis dieser Mengen-Gleichheit in den Nachweis zweier Inklusionen auf.
\begin{enumerate}
\item ``$\subseteq$'': Es sei $u \in O + A$.  Wir m\"ussen $u \in A$ zeigen.

      Nach Definition von $O + A$ existiert ein $o \in O$
      und ein $a \in A$ mit $u = o + a$.  Aus $o \in O$ folgt $o < 0$.  Also haben wir
      \\[0.2cm]
      \hspace*{1.3cm}
      $u < a$
      \\[0.2cm]
      und da $A$ als Dedekind-Menge nach unten abgeschlossen ist, folgt $u \in A$.
\item ``$\supseteq$'': Sei nun $a \in A$.  Zu zeigen ist $a \in O + A$.

      Da die Menge $A$ eine Dedekind-Menge ist, kann $a$ nicht das Maximum der Menge $A$ sein.
      Folglich gibt es ein $b \in A$, dass gr\"o\ss{}er als $a$ ist, wir haben also
      \\[0.2cm]
      \hspace*{1.3cm}
      $a < b$.
      \\[0.2cm]
      Wir definieren $u := a - b$.  Aus $a < b$ folgt dann $u < 0$ und damit gilt $u \in O$.
      Also haben wir
      \\[0.2cm]
      \hspace*{1.3cm}
      $u + b \in O + A$.
      \\[0.2cm]
      Andererseits gilt
      \\[0.2cm]
      \hspace*{1.3cm}
      $u + b = (a - b) + b = a$,
      \\[0.2cm]
      so dass wir insgesamt $a \in O + A$ gezeigt haben. \qed
\end{enumerate}

\exercise
Zeigen Sie, dass f\"ur die Addition von Dedekind-Mengen sowohl das Kommutativ-Gesetz als auch das
Assoziativ-Gesetz gilt. \eox

Als n\"achstes \"uberlegen wir, wie wir f\"ur eine Dedekind-Menge $A$ das additive Inverse $-\!A$ 
definieren k\"onnen.  Ist $A \in \mathcal{D}$ eine Dedekind-Menge, die wir als die reelle Zahl $x$
interpretieren wollen, so gilt
\\[0.2cm]
\hspace*{1.3cm}
$A = \{ q \in \mathbb{Q} \mid q < x \}$.
\\[0.2cm]
Daher liegt es nahe zu fordern, dass 
\\[0.2cm]
\hspace*{1.3cm}
$-A = \{ q \in \mathbb{Q} \mid q < -x \}$
\\[0.2cm]
gilt.  Wir betrachten zun\"achst den Fall, dass $x \not\in \mathbb{Q}$ ist.
Dann gilt
\\[0.2cm]
\hspace*{1.3cm}
$
\begin{array}{lcll}
q < -x & \Leftrightarrow & \neg (-x \leq q)                                              \\
       & \Leftrightarrow & \neg (-x < q \vee -x = q)                                     \\
       & \Leftrightarrow & \neg (-x < q)             & \mbox{denn $x \not\in\mathbb{Q}$} \\
       & \Leftrightarrow & \neg (x > -q)                                                 \\
       & \Leftrightarrow & \neg (-q < x)                                                 \\
       & \Leftrightarrow & \neg (-q \in A)                                               \\
       & \Leftrightarrow & -q \not\in A.
\end{array}
$
\\[0.2cm]
In diesem Fall k\"onnen wir daher $-A$ als
\\[0.2cm]
\hspace*{1.3cm}
$-A := \{ q \in \mathbb{Q} \mid -q \not\in A \}$
\\[0.2cm]
definieren.  Leider funktioniert diese Definition dann nicht mehr, wenn die Menge 
\\[0.2cm]
\hspace*{1.3cm}
$\{ q \in \mathbb{Q} \mid -q \not\in A \}$
\\[0.2cm]
ihr Maximum enth\"alt.  Dieser Fall w\"urde beispielsweise eintreten, wenn wir
\\[0.2cm]
\hspace*{1.3cm}
$A := \{ q \in \mathbb{Q} \mid q < 1 \}$
\\[0.2cm]  
h\"atten, denn die Bedingung 
\\[0.2cm]
\hspace*{1.3cm}
$\neg (-q < 1)$ \quad ist zu  \quad $q \leq -1$
\\[0.2cm]
\"aquivalent und die Menge
\\[0.2cm]
\hspace*{1.3cm}
$\{ q \in \mathbb{Q} \mid q \leq -1 \}$
\\[0.2cm]
ist keine Dedekind-Menge, da das Maximum dieser Menge $-1$ ist und diese Zahl ist selber ein Element
dieser Menge.  Daher definieren wir zun\"achst f\"ur eine beliebige Menge $A \subseteq \mathbb{Q}$ die
Menge $A^*$, die aus $A$ dadurch entsteht, dass wir das Maximum der Menge $A$ aus $A$ entfernen,
wenn es erstens existiert und zweitens ein Element der Menge $A$ ist.
\\[0.2cm]
\hspace*{1.3cm}
$A^* := \left\{
\begin{array}{ll}
 A - \{ \max(A) \} & \mbox{falls $\max(A)$ existiert und $\max(A) \in A$ ist,} \\
 A                 & \mbox{sonst}.
\end{array}\right.
$
\\[0.2cm]
Damit k\"onnen wir nun $-A$ durch die Festlegung
\\[0.2cm]
\hspace*{1.3cm}
$-\!A := \{ q \in \mathbb{Q} \mid -q \not\in A \}^* = \{ -q \in \mathbb{Q} \mid q \not\in A \}^*$
\\[0.2cm]
definieren.


\begin{Satz}
  Die Menge $-\!A$ ist eine Dedekind-Menge.  \eox
\end{Satz}

\proof
Wir zeigen zun\"achst, dass die Menge
\\[0.2cm]
\hspace*{1.3cm}
 $B := \{ q \in \mathbb{Q} \mid -q \not\in A \}$ 
\\[0.2cm]
nach unten abgeschlossen ist.  Sei also $x \in B$ und $y < x$.  Wir f\"uhren den Nachweis indirekt und
nehmen $y \not\in B$ an.  Aus $y \not\in B$ folgt
\\[0.2cm]
\hspace*{1.3cm}
$-y \in A$.
\\[0.2cm]
Aus $x \in B$ folgt
\\[0.2cm]
\hspace*{1.3cm}
$-x \not\in A$.
\\[0.2cm]
Aus $y < x$ folgt 
\\[0.2cm]
\hspace*{1.3cm}
$-x < -y$
\\[0.2cm]
Da $A$ nach unten abgeschlossen ist, folgt aus $-y \in A$ und $-x < -y$ 
\\[0.2cm]
\hspace*{1.3cm}
$-x \in A$, 
\\[0.2cm]
was im Widerspruch zu $-x \not\in A$ steht.  Also muss $y \in B$ gelten und wir haben gezeigt, das die Menge $B$
tats\"achlich nach unten abgeschlossen ist.
\begin{enumerate}
\item Wir zeigen $-\!A \not= \emptyset$.

      Da $A$ eine Dedekind-Menge ist, ist $A \not= \mathbb{Q}$.  Daher gibt es ein $y \in \mathbb{Q}$ 
      mit $y \not\in A$.  Definieren wir $x := -y$, so gilt
      \\[0.2cm]
      \hspace*{1.3cm}
      $x \in \{ q \in \mathbb{Q} \mid -q \not\in A \} = B$.
      \\[0.2cm]
      Falls $\max(B) \not\in B$ ist, haben wir $-A = B$ und $x \in -A$.

      Sollte $-A = B \backslash \{ \max(B) \}$ und au\ss{}erdem $x = \max(B)$ gelten, so gilt $x-1 < x$ und da die Menge $B$ nach unten
      abgeschlossen ist, ist dann auch $x-1 \in B$.  Au\ss{}erdem ist dann  $x - 1 < \max(B)$ und damit
      gilt $x-1 \in -A$.
\item Wir zeigen $-\!A \not= \mathbb{Q}$.

      Da $A$ eine Dedekind-Menge ist, gilt $A \not= \emptyset$.  Also gibt es ein $y \in A$.
      Wir definieren
      \\[0.2cm]
      \hspace*{1.3cm}
      $q := -y$.
      \\[0.2cm]
      Nach der Definition von $B$ folgt nun, dass $q$ kein Element von $B$ ist.
      Da $-A \subseteq B$ ist, ist $q$ dann sicher auch kein Element von $-A$ und damit 
      gilt $-\!A \not= \mathbb{Q}$.
\item Wir zeigen, dass die Menge $-\!A$ nach unten abgeschlossen ist.
  
      Sei $x \in -A$ und $y < x$.  Wegen $-A \subseteq B$ folgt $x \in B$.
      Wir haben oben schon gesehen, dass die Menge $B$ nach unten abgeschlossen ist.
      Also gilt dann auch $y \in B$ und da $y$ sicher nicht das Maximum von $B$ ist, folgt 
      $y \in -A$.
\item Wir zeigen, dass $-\!A$ kein Maximum hat.

      Hier gibt es zwei F\"alle zu unterscheiden.
      \begin{enumerate}
      \item Die Menge $B$ hat kein Maximum.  Dann gilt $-A = B$ und damit hat $-A$ sicher auch kein
            Maximum. 
      \item Es sei $m := \max(B)$ und es gelte $m \in B$.  Dann haben wir $-A := B \backslash \{ m \}$
            und m\"ussen zeigen, dass auch $-A$ kein Maximum besitzt.   Sei dazu $x \in -A$.
            Dann folgt 
            \\[0.2cm]
            \hspace*{1.3cm}
            $x \in B$ \quad und \quad $x < m$, 
            \\[0.2cm]
            denn $m$ ist ja das Maximum vom $B$.  f\"ur das arithmetische Mittel von $x$ und $m$ gilt
            dann
            \\[0.2cm]
            \hspace*{1.3cm}
            $x < \frac{1}{2} \cdot (x + m) < m$
            \\[0.2cm]
            und da $B$ nach unten abgeschlossen ist und $m \in B$ ist, folgt zun\"achst $\frac{1}{2} \cdot (x + m) \in
            B$ und dann auch
            \\[0.2cm]
            \hspace*{1.3cm}
            $\frac{1}{2} \cdot (x + m) \in -A$,
            \\[0.2cm]  
            denn die Mengen $-A$ und $B$ unterscheiden sich ja nur um $m$.  Insgesamt haben wir
            jetzt zu beliebigem $x \in -A$ ein $y := \frac{1}{2} \cdot (x + m)$ gefunden, f\"ur das 
            $x < y$ und $y \in -A$ gilt.  Damit kann die Menge $-A$ kein Maximum haben. \qed
      \end{enumerate}

\end{enumerate}

\begin{Satz}
  f\"ur jede Dedekind-Menge $A$ gilt die Gleichung
  \\[0.2cm]
  \hspace*{1.3cm}
  $A + (-A) = O$. \eox
\end{Satz}

\proof
  Wir spalten den Nachweis dieser Mengengleicheit in zwei Teile auf.
\begin{enumerate}
\item ``$\subseteq$'':  Es sei $x + y \in A + -\!A$, also $x \in A$ und $y \in -A$.
      Wir haben zu zeigen, dass $x + y \in O$ ist und das ist gleichbedeutend mit $x + y < 0$.

      Wegen $y \in -\!A$ gilt nach Definition der Menge $-\!A$
      \\[0.2cm]
      \hspace*{1.3cm}
      $-y \not \in A$.
      \\[0.2cm]
      w\"are $-y \leq x$, so w\"urde aus der Tatsache, dass $A$ nach unten abgeschlossen und $x \in A$
      ist, sofort $-y \in A$ folgen, was nicht sein kann.  Also gilt
      \\[0.2cm]
      \hspace*{1.3cm}
      $-y > x$
      \\[0.2cm]
      und daraus folgt $0 > x + y$, was zu zeigen war.
\item ``$\supseteq$'': Es sei nun $o \in O$, also $o < 0$.  Wir m\"ussen ein $x \in A$ und ein $y \in -A$
      finden, so dass $o = x + y$ gilt, denn dann haben wir $o \in A + (-A)$ gezeigt.

      Wir definieren 
      \\[0.2cm]
      \hspace*{1.3cm}
      $r := -\frac{1}{2} \cdot o$.
      \\[0.2cm]
      Da $o < 0$ ist, folgt $r > 0$ und au\ss{}erdem gilt $r \in \mathbb{Q}$.  Wir definieren die Menge
      $M$ als
      \\[0.2cm]
      \hspace*{1.3cm}
      $M := \{ n \in \mathbb{Z} \mid n \cdot r \in  A \}$
      \\[0.2cm]
      Da $A \not= \mathbb{Q}$ ist, gibt es ein $z \in \mathbb{Q}$ so dass $z \not\in A$ ist.
      f\"ur die Zahlen $n \in \mathbb{Z}$, f\"ur die $n \cdot r > z$ ist, folgt dann $n \not\in M$.
      Folglich ist die Menge $M$ nach oben beschr\"ankt.
      Da die Menge $A$ nach unten abgeschlossen ist, ist $M$ sicher nicht leer.
      Als beschr\"ankte und nicht-leere Menge von nat\"urlichen Zahlen hat $M$ ein Maximum.  Wir definieren 
      \\[0.2cm]
      \hspace*{1.3cm}
      $\widehat{n} := \max(M)$.
      \\[0.2cm]
      Dann gilt $\widehat{n} + 1 \not\in M$, also 
      \\[0.2cm]
      \hspace*{1.3cm}
      $(\widehat{n} + 1) \cdot r \not\in A$.  
      \\[0.2cm]
      Wir  definieren  jetzt
      \\[0.2cm]
      \hspace*{1.3cm}
      $x := \widehat{n} \cdot r$ \quad und \quad $y := -(\widehat{n} + 2) \cdot r$.  
      \\[0.2cm]
      Nach  Definition von $\widehat{n}$ und $M$ gilt dann $x \in A$ und aus 
      $(\widehat{n} + 1) \cdot r \not\in A$ folgt
      \\[0.2cm]
      \hspace*{1.3cm}
      $-y - r = (\widehat{n} + 2) \cdot r - r = (\widehat{n} + 1) \cdot r \not\in A$,
      \\[0.2cm]
      so dass $y + r\in B := \{ q \in \mathbb{Q} \mid -q \not\in A \}$ ist.  
      Wegen $-A := B^*$ k\"onnte es sein, dass $y+r \not\in -A$ ist.  Dies w\"are dann der Fall,
      wenn $y+r = \max(B)$ w\"are.  Da $B$ nach unten abgeschlossen, gilt aber sicher $y \in B$
      und da $y < y + r$ ist, haben wir $y \not= \max(B)$.  Also gilt  $y \in -A$.  Damit haben wir $x + y \in A + (-A)$.
      Au\ss{}erdem gilt
      \\[0.2cm]
      \hspace*{1.3cm}
      $x + y = \widehat{n} \cdot r - (\widehat{n} + 2) \cdot r = - 2 \cdot r = o$, \quad also \quad $o \in A + (-A)$.
      \\[0.2cm]
      Damit haben wir $O \subseteq A + -\!A$ gezeigt. 
      \qed
\end{enumerate}

Wir \"uberlegen uns nun, wie sich auf der Menge $\mathcal{D}$ eine Multiplikation so definieren l\"asst,
so dass $\mathcal{D}$ mit dieser Multiplikation und der oben definierten Addition ein K\"orper wird.
Dazu nennen wir eine Dedekind-Menge $A$ positiv, wenn $0 \in A$ gilt.  f\"ur zwei positive Dedekind-Mengen $A$
und $B$ l\"asst sich die Multiplikation $A \cdot B$ als
\\[0.2cm]
\hspace*{1.3cm}
$A \cdot B := \{ x \cdot y \mid x \in A \wedge y \in B \wedge x > 0 \wedge y > 0 \} \cup 
              \{ z \in \mathbb{Q} \mid z \leq 0 \}$
\\[0.2cm]
definieren.  Wir zeigen, dass die so definerte Menge $A \cdot B$ eine Dedekind-Menge ist.
Dazu weisen wir die einzelnen Eigenschaften getrennt nach.
\begin{enumerate}
\item Wir zeigen $A \cdot B \not= \{\}$.

      Nach Definition von $A \cdot B$ gilt $0 = 0 \cdot 0 \in A \cdot B$.   Daraus folgt sofort $A \cdot B \not= \{\}$.
\item Wir zeigen $A \cdot B \not= \mathbb{Q}$.

      Da $A$ und $B$ als Dedekind-Mengen von der Menge $\mathbb{Q}$ verschieden sind, gibt es 
      $u,v \in \mathbb{Q}$ mit $u \not\in A$ und $v \not\in B$. Wir definieren $w := \max(u, v)$.  Dann gilt 
      \\[0.2cm]
      \hspace*{1.3cm}
      $(\forall x \in A: x < w) \wedge (\forall y \in B: y < w)$
      \\[0.2cm]
      Daraus folgt sofort, dass f\"ur alle $x \in A$ und $y \in B$ die Ungleichung
      \\[0.2cm]
      \hspace*{1.3cm}
      $x \cdot y < w \cdot w$
      \\[0.2cm]
      gilt.  Das hei\ss{}t aber $w^2 \not\in A \cdot B$ und damit ist $A \cdot B \not= \mathbb{Q}$.
\item Wir zeigen, dass $A \cdot B$ nach unten abgeschlossen ist.

      Es sei $x \cdot y \in A \cdot B$ und $z \in \mathbb{Q}$ mit $z < x \cdot y$.  
      Wir m\"ussen $z \in A \cdot B$ zeigen.  
      Wir f\"uhren eine Fall-Unterscheidung danach durch, ob $z > 0$ ist.
      \begin{enumerate}
      \item Fall: $z > 0$.  Dann definieren wir
            \\[0.2cm]
            \hspace*{1.3cm}
            $\alpha := \bruch{z}{x \cdot y}$
            \\[0.2cm]
            Aus $z < x \cdot y$ folgt $\alpha < 1$.  Wir setzen $u := \alpha \cdot x$
            und folglich gilt $u < x$.  Da $A$ nach unten abgeschlossen ist, folgt $u \in A$.
            Damit haben wir insgesamt $u \cdot y \in A \cdot B$.  Es gilt aber
            \\[0.2cm]
            \hspace*{1.3cm}
            $u \cdot y = \alpha \cdot x \cdot y = \bruch{z}{x \cdot y} \cdot x \cdot y = z$,
            \\[0.2cm]
            so dass wir insgesamt $z \in A \cdot B$ gezeigt haben.
      \item Fall: $z \leq 0$.  Dann folgt unmittelbar aus der Definition von $A \cdot B$, dass 
            $z \in A \cdot B$  ist.
      \end{enumerate}
\item Wir zeigen, dass $A \cdot B$ kein Maximum hat.  

      Wir f\"uhren den Nachweis indirekt und nehmen an, dass die Menge $A \cdot B$ eine Maximum $c$ hat.
      Es gilt dann 
      \\[0.2cm]
      \hspace*{1.3cm}
      $c \in A \cdot B$ \quad und \quad $\forall z \in A \cdot B: z \leq c$.
      \\[0.2cm]
      Nach Definition von $A \cdot B$ gibt es dann ein $a \in A$ und ein $b \in B$ mit $c = a \cdot b$.
      Wir zeigen, dass $a$ das Maximum der Menge $A$ ist.  Sei also $u \in A$.  Dann gilt
      \\[0.2cm]
      \hspace*{1.3cm}
      $u \cdot b \in A \cdot B$ \quad und folglich gilt \quad $u \cdot b \leq c = a \cdot b$.
      \\[0.2cm]
      Teilen wir die letzte Ungleichung durch $b$, so folgt
      \\[0.2cm]
      \hspace*{1.3cm}
      $u \leq a$
      \\[0.2cm]
      und damit w\"are $a$ das Maximum der Menge $A$.  Das ist eine Widerspruch zu der Tatsache, dass
      $A$ eine Dedekind-Menge ist.
\end{enumerate}
\renewcommand{\labelenumi}{\arabic{enumi}.}
Bisher haben wir das Produkt $A \cdot B$ nur f\"ur den Fall definiert, dass $A$ und $B$ beide positiv
sind.  Falls $A$ oder $B$ gleich $O$ ist, definieren wir das Produkt wie folgt:
\\[0.2cm]
\hspace*{1.3cm}
$A \cdot O := O \cdot B := O$.
\\[0.2cm]
Falls $A$ weder positiv noch gleich $O$ ist, sagen wir, dass $A$  \emph{negativ} ist.  In einem
solchen Fall ist $-\!A$ positiv.  Falls $A$ oder $B$ negativ ist, lautet die Definition wie folgt:
\begin{enumerate}
\item[2.] Fall: $A$ ist positiv, aber $B$ negativ.  Dann ist $-\!B$ positiv und wir k\"onnen
          \\[0.2cm]
          \hspace*{1.3cm}
          $A \cdot B := -\!\bigl(A \cdot (-\!B)\bigr)$
          \\[0.2cm]
          definieren.
\item[3.] Fall: $B$ ist positiv, aber $A$ ist negativ.  Dann setzen wir
          \\[0.2cm]
          \hspace*{1.3cm}
          $A \cdot B := -\!\bigl((-\!A) \cdot B\bigr)$.
\item[4.] Fall: $A$ und $B$ sind negativ.  Wir definieren
          \\[0.2cm]
          \hspace*{1.3cm}
          $A \cdot B := (-\!A) \cdot (-\!B)$.
\end{enumerate}
Nun m\"ussten wir noch nachweisen, dass f\"ur die so definierte Multiplikation zusammen mit der oben definierten
Addition die K\"orper-Axiome gelten.  Bei diesen Beweisen tauchen keine Ideen auf, die wir nicht schon gesehen haben.
Daher verzichten wir aus Zeitgr\"unden darauf, diese Beweise im Detail auszuf\"uhren und verweisen statt dessen
auf die unten angegebene Literatur.
\vspace*{0.3cm}

\noindent
\textbf{Literatur-Hinweise} \\
In dem Buch 
\href{http://www.fernuni-hagen.de/imperia/md/content/ausstellung/juedischemathematiker/landau.pdf}{Grundlagen der Analysis} 
von Edmund Landau \cite{landau:1930} wird die oben skizzierte Konstruktion der reellen Zahlen im Detail beschrieben.   
Auch das Buch ``\emph{Principles of  Mathematical Analysis}'' von Walter Rudin \cite{rudin:1976}
diskutiert die Konstruktion der reellen Zahlen mit Hilfe von Dedekind-Mengen ausf\"uhrlicher als dies in
dem zeitlichen Rahmen meiner Vorlesung m\"oglich ist.


\section{Geschichte}
Die Konstruktion der reellen
Zahlen mit Hilfe von Schnitten geht auf Richard Dedekind zur\"uck, der die nach ihm benannten Schnitte in dem Buch
\href{http://books.google.de/books?id=n-43AAAAMAAJ&printsec=frontcover&source=gbs_ge_summary_r&cad=0#v=onepage&q&f=false}{Stetigkeit und irrationale Zahlen} 
\cite{dedekind:1872}, das im Jahre 1872 erschienen ist, beschrieben hat.  Damit war erstmals eine
formale Definition des Begriffs der reellen Zahlen gefunden worden.  Diese Definition war eine der wichtigsten
Fortschritte im Bereich der mathematischen Grundlagenforschung des 19.~Jahrhunderts, denn sie
erm\"oglichte es, die Analysis auf ein solides Fundament zu stellen.

%%% Local Variables: 
%%% mode: latex
%%% TeX-master: "analysis"
%%% End: 

