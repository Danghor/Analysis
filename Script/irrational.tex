\chapter[Die Zahlen $\pi$ und $e$ sind irrational]{Die Kreiszahl $\pi$ und die Euler'sche Zahl $e$ sind irrational$^*$}
In diesem Kapitel zeigen wir, dass sowohl die Kreiszahl $\pi$, die als Fl\"ache eines Kreises mit dem Radius 1
definiert ist, als auch die Euler'sche Zahl $e$, die als Grenzwert der Reihe
\\[0.2cm]
\hspace*{1.3cm}
$\exp(1) := \sum\limits_{k=0}^\infty \bruch{1}{k!}$
\\[0.2cm] 
festgelegt ist, irrational sind.  Da der Nachweis der Irrationalit\"at von $e$ einfacher ist, beginnen wir
damit.

\section{Die Euler'sche Zahl $e$ ist irrational}
Nach Definition von $e$ gilt
\\[0.2cm]
\hspace*{1.3cm}
$e = \sum\limits_{k=0}^\infty \bruch{1}{k!}$.
\\[0.2cm]
F\"ur alle $n \in \mathbb{N}$ definieren wir die $n$-te Partial-Summe $s_n$ als
\\[0.2cm]
\hspace*{1.3cm}
$s_n := \sum\limits_{k=0}^n \bruch{1}{k!}$.
\\[0.2cm]
Als n\"achstes definieren wir f\"ur alle nat\"urlichen Zahlen $n \in \mathbb{N}$ den $n$-ten Rest
\\[0.2cm]
\hspace*{1.3cm}
$r_n := e - s_n = \sum\limits_{k=0}^\infty \bruch{1}{k!} - \sum\limits_{k=0}^n \bruch{1}{k!} 
                = \sum\limits_{k=n+1}^\infty \bruch{1}{k!}
$
\\[0.2cm]
Offenbar gilt
\begin{equation}
  \label{eq:e_irrational1}
  0 < r_n,
\end{equation}
denn der $n$-te Rest $r_n$ enth\"alt auf jeden Fall den Term $\frac{1}{(n+1)!}$ und der ist positiv.  
Wir wollen nun den $n$-ten Rest $r_n$ nach oben hin absch\"atzen.  Dazu ben\"otigen wir zun\"achst die folgende
Absch\"atzung, die f\"ur alle $k > n + 1$ g\"ultig ist:
\\[0.2cm]
\hspace*{1.3cm}
$
\begin{array}[t]{llcl}
                & \bruch{(n+1)!}{k!} & < & \bruch{1}{(n+1)^{k-(n+1)}}  \\[0.4cm]
\Leftrightarrow & \bruch{k!}{(n+1)!} & > & (n+1)^{k-(n+1)}             \\[0.4cm]
\Leftrightarrow & \underbrace{(n+2) \cdot (n+3) \cdot {\dots} \cdot (k-1) \cdot k}_{\mbox{$k-(n+1)$ Faktoren}} & > & 
                  \underbrace{(n+1) \cdot {\dots} \cdot (n+1)}_{\mbox{$k-(n+1)$ Faktoren}} \\[0.4cm]
\end{array}
$
\\[0.2cm]
Die letzte Ungleichung ist richtig, denn die Faktoren auf der linken Seite haben die Form
$(n + 1) + i$ mit $i \in \{1, \cdots, k-(n+1)\}$ und offenbar gilt
\\[0.2cm]
\hspace*{1.3cm}
$(n+1) + i > n+1$ \quad f\"ur $i \in \{1, \cdots, k-(n+1)\}$,
\\[0.2cm]
so dass zu jedem Faktor in dem Produkt auf der linken Seite ein kleinerer Faktor auf der rechten Seite
korrespondiert.  Falls $k = (n+1)$ ist, gilt
\\[0.2cm]
\hspace*{1.3cm}
$\bruch{(n+1)!}{k!} = \bruch{1}{(n+1)^{k-(n+1)}} = 1$,
\\[0.2cm]
was man unmittelbar durch Einsetzen best\"atigen kann.  Wir haben also insgesamt Folgendes gezeigt:
\begin{equation}
  \label{eq:e_irrational2}
 \bruch{(n+1)!}{k!} < \bruch{1}{(n+1)^{k-(n+1)}}  \quad \mbox{f\"ur alle $k > n+1$}
\end{equation}
und f\"ur $k = n + 1$ haben wir die Gleichheit beider Seiten.
Nun gilt f\"ur alle $n \in \mathbb{N}$ mit $n \geq 1$ die folgende Ungleichungs-Kette:
\\[0.2cm]
\hspace*{1.3cm}
$
\begin{array}[t]{lcll}
r_n & = & \sum\limits_{k=n+1}^\infty \bruch{1}{k!}                                      \\[0.2cm]
    & = & \bruch{1}{(n+1)!} \cdot \sum\limits_{k=n+1}^\infty \bruch{(n+1)!}{k!}         \\[0.5cm]
    & < & \bruch{1}{(n+1)!} \cdot \sum\limits_{k=n+1}^\infty \bruch{1}{(n+1)^{k-(n+1)}} 
        & (\mbox{nach Gleichung (\ref{eq:e_irrational2})})                                \\[0.5cm]
    & = & \bruch{1}{(n+1)!} \cdot \sum\limits_{k=0}^\infty \bruch{1}{(n+1)^{k}}    
        & (\mbox{Index-Verschiebung})                                                     \\[0.5cm]
    & = & \bruch{1}{(n+1)!} \cdot \bruch{1}{1 - \frac{1}{n+1}}                    
        & (\mbox{geometrische Reihe})                                                    \\[0.8cm]
    & = & \bruch{1}{(n+1)!} \cdot \bruch{1}{\frac{n + 1 - 1}{n+1}}                    
        & (\mbox{Hauptnenner})                                                            \\[0.8cm]
    & = & \bruch{1}{(n+1)!} \cdot \bruch{n+1}{n}                    
        &                                                                               \\[0.5cm]
    & = & \bruch{1}{n! \cdot n}                     
        &                                                                               \\[0.5cm]
\end{array}
$
\\[0.2cm]
Multiplizieren wir die resultierende Ungleichung mit $n!$, so sehen wir, dass
\\[0.2cm]
\hspace*{1.3cm}
$n! \cdot r_n < \bruch{1}{n}$ 
\\[0.2cm]
gilt.
Fassen wir diese Gleichung zusammen mit der Gleichung (\ref{eq:e_irrational1}), so haben wir insgesamt
\begin{equation}
  \label{eq:e_irrational3}
  0 < n! \cdot r_n < \bruch{1}{n} \quad \mbox{falls $n \geq 1$ ist.}
\end{equation}
Damit ist aber klar, dass der Ausdruck $n! \cdot r_n$ f\"ur $n \geq 1$ keine nat\"urliche Zahl sein kann.

\begin{Theorem}
  Die Eulersche Zahl $e$ ist irrational.
\end{Theorem}

\proof
Wir nehmen an, dass $e$ rational ist.  Dann gibt es nat\"urliche Zahlen $p,q \in \mathbb{N}$ mit
$q \geq 1$ und
\\[0.2cm]
\hspace*{1.3cm}
$e = \bruch{p}{q}$.
\\[0.2cm]
Wir betrachten den Ausdruck $q! \cdot r_q = q! \cdot (e - s_q)$ und setzten dort f\"ur $e$ den Wert $\ds\frac{p}{q}$ ein:
\\[0.2cm]
\hspace*{1.3cm}
$q! \cdot r_q = q! \cdot \Bigl(\bruch{p}{q} - \sum\limits_{k=0}^q \bruch{1}{k!}\Bigr)  
              = \displaystyle (q-1)! \cdot p - \sum\limits_{k=0}^q \bruch{q!}{k!} \in \mathbb{Z}
$,
\\[0.2cm]
denn $(q-1)! \cdot p$ ist auf jeden Fall eine nat\"urliche Zahl und f\"ur $k \leq q$ hat der Ausdruck
$\frac{q!}{k!}$ die Form
\\[0.2cm]
\hspace*{1.3cm}
$\bruch{q!}{k!} = \bruch{1 \cdot 2 \cdot {\dots} \cdot k \cdot (k + 1) \cdot (k + 2) \cdot {\dots} \cdot (q-1) \cdot q}{1 \cdot
  2 \cdot {\dots} \cdot k} 
   = (k + 1) \cdot (k + 2) \cdot {\dots} \cdot (q-1) \cdot q$
\\[0.2cm]
und das ist ebenfalls eine nat\"urliche Zahl.  Damit haben wir aber einen Widerspruch, denn die Aussagen
\\[0.2cm]
\hspace*{1.3cm}
$\ds 0 < q! \cdot r_q < \frac{1}{q}$  \quad und \quad $q! \cdot r_q \in \mathbb{Z}$ 
\\[0.2cm]
sind unvereinbar.  \qed


\section{Die Kreiszahl $\pi$ ist irrational$^*$}
Zur Vorbereitung des Beweises ben\"otigen wir das folgende Lemma.
\begin{Lemma}
Die Funktion $f:\mathbb{R} \rightarrow \mathbb{R}$ sei beliebig oft differenzierbar. Dann gilt 
f\"ur alle nat\"urlichen Zahlen $n$
\\[0.2cm]
\hspace*{0.8cm}
$\dint{0}{{\pi}} f(x) \cdot \sin(x)\, dx = 
  \sum\limits_{k=0}^n (-1)^k \cdot \bigl(f^{(2k)}(\pi) + f^{(2k)}(0)\bigr) + 
  (-1)^{n+1} \cdot \dint{0}{\pi} f^{(2n+2)}(x) \cdot \sin(x)\, dx
$.
\end{Lemma}

\proof
Zur Abk\"urzung definieren wir
\\[0.2cm]
\hspace*{1.3cm}
$I := \dint{0}{{\pi}} f(x) \cdot \sin(x)\, dx$ 
\\
und 
\\
\hspace*{1.3cm}
$\ds S_n := \sum\limits_{k=0}^n (-1)^k \cdot \bigl(f^{(2k)}(\pi) + f^{(2k)}(0)\bigr) + 
  (-1)^{n+1} \cdot \dint{0}{\pi} f^{(2n+2)}(x) \cdot \sin(x)\, dx
$.
\\[0.2cm]
Die Behauptung
\\[0.2cm]
\hspace*{1.3cm}
$I = S_n$
\\[0.2cm]
wird nun durch Induktion nach $n$ bewiesen.  Dabei werden  wir sowohl im Induktions-Anfang als auch
im Induktions-Schritt zwei partielle Integrationen durchf\"uhren.  
\begin{enumerate}
\item[I.A.:] $n=0$.
     
     Nach Definition von $I$ gilt
     \\[0.2cm]
     \hspace*{1.3cm}
     $I = \dint{0}{\pi} f(x) \cdot \sin(x)\, dx$
     \\[0.2cm]
     Wir integrieren partiell und setzen $u(x) = f(x)$ und $v'(x) = \sin(x)$.
     Dann gilt $u'(x) = f'(x)$ und $v(x) = -\cos(x)$.  Also haben wir
     \\[0.2cm]
     \hspace*{1.3cm}
     $I = - f(x) \cdot \cos(x) \Bigr|_0^\pi + \dint{0}{\pi} f'(x) \cdot \cos(x)$.
     \\[0.2cm]
     F\"ur den ersten Summanden auf der rechten Seite dieser Gleichung finden wir
     \\[0.2cm]
     \hspace*{1.3cm}
     $  - f(x) \cdot \cos(x) \Bigr|_0^\pi = - f(\pi) \cdot \cos(\pi) + f(0) \cdot \cos(0)
      = f(\pi) + f(0)$.
     \\[0.2cm]
     Um das verbleibende Integral zu berechnen f\"uhren wir eine erneute partielle Integration durch, bei
     der wir diesmal $u(x) = f'(x)$ und $v'(x) = \cos(x)$ setzen.  Dann gilt $u'(x) = f^{(2)}(x)$ und
     $v(x) = \sin(x)$.  Wegen 
     \\[0.2cm]
     \hspace*{1.3cm}
     $\sin(\pi) = \sin(0) = 0$
     \\[0.2cm]
     finden wir damit f\"ur das Integral $I$ den Wert
     \\[0.2cm]
     \hspace*{1.3cm}
     $I = f(\pi) + f(0) - \dint{0}{\pi} f^{(2)}(x) \cdot \sin(x)\, dx$.
     \\[0.2cm]
     Auf der anderen Seite haben wir
     \\[0.2cm]
     \hspace*{1.3cm}
     $
     \begin{array}[t]{lcl}
       S_0 & = & \ds \sum\limits_{k=0}^0 (-1)^k \cdot \bigl(f^{(2k)}(\pi) + f^{(2k)}(0)\bigr) + 
                  (-1)^{0+1} \cdot \dint{0}{\pi} f^{(2\cdot 0+2)}(x) \cdot \sin(x)\, dx           \\[0.5cm]
           & = & (-1)^0 \cdot \bigl(f^{(0)}(\pi) + f^{(0)}(0)\bigr)  
                  - \dint{0}{\pi} f^{(2)}(x) \cdot \sin(x)\, dx           \\[0.5cm]
           & = & f(\pi) + f(0) - \dint{0}{\pi} f^{(2)}(x) \cdot \sin(x)\, dx           \\[0.2cm]
           & = & I.
     \end{array}
     $
\item[I.S.:] $n \mapsto n + 1$
     
     Zur Abk\"urzung definieren wir
     \\[0.2cm]
     \hspace*{1.3cm}
     $J_n = \dint{0}{\pi} f^{(2n+2)}(x) \cdot \sin(x)\, dx$
     \\[0.2cm]
     Wir berechnen $J_n$ durch partielle Integration und setzen $u(x) := f^{(2n+2)}(x)$ und
     $v'(x) := \sin(x)$.  Dann haben wir $u'(x) = f^{(2n+3)}(x)$ und $v(x) = -\cos(x)$.
     Das liefert
     \\[0.2cm]
     \hspace*{1.3cm}
     $J_n = - f^{(2n+2)}(x) \cdot \cos(x)\Bigr|_0^\pi + \dint{0}{\pi} f^{(2n+3)}(x) \cdot \cos(x)\, dx$.
     \\[0.2cm]
     Wegen $\cos(\pi) = -1$ und $\cos(0) = 1$ vereinfacht sich der erste Summand auf der rechten Seite
     wie folgt:
     \\[0.2cm]
     \hspace*{1.3cm}
     $- f^{(2n+2)}(x) \cdot \cos(x)\Bigr|_0^\pi =  f^{(2n+2)}(\pi) + f^{(2n+2)}(0)$.
     \\[0.2cm]
     Das auf der rechten Seite der obigen Gleichung verbleibende Integral berechnen wir durch eine
     weitere partielle Integration.  Diesmal setzen wir $u(x) = f^{(2n+3)}(x)$ und 
     $v'(x) = \cos(x)$.  Dann haben wir $u'(x) = f^{(2n+3)}(x)$ und $v(x) = \sin(x)$ und f\"ur das Integral
     finden wir
     \\[0.2cm]
     \hspace*{1.3cm}
     $
     \begin{array}[t]{lcl}
            \dint{0}{\pi} f^{(2n+3)}(x) \cdot \cos(x)\, dx  
      & = & f^{(2n+3)}(x) \cdot \sin(x)\Bigr|_0^\pi - \dint{0}{\pi} f^{(2n+4)}(x) \cdot \sin(x)\, dx  \\[0.4cm]
      & = &  - \dint{0}{\pi} f^{(2n+4)}(x) \cdot \sin(x)\, dx  \\[0.4cm]
     \end{array}
     $
     \\[0.2cm]
     Insgesamt haben wir damit f\"ur $J_n$ den Ausdruck
     \\[0.2cm]
     \hspace*{1.3cm}
     $
     \begin{array}[t]{lcl}
       J_n & = & f^{(2n+2)}(\pi) + f^{(2n+2)}(0) - \dint{0}{\pi} f^{(2n+4)}(x) \cdot \sin(x)\, dx \\[0.4cm]
           & = & f^{(2n+2)}(\pi) + f^{(2n+2)}(0) - J_{n+1}
     \end{array}
     $
     \\[0.2cm]
     gefunden.  Jetzt rechnen wir wie folgt:
     \\[0.2cm]
     \hspace*{0.0cm}
     $
     \begin{array}[t]{lcl}
      I & \stackrel{IV}{=} & \ds
          \sum\limits_{k=0}^n (-1)^k \cdot \bigl(f^{(2k)}(\pi) + f^{(2k)}(0)\bigr) + (-1)^{n+1} \cdot J_n 
             \\[0.4cm]
        & = & \ds
         \sum\limits_{k=0}^n (-1)^k \cdot \bigl(f^{(2k)}(\pi) + f^{(2k)}(0)\bigr) + (-1)^{n+1} \cdot \Bigl( f^{(2n+2)}(\pi) + f^{(2n+2)}(0) - J_{n+1} \Bigr)
            \\[0.4cm]
        & = & \ds
            \sum\limits_{k=0}^{n+1} (-1)^k \cdot \bigl(f^{(2k)}(\pi) + f^{(2k)}(0)\bigr) + (-1)^{n+2} \cdot J_{n+1} \; = \;S_{n+1} 
     \end{array}
     $
     \\[0.2cm]
     Damit haben wir gezeigt, dass $I = S_{n+1}$ gilt und die Induktion ist abgeschlossen. \qed
\end{enumerate}


\begin{Theorem}
  Die Kreiszahl $\pi$ ist irrational.
\end{Theorem}

\proof
Wir f\"uhren den Beweis indirekt und nehmen an, dass $\pi \in \mathbb{Q}$ ist.  Dann gibt es Zahlen
$p,q \in \mathbb{N}$ mit 
\\[0.2cm]
\hspace*{1.3cm}
$\pi = \bruch{p}{q}$.
\\[0.2cm]
F\"ur beliebige $n \in \mathbb{N}$ definieren wir das Polynom $g_n(x)$ wie folgt:
\\[0.2cm]
\hspace*{1.3cm}
$g_n(x) := \bruch{1}{n!} \cdot x^n \cdot (p - q \cdot x)^n$.
\\[0.2cm]
Es gilt
\\
\hspace*{1.3cm}
$
\begin{array}[t]{lcl}
  g_n(\pi - x) 
& = & 
 \bruch{1}{n!} \cdot \left(\bruch{p}{q} - x\right)^n \cdot 
  \left(p - q \cdot \Bigl(\bruch{p}{q} - x\Bigr)\right)^n    \\[0.4cm]
& = & 
 \bruch{1}{n!} \cdot \left(\bruch{p}{q} - x\right)^n \cdot (q \cdot x)^n    \\[0.4cm]
& = & 
 \bruch{1}{n!} \cdot \left(p - q \cdot x\right)^n \cdot x^n    \\[0.4cm]
& = & 
 g_n(x)   
\end{array}
$
\\[0.2cm]
Diese Gleichung \"ubertr\"agt sich nat\"urlich auf die Ableitungen und daher haben wir
\\[0.2cm]
\hspace*{1.3cm}
$g_n^{(k)}(\pi - x) = (-1)^k \cdot g_n^{(k)}(x)$.
\\[0.2cm]
Offenbar ist $g_n$ ein Polynom vom Grad $2 \cdot n$ und damit ist klar, dass $g_n^{(2n+2)}(x) = 0$ ist.
Setzen wir in der Behauptung des letzten Lemmas f\"ur $f$ die Funktion $g_n(x)$ ein, so folgt daher
\\[0.2cm]
\hspace*{1.3cm}
$\dint{0}{{\pi}} g_n(x) \cdot \sin(x)\, dx = 
  \sum\limits_{k=0}^n (-1)^k \cdot \Bigl(g_n^{(2k)}(\pi) + g_n^{(2k)}(0)\Bigr) 
$.
\\[0.2cm]
Wir zeigen, dass alle Summanden in der Summe auf der rechten Seite dieser Gleichung ganze Zahlen sind.
Dabei reicht es aus, dies f\"ur die Summanden der Form $g_n^{(2k)}(0)$ zu zeigen, denn es gilt
\\[0.2cm]
\hspace*{1.3cm}
$g_n^{(2k)}(\pi) = (-1)^{2\cdot k} \cdot g_n^{(2k)}(\pi - \pi) = g_n^{(2k)}(0)$.
\\[0.2cm]
Wir zeigen mit Hilfe einer Fallunterscheidung, dass    
$g_n^{(k)}(0) \in \mathbb{N}$ f\"ur alle $k \in \mathbb{N}$ gilt.
\begin{enumerate}
\item Fall: $k < n$.

      Da das Polynom $g_n(x)$ die Form
      \\[0.2cm]
      \hspace*{1.3cm}
      $g_n(x) = \bruch{1}{n!} \cdot \sum\limits_{i=n}^{2\cdot n} c_i \cdot x^i$
      \\[0.2cm]
      mit Koeffizienten $c_i \in \mathbb{Z}$ hat, folgt, dass f\"ur $k < n$ 
      \\[0.2cm]
      \hspace*{1.3cm}
      $g_n^{(k)}(x) = \bruch{1}{n!} \cdot \displaystyle\sum\limits_{i=n}^{2\cdot n} 
             \bruch{i!}{(i- k)!} \cdot c_i \cdot x^{i- k}$
      \\[0.2cm]
      gilt.  Jeder Term dieser Summe enth\"alt mindestens den Faktor $x$.
      Setzen wir hier f\"ur $x$ den Wert $0$ ein, so wird daher jeder Term in der Summe $0$.  Damit
      gilt 
      \\[0.2cm]
      \hspace*{1.3cm}
      $g_n^{(k)}(0) = 0 \in \mathbb{N}$.
\item Fall: $k \geq n$.

      Diesmal verschwinden beim Ableiten alle Summanden mit Index $i < k$.  Wir haben also
      \\[0.2cm]
      \hspace*{1.3cm}
      $
      \begin{array}[t]{lcl}
        g_n^{(k)}(x) & = &
        \bruch{1}{n!} \cdot \displaystyle\sum\limits_{i=k}^{2\cdot n} 
        \bruch{i!}{(i-k)!} \cdot c_i \cdot x^{i-k}                     \\[0.5cm]
        & = &
        \displaystyle\sum\limits_{i=k}^{2\cdot n} 
        \bruch{k!}{n!} \cdot \bruch{i!}{k! \cdot (i-k)!} \cdot c_i \cdot x^{i-k}                     \\[0.5cm]
        & = &
        \displaystyle\sum\limits_{i=k}^{2\cdot n} 
        \bruch{k!}{n!} \cdot {i \choose k} \cdot c_i \cdot x^{i-k}                     \\[0.5cm]
      \end{array}
      $
      \\[0.2cm]
      F\"ur $x = 0$ folgt dann
      \\[0.2cm]
      \hspace*{1.3cm}
      $g_n^{(k)}(0) = \bruch{k!}{n!} \cdot {k \choose k} \cdot c_k = \bruch{k!}{n!} \cdot c_k \in \mathbb{Z}$,
      \\[0.2cm]
      denn wenn $k \geq n$ ist, ist $\bruch{k!}{n!}$ eine nat\"urliche Zahl.
\end{enumerate}
Insgesamt wissen wir jetzt, dass das Integral
\\[0.2cm]
\hspace*{1.3cm}
$I_n := \dint{0}{{\pi}} g_n(x) \cdot \sin(x)\, dx$
\\[0.2cm]
f\"ur alle $n \in \mathbb{N}$ eine ganze Zahl ist.  F\"ur alle $x \in [0, \pi]$ gilt nun
\\[0.2cm]
\hspace*{1.3cm}
$0 \leq \sin(x)$ \quad und \quad $0 \leq g_n(x)$.
\\[0.2cm]
Also muss auch
\\[0.2cm]
\hspace*{1.3cm}
$0 < I_n$
\\[0.2cm]
gelten.  Die Ungleichung ist echt, denn die beiden Funktionen $\sin(x)$ und $g_n(x)$ haben
nur bei $x = 0$ und $x = \pi$ eine Nullstelle.
Au{\ss}erdem gilt f\"ur alle $x \in [0,\pi]$
\\[0.2cm]
\hspace*{1.3cm}
$\sin(x) \leq 1$ \quad und \quad $g_n(x) \leq \bruch{1}{n!} \cdot \pi^n \cdot p^n$.
\\[0.2cm]
Die letzte dieser beiden Ungleichungen folgt aus der Tatsache, dass einerseits $x \leq \pi$ und andererseits
$p - q\cdot x \leq p$ ist.  Aus den beiden oberen Ungleichungen folgt durch Intergration
\\[0.2cm]
\hspace*{1.3cm}
$0 \leq I_n \leq \pi \cdot \bruch{\pi^n \cdot p^n}{n!}$
\\[0.2cm]
Nun gilt
\\[0.2cm]
\hspace*{1.3cm}
$\lim\limits_{n \rightarrow \infty} \pi \cdot \bruch{\pi^n \cdot p^n}{n!} = 0$
\\[0.2cm]
Daher gibt es ein $n \in \mathbb{N}$, so dass $\pi \cdot \bruch{\pi^n \cdot p^n}{n!} < 1$ ist und f\"ur dieses
$n$ haben wir 
\\[0.2cm]
\hspace*{1.3cm}
$0 < I_n < 1$.
\\[0.2cm]
Das ist aber ein Widerspruch dazu, dass wir oben nachgewiesen haben, dass $I_n$ eine ganze Zahl ist.  \qed

\section{Tranzendente Zahlen}
\begin{Definition}[Algebraische Zahlen] \lb
Eine Zahl $r \in \mathbb{R}$ hei{\ss}t \emph{algebraisch} genau dann, wenn es ein Polynom
\\[0.2cm]
\hspace*{1.3cm}
$p(x) = \sum\limits_{i=0}^n a_i \cdot x^i$  \quad mit $a_i \in \mathbb{Z}$ f\"ur alle $i = 0, 1, \cdots, n$
\\[0.2cm]
gibt, so dass $r$ Nullstelle des Polynoms $p$ ist, es muss also gelten
\\[0.2cm]
\hspace*{1.3cm}
$p(r) = \sum\limits_{i=0}^n a_i \cdot r^i = 0$.  \eod
\end{Definition}

Bei der obigen Definition ist die Forderung, dass die Koeffizienten $a_i$ ganze Zahlen sind, entscheidend,
denn sonst k\"onnten wir zu beliebigem $r \in \mathbb{R}$ einfach das Polyom
\\[0.2cm]
\hspace*{1.3cm}
$p_r(x) := x - r$
\\[0.2cm]
definieren und offenbar gilt $p_r(r) = 0$.  Ein solches Polyom ist zur Definition einer algebraischen
Zahl aber nur zugelassen, wenn $r$ eine ganze Zahl ist. \eod

\example
Jede rationale Zahl $r$ ist eine algebraische Zahl, denn wenn $r \in \mathbb{Q}$ ist, dann gibt es
ganze Zahlen $a$ und $b$ mit $b \not= 0$, so dass
\\[0.2cm]
\hspace*{1.3cm}
$r = \bruch{a}{b}$
\\[0.2cm]
gilt.  Damit k\"onnen wir ein Polynom $p$ als
\\[0.2cm]
\hspace*{1.3cm}
$p(x) := a - b \cdot x$
\\[0.2cm]
definieren.  F\"ur dieses Polynom gilt dann
\\[0.2cm]
\hspace*{1.3cm}
$p(r) = p\left(\bruch{a}{b}\right) = a - b \cdot \bruch{a}{b} = a - a = 0$
\\[0.2cm]
und damit ist gezeigt, dass jede rationale Zahl $r$ algebraisch ist.  Der Begriff der algebraischen
Zahlen ist also eine Verallgemeinerung des Begriffs der rationalen Zahlen.  \eod

\example
Die Zahl $\sqrt{2\,}$ ist eine algebraische Zahl, denn wenn wir das Polynom $p$ als
\\[0.2cm]
\hspace*{1.3cm}
$p(x) := x^2 - 2$
\\[0.2cm]
definieren, gilt offenbar
\\[0.2cm]
\hspace*{1.3cm}
$p\Bigl(\sqrt{2\,}\Bigr) = \Bigl(\sqrt{2\,}\Bigr)^2 - 2 = 2 - 2 = 0$.  
\\[0.2cm]
Dieses Beispiel zeigt, dass es sich bei dem Begriff der algebraischen Zahlen um eine \underline{echte} 
Verallgemeinerung des Begriffs der rationalen Zahlen handelt, denn wir haben ja bereits im ersten
Semester gesehen, dass die Zahl $\sqrt{2\,}$ keine rationale Zahl ist.
\eod

\exercises
Zeigen Sie, dass die Zahl $\sqrt{2\,} + \sqrt{3\,}$ eine algebraische Zahl ist.  Zeigen Sie au{\ss}erdem,
dass diese Zahl keine rationale Zahl ist.  \eox

\begin{Definition}[Transzendente Zahl]
  Eine Zahl $x \in \mathbb{R}$ ist genau dann \emph{transzendent}, wenn $x$ nicht algebraisch ist.
\end{Definition}

Es l\"asst sich zeigen, dass die Menge aller Polynome mit ganzzahligen Koeffizienten abz\"ahlbar ist.  Damit
ist nat\"urlich auch die Menge der algebraischen Zahlen abz\"ahlbar.  Da die Menge der reellen Zahlen
\"uberabz\"ahlbar ist, muss es also sehr viele reelle Zahlen geben, die transzendent sind.  Allerdings ist
der Nachweis der Transzendenz einer Zahl in der Regel recht aufw\"andig.  

\begin{Theorem}[Charles Hermite, 1873] \lb
  Die Eulersche Zahl $e$ ist transzendent.
\end{Theorem}

\begin{Theorem}[Ferdinand von Lindemann, 1882] \lb
  Die Kreiszahl $\pi$ ist transzendent.
\end{Theorem}
\pagebreak

\noindent
Leider bleibt in dieser Vorlesung keine Zeit mehr zum Nachweis dieser beiden Theoreme.  Unter
\\[0.2cm]
\hspace*{1.3cm}
\href{http://www.mathematik.uni-muenchen.de/~fritsch/euler.pdf}{\texttt{http://www.mathematik.uni-muenchen.de/\symbol{126}fritsch/euler.pdf}}
\\[0.2cm]
finden Sie eine Ausarbeitung des Nachweises der Transzendenz von $e$, einen Nachweis der Transzendenz
von $\pi$ finden Sie in dem folgenden Artikel von Herrn Prof.~Fritsch:
\\[0.2cm]
\hspace*{1.3cm}
\href{http://www.mathematik.uni-muenchen.de/~fritsch/pi.pdf}{\texttt{http://www.mathematik.uni-muenchen.de/\symbol{126}fritsch/pi.pdf}}.

\exercises 
Zeigen Sie, dass f\"ur alle nat\"urlichen Zahlen $n \in \mathbb{N}$
\\[0.2cm]
\hspace*{1.3cm}
$\dint{0}{\infty} t^n \cdot e^{-t} dt = n!$
\\[0.2cm]
gilt. \eox

\remark
Die letzte Gleichung motiviert die folgende Definition der
\emph{Gamma-Funktion}.  Wir setzen
\\[0.2cm]
\hspace*{1.3cm}
$\Gamma(x) := \dint{0}{\infty} t^{x-1} \cdot e^{-t} dt$.
\\[0.2cm]
Mit dieser Definition gilt
\\[0.2cm]
\hspace*{1.3cm}
$\Gamma(n+1) = n!$
\\[0.2cm]
und daher k\"onnen wir die Gamma-Funktion als eine Erweiterung der Fakult\"ats-Funktion auf
die nat\"urlichen Zahlen {auf}fassen.

%%% Local Variables: 
%%% mode: latex
%%% TeX-master: "analysis"
%%% End: 
